% ============================================================================
% SECCIÓN 03: DESCRIPCIÓN DEL DATASET
% ============================================================================

\section{Descripción del Conjunto de Datos}
\label{sec:dataset}

% NOTA: Esta sección integra la caracterización del archivo dudas_variables/reporte.tex

\subsection{Fuente de Datos}

\begin{table}[H]
\centering
\caption{Información general del conjunto de datos}
\label{tab:info_dataset}
\begin{tabular}{@{}ll@{}}
\toprule
\textbf{Característica} & \textbf{Descripción} \\
\midrule
Nombre del registro & \placeholder{Nombre del registro/base de datos} \\
Institución & \placeholder{Hospital/Centro de origen} \\
País/Región & \placeholder{País y región geográfica} \\
Período de recolección & \placeholder{Fecha inicio} -- \placeholder{Fecha fin} \\
Población objetivo & Pacientes con diagnóstico de IAM \\
Criterios de inclusión & \placeholder{Enumerar criterios de inclusión} \\
Criterios de exclusión & \placeholder{Enumerar criterios de exclusión} \\
\bottomrule
\end{tabular}
\end{table}

\subsection{Tamaño Muestral y Estructura}

\begin{table}[H]
\centering
\caption{Resumen del tamaño muestral}
\label{tab:tamano_muestral}
\begin{tabular}{@{}lr@{}}
\toprule
\textbf{Métrica} & \textbf{Valor} \\
\midrule
Número total de pacientes & \placeholder{N total} \\
Variables totales (originales) & 185 \\
Eventos de interés (mortalidad) & \placeholder{n eventos (\%)} \\
Supervivientes & \placeholder{n no eventos (\%)} \\
Tasa de eventos & \placeholder{X.XX\%} \\
\bottomrule
\end{tabular}
\end{table}

\subsection{Categorización de Variables}

El conjunto de datos contiene 185 variables organizadas en las siguientes categorías funcionales:

\subsubsection{Variables Demográficas y Administrativas}

\begin{table}[H]
\centering
\caption{Variables demográficas y administrativas}
\label{tab:vars_demograficas}
\begin{tabular}{@{}llp{6cm}@{}}
\toprule
\textbf{Variable} & \textbf{Tipo} & \textbf{Descripción} \\
\midrule
EDAD & Numérica & Edad del paciente en años \\
SEXO & Categórica & Sexo biológico (Masculino/Femenino) \\
FECHA\_INGRESO & Fecha & Fecha de admisión hospitalaria \\
FECHA\_ALTA & Fecha & Fecha de alta o fallecimiento \\
ESTANCIA & Numérica & Días de estancia hospitalaria \\
\bottomrule
\end{tabular}
\end{table}

\subsubsection{Antecedentes Patológicos}

\begin{longtable}{@{}llp{5cm}p{3cm}@{}}
\caption{Variables de antecedentes patológicos personales}
\label{tab:vars_antecedentes} \\
\toprule
\textbf{Variable} & \textbf{Tipo} & \textbf{Descripción} & \textbf{Valores posibles} \\
\midrule
\endfirsthead
\multicolumn{4}{c}{\tablename\ \thetable{} -- Continuación} \\
\toprule
\textbf{Variable} & \textbf{Tipo} & \textbf{Descripción} & \textbf{Valores posibles} \\
\midrule
\endhead
\bottomrule
\endfoot
APP\_HTA & Binaria & Hipertensión arterial & Sí/No \\
APP\_DM & Binaria & Diabetes mellitus & Sí/No \\
APP\_DISLIPIDEMIA & Binaria & Dislipidemia & Sí/No \\
APP\_TABAQUISMO & Categórica & Estado de tabaquismo & Activo/Exfumador/Nunca \\
APP\_OBESIDAD & Binaria & Obesidad (IMC $\geq$ 30) & Sí/No \\
APP\_IAM\_PREVIO & Binaria & IAM previo & Sí/No \\
APP\_ICP\_PREVIO & Binaria & Angioplastia previa & Sí/No \\
APP\_CABG\_PREVIO & Binaria & Cirugía coronaria previa & Sí/No \\
APP\_ERC & Binaria & Enfermedad renal crónica & Sí/No \\
APP\_FA & Binaria & Fibrilación auricular & Sí/No \\
APP\_ICTUS & Binaria & Ictus/ACV previo & Sí/No \\
APP\_EAP & Binaria & Enfermedad arterial periférica & Sí/No \\
\placeholder{Otras variables de antecedentes...} & & & \\
\end{longtable}

\subsubsection{Presentación Clínica}

\begin{table}[H]
\centering
\caption{Variables de presentación clínica al ingreso}
\label{tab:vars_clinicas}
\begin{tabular}{@{}llp{5cm}l@{}}
\toprule
\textbf{Variable} & \textbf{Tipo} & \textbf{Descripción} & \textbf{Unidad} \\
\midrule
PAS & Numérica & Presión arterial sistólica & mmHg \\
PAD & Numérica & Presión arterial diastólica & mmHg \\
FC & Numérica & Frecuencia cardíaca & lpm \\
FR & Numérica & Frecuencia respiratoria & rpm \\
SAT\_O2 & Numérica & Saturación de oxígeno & \% \\
KILLIP & Categórica & Clasificación Killip-Kimball & I--IV \\
DOLOR\_TIPICO & Binaria & Dolor torácico típico & Sí/No \\
TIEMPO\_SINTOMAS & Numérica & Tiempo desde inicio de síntomas & horas \\
\bottomrule
\end{tabular}
\end{table}

\subsubsection{Variables Electrocardiográficas}

\begin{table}[H]
\centering
\caption{Variables electrocardiográficas}
\label{tab:vars_ecg}
\begin{tabular}{@{}llp{6cm}@{}}
\toprule
\textbf{Variable} & \textbf{Tipo} & \textbf{Descripción} \\
\midrule
ECG\_RITMO & Categórica & Ritmo cardíaco (sinusal, FA, etc.) \\
ECG\_ELEVACION\_ST & Binaria & Presencia de elevación del ST \\
ECG\_N\_DERIVACIONES\_ST & Numérica & Número de derivaciones con elevación ST \\
ECG\_LOCALIZACION & Categórica & Localización del IAM (anterior, inferior, etc.) \\
ECG\_BRIHH & Binaria & Bloqueo de rama izquierda \\
ECG\_QTc & Numérica & Intervalo QT corregido (ms) \\
\bottomrule
\end{tabular}
\end{table}

\subsubsection{Biomarcadores de Laboratorio}

\begin{longtable}{@{}llp{4cm}ll@{}}
\caption{Biomarcadores y pruebas de laboratorio}
\label{tab:vars_laboratorio} \\
\toprule
\textbf{Variable} & \textbf{Tipo} & \textbf{Descripción} & \textbf{Unidad} & \textbf{Referencia} \\
\midrule
\endfirsthead
\multicolumn{5}{c}{\tablename\ \thetable{} -- Continuación} \\
\toprule
\textbf{Variable} & \textbf{Tipo} & \textbf{Descripción} & \textbf{Unidad} & \textbf{Referencia} \\
\midrule
\endhead
\bottomrule
\endfoot
TROPONINA\_I & Numérica & Troponina I cardíaca & ng/mL & $<$0.04 \\
TROPONINA\_T & Numérica & Troponina T cardíaca & ng/mL & $<$0.01 \\
CK\_MB & Numérica & Creatina quinasa MB & U/L & $<$25 \\
BNP & Numérica & Péptido natriurético B & pg/mL & $<$100 \\
NT\_PROBNP & Numérica & NT-proBNP & pg/mL & $<$300 \\
CREATININA & Numérica & Creatinina sérica & mg/dL & 0.7--1.3 \\
TFG\_E & Numérica & Tasa filtrado glomerular estimada & mL/min/1.73m² & $>$60 \\
GLUCOSA & Numérica & Glucemia & mg/dL & 70--100 \\
HBA1C & Numérica & Hemoglobina glicosilada & \% & $<$5.7 \\
HEMOGLOBINA & Numérica & Hemoglobina & g/dL & 12--16 \\
LEUCOCITOS & Numérica & Recuento de leucocitos & $\times 10^9$/L & 4--11 \\
PLAQUETAS & Numérica & Recuento de plaquetas & $\times 10^9$/L & 150--400 \\
COLESTEROL\_TOTAL & Numérica & Colesterol total & mg/dL & $<$200 \\
LDL & Numérica & Colesterol LDL & mg/dL & $<$100 \\
HDL & Numérica & Colesterol HDL & mg/dL & $>$40 \\
TRIGLICERIDOS & Numérica & Triglicéridos & mg/dL & $<$150 \\
PCR & Numérica & Proteína C reactiva & mg/L & $<$3 \\
\placeholder{Otras variables de laboratorio...} & & & & \\
\end{longtable}

\subsubsection{Tratamientos y Procedimientos}

\begin{table}[H]
\centering
\caption{Variables de tratamiento y procedimientos}
\label{tab:vars_tratamiento}
\begin{tabular}{@{}llp{6cm}@{}}
\toprule
\textbf{Variable} & \textbf{Tipo} & \textbf{Descripción} \\
\midrule
TTO\_FIBRINOLISIS & Binaria & Fibrinólisis administrada \\
TTO\_ICP\_PRIMARIA & Binaria & Intervención coronaria percutánea primaria \\
TIEMPO\_PUERTA\_BALON & Numérica & Tiempo puerta-balón (minutos) \\
TTO\_ASPIRINA & Binaria & Ácido acetilsalicílico \\
TTO\_CLOPIDOGREL & Binaria & Clopidogrel \\
TTO\_TICAGRELOR & Binaria & Ticagrelor \\
TTO\_HEPARINA & Binaria & Heparina \\
TTO\_BETABLOQUEANTE & Binaria & Betabloqueantes \\
TTO\_IECA\_ARA2 & Binaria & IECA o ARA2 \\
TTO\_ESTATINA & Binaria & Estatinas \\
TTO\_VENTILACION & Binaria & Ventilación mecánica \\
TTO\_VASOACTIVOS & Binaria & Fármacos vasoactivos \\
\bottomrule
\end{tabular}
\end{table}

\subsubsection{Variables de Imagen}

\begin{table}[H]
\centering
\caption{Variables de estudios de imagen}
\label{tab:vars_imagen}
\begin{tabular}{@{}llp{5cm}l@{}}
\toprule
\textbf{Variable} & \textbf{Tipo} & \textbf{Descripción} & \textbf{Unidad} \\
\midrule
ECO\_FEVI & Numérica & Fracción de eyección VI & \% \\
ECO\_ALTERACION\_SEGMENTARIA & Binaria & Alteraciones de contractilidad & Sí/No \\
ECO\_VALVULOPATIA & Binaria & Valvulopatía asociada & Sí/No \\
CORO\_N\_VASOS & Numérica & Número de vasos enfermos & 0--3 \\
CORO\_TCI & Binaria & Enfermedad del tronco coronario izquierdo & Sí/No \\
\bottomrule
\end{tabular}
\end{table}

\subsubsection{Complicaciones Intrahospitalarias}

\begin{table}[H]
\centering
\caption{Variables de complicaciones durante la hospitalización}
\label{tab:vars_complicaciones}
\begin{tabular}{@{}llp{6cm}@{}}
\toprule
\textbf{Variable} & \textbf{Tipo} & \textbf{Descripción} \\
\midrule
COMP\_ARRITMIA\_VENT & Binaria & Arritmia ventricular maligna (TV/FV) \\
COMP\_BAV & Binaria & Bloqueo auriculoventricular \\
COMP\_ICC & Binaria & Insuficiencia cardíaca congestiva \\
COMP\_SHOCK & Binaria & Shock cardiogénico \\
COMP\_REINFARTO & Binaria & Reinfarto intrahospitalario \\
COMP\_SANGRADO & Binaria & Sangrado mayor (criterios BARC) \\
COMP\_IRA & Binaria & Injuria renal aguda \\
COMP\_ICTUS & Binaria & Ictus intrahospitalario \\
\bottomrule
\end{tabular}
\end{table}

\subsubsection{Variables Outcome}

\begin{table}[H]
\centering
\caption{Variables de desenlace (outcome)}
\label{tab:vars_outcome}
\begin{tabular}{@{}llp{6cm}@{}}
\toprule
\textbf{Variable} & \textbf{Tipo} & \textbf{Descripción} \\
\midrule
\rowcolor{accentgold!20}
MORTALIDAD\_HOSP & Binaria & \textbf{Variable objetivo principal}: Mortalidad intrahospitalaria \\
CAUSA\_MUERTE & Categórica & Causa de muerte (cardiogénica, arrítmica, etc.) \\
DESTINO\_ALTA & Categórica & Destino al alta (domicilio, otro hospital, etc.) \\
\bottomrule
\end{tabular}
\end{table}

\subsection{Consideraciones sobre Calidad de Datos}

\subsubsection{Datos Faltantes}

\begin{placeholderblock}
\textbf{[COMPLETAR CON ANÁLISIS REAL]}

Incluir aquí:
\begin{itemize}
    \item Tabla o figura con porcentaje de datos faltantes por variable
    \item Identificación de patrones de missingness (MCAR, MAR, MNAR)
    \item Variables con $>$20\% de valores faltantes que requieren tratamiento especial
\end{itemize}

Ejemplo de presentación:
\begin{verbatim}
Variable           | % Faltantes | Patrón estimado
-------------------|-------------|----------------
NT_PROBNP          |    35.2%    | MAR
TIEMPO_PUERTA_BALON|    28.7%    | MNAR (no ICP)
ECO_FEVI           |    15.3%    | MAR
\end{verbatim}
\end{placeholderblock}

\subsubsection{Valores Atípicos Identificados}

\begin{placeholderblock}
\textbf{[COMPLETAR CON ANÁLISIS REAL]}

Documentar:
\begin{itemize}
    \item Variables con valores fuera de rangos fisiológicos
    \item Estrategia de manejo de outliers
    \item Número de observaciones afectadas
\end{itemize}
\end{placeholderblock}

\subsection{Resumen Descriptivo Preliminar}

\begin{placeholderblock}
\textbf{[INSERTAR TABLA RESUMEN]}

Tabla con estadísticos descriptivos básicos:
\begin{itemize}
    \item Variables numéricas: media ± DE, mediana [IQR], rango
    \item Variables categóricas: frecuencia (porcentaje)
    \item Comparación entre fallecidos vs. supervivientes (test t, U de Mann-Whitney, $\chi^2$)
\end{itemize}

\textbf{Formato sugerido:}
\begin{verbatim}
| Variable      | Total (N=xxx) | Fallecidos (n=xx) | Supervivientes (n=xxx) | p-valor |
|---------------|---------------|-------------------|------------------------|---------|
| Edad, años    | xx.x ± xx.x   | xx.x ± xx.x       | xx.x ± xx.x            | <0.001  |
| Sexo masculino| xxx (xx.x%)   | xx (xx.x%)        | xxx (xx.x%)            | 0.xxx   |
\end{verbatim}
\end{placeholderblock}

\subsection{Consideraciones Éticas y de Privacidad}

\begin{itemize}
    \item \textbf{Aprobación ética}: \placeholder{Número/referencia del comité de ética}.
    \item \textbf{Consentimiento informado}: \placeholder{Tipo de consentimiento obtenido}.
    \item \textbf{Anonimización}: Todos los datos fueron desidentificados según normativa \placeholder{LOPD/RGPD/equivalente local}.
    \item \textbf{Almacenamiento}: Datos almacenados en servidores seguros con acceso restringido.
\end{itemize}
