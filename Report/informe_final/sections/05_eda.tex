% ============================================================================
% SECCIÓN 05: ANÁLISIS EXPLORATORIO DE DATOS (EDA)
% ============================================================================

\section{Análisis Exploratorio de Datos}
\label{sec:eda}

El \gls{eda} constituye una fase fundamental para comprender la estructura, calidad y distribución de los datos antes del modelado predictivo.

\subsection{Distribución de la Variable Objetivo}

\begin{figure}[H]
\centering
\begin{placeholderblock}
\textbf{[INSERTAR GRÁFICO DE BARRAS]}

Gráfico de barras mostrando:
\begin{itemize}
    \item Número de supervivientes vs. fallecidos
    \item Porcentajes correspondientes
    \item Destacar el desbalance de clases
\end{itemize}

\textit{Código sugerido: plt.bar(['Supervivientes', 'Fallecidos'], [n\_surv, n\_fall])}
\end{placeholderblock}
\caption{Distribución de la variable objetivo (mortalidad intrahospitalaria)}
\label{fig:distribucion_outcome}
\end{figure}

\begin{keypoint}
\textbf{Hallazgo clave:} La tasa de mortalidad intrahospitalaria fue del \placeholder{X.XX\%} (n=\placeholder{XX} de \placeholder{N} pacientes), confirmando el desbalance de clases característico de este tipo de estudios.
\end{keypoint}

\subsection{Análisis de Variables Demográficas}

\subsubsection{Distribución de Edad}

\begin{figure}[H]
\centering
\begin{placeholderblock}
\textbf{[INSERTAR HISTOGRAMA/BOXPLOT DE EDAD]}

Panel con:
\begin{itemize}
    \item Histograma de edad con curva de densidad
    \item Boxplot comparando edad entre fallecidos y supervivientes
    \item Estadísticos: media, mediana, DE, rango
\end{itemize}
\end{placeholderblock}
\caption{Distribución de edad en la población de estudio}
\label{fig:distribucion_edad}
\end{figure}

\begin{table}[H]
\centering
\caption{Estadísticos descriptivos de edad}
\label{tab:estadisticos_edad}
\begin{tabular}{@{}lcccc@{}}
\toprule
\textbf{Grupo} & \textbf{N} & \textbf{Media ± DE} & \textbf{Mediana [IQR]} & \textbf{Rango} \\
\midrule
Total & \placeholder{N} & \placeholder{XX.X ± XX.X} & \placeholder{XX [XX--XX]} & \placeholder{XX--XX} \\
Supervivientes & \placeholder{n} & \placeholder{XX.X ± XX.X} & \placeholder{XX [XX--XX]} & \placeholder{XX--XX} \\
Fallecidos & \placeholder{n} & \placeholder{XX.X ± XX.X} & \placeholder{XX [XX--XX]} & \placeholder{XX--XX} \\
\midrule
\multicolumn{5}{l}{\textit{p-valor (test t/Mann-Whitney): \placeholder{p = X.XXX}}} \\
\bottomrule
\end{tabular}
\end{table}

\subsubsection{Distribución por Sexo}

\begin{figure}[H]
\centering
\begin{placeholderblock}
\textbf{[INSERTAR GRÁFICO DE BARRAS AGRUPADAS]}

Gráfico mostrando:
\begin{itemize}
    \item Distribución de sexo en toda la cohorte
    \item Distribución de sexo según outcome
    \item Porcentajes y frecuencias
\end{itemize}
\end{placeholderblock}
\caption{Distribución por sexo y asociación con mortalidad}
\label{fig:distribucion_sexo}
\end{figure}

\subsection{Análisis de Antecedentes Patológicos}

\begin{figure}[H]
\centering
\begin{placeholderblock}
\textbf{[INSERTAR GRÁFICO DE BARRAS HORIZONTALES]}

Gráfico de barras horizontales mostrando prevalencia de antecedentes:
\begin{itemize}
    \item Hipertensión
    \item Diabetes
    \item Dislipidemia
    \item Tabaquismo
    \item IAM previo
    \item Enfermedad renal crónica
    \item etc.
\end{itemize}

Preferiblemente separado por outcome para visualizar diferencias.
\end{placeholderblock}
\caption{Prevalencia de antecedentes patológicos personales}
\label{fig:prevalencia_app}
\end{figure}

\begin{table}[H]
\centering
\caption{Comparación de antecedentes patológicos entre grupos}
\label{tab:comparacion_app}
\begin{tabular}{@{}lccc@{}}
\toprule
\textbf{Antecedente} & \textbf{Supervivientes n(\%)} & \textbf{Fallecidos n(\%)} & \textbf{p-valor} \\
\midrule
Hipertensión & \placeholder{n (XX.X\%)} & \placeholder{n (XX.X\%)} & \placeholder{X.XXX} \\
Diabetes mellitus & \placeholder{n (XX.X\%)} & \placeholder{n (XX.X\%)} & \placeholder{X.XXX} \\
Dislipidemia & \placeholder{n (XX.X\%)} & \placeholder{n (XX.X\%)} & \placeholder{X.XXX} \\
Tabaquismo activo & \placeholder{n (XX.X\%)} & \placeholder{n (XX.X\%)} & \placeholder{X.XXX} \\
IAM previo & \placeholder{n (XX.X\%)} & \placeholder{n (XX.X\%)} & \placeholder{X.XXX} \\
ICP previa & \placeholder{n (XX.X\%)} & \placeholder{n (XX.X\%)} & \placeholder{X.XXX} \\
ERC & \placeholder{n (XX.X\%)} & \placeholder{n (XX.X\%)} & \placeholder{X.XXX} \\
Fibrilación auricular & \placeholder{n (XX.X\%)} & \placeholder{n (XX.X\%)} & \placeholder{X.XXX} \\
\bottomrule
\end{tabular}
\end{table}

\subsection{Análisis de Variables Clínicas de Presentación}

\subsubsection{Signos Vitales al Ingreso}

\begin{figure}[H]
\centering
\begin{placeholderblock}
\textbf{[INSERTAR GRID DE BOXPLOTS]}

Panel 2x3 o similar con boxplots comparativos (supervivientes vs. fallecidos):
\begin{itemize}
    \item Presión arterial sistólica
    \item Presión arterial diastólica
    \item Frecuencia cardíaca
    \item Frecuencia respiratoria
    \item Saturación de oxígeno
    \item Temperatura (si disponible)
\end{itemize}
\end{placeholderblock}
\caption{Distribución de signos vitales al ingreso según outcome}
\label{fig:signos_vitales}
\end{figure}

\subsubsection{Clasificación Killip al Ingreso}

\begin{figure}[H]
\centering
\begin{placeholderblock}
\textbf{[INSERTAR GRÁFICO DE BARRAS APILADAS]}

Distribución de clases Killip I--IV:
\begin{itemize}
    \item En toda la cohorte
    \item Estratificada por outcome
    \item Mostrar gradiente de mortalidad por clase Killip
\end{itemize}
\end{placeholderblock}
\caption{Distribución de clasificación Killip-Kimball y asociación con mortalidad}
\label{fig:killip}
\end{figure}

\begin{table}[H]
\centering
\caption{Mortalidad según clasificación Killip}
\label{tab:mortalidad_killip}
\begin{tabular}{@{}lcccc@{}}
\toprule
\textbf{Clase Killip} & \textbf{N pacientes} & \textbf{Fallecidos} & \textbf{Mortalidad (\%)} & \textbf{IC 95\%} \\
\midrule
I (sin IC) & \placeholder{N} & \placeholder{n} & \placeholder{X.X\%} & \placeholder{[X.X--X.X]} \\
II (IC leve) & \placeholder{N} & \placeholder{n} & \placeholder{X.X\%} & \placeholder{[X.X--X.X]} \\
III (EAP) & \placeholder{N} & \placeholder{n} & \placeholder{X.X\%} & \placeholder{[X.X--X.X]} \\
IV (Shock) & \placeholder{N} & \placeholder{n} & \placeholder{X.X\%} & \placeholder{[X.X--X.X]} \\
\bottomrule
\end{tabular}
\end{table}

\subsection{Análisis de Biomarcadores}

\subsubsection{Troponinas y Marcadores de Necrosis}

\begin{figure}[H]
\centering
\begin{placeholderblock}
\textbf{[INSERTAR BOXPLOTS DE TROPONINAS]}

Comparación de niveles de:
\begin{itemize}
    \item Troponina I o T
    \item CK-MB
\end{itemize}
Entre supervivientes y fallecidos (usar escala logarítmica si hay valores extremos).
\end{placeholderblock}
\caption{Distribución de troponinas según outcome}
\label{fig:troponinas}
\end{figure}

\subsubsection{Biomarcadores Renales y Metabólicos}

\begin{figure}[H]
\centering
\begin{placeholderblock}
\textbf{[INSERTAR PANEL DE DISTRIBUCIONES]}

Variables a mostrar:
\begin{itemize}
    \item Creatinina
    \item TFG estimada
    \item Glucemia
    \item HbA1c (si disponible)
\end{itemize}
\end{placeholderblock}
\caption{Distribución de marcadores renales y metabólicos}
\label{fig:renales_metabolicos}
\end{figure}

\subsubsection{Péptidos Natriuréticos}

\begin{figure}[H]
\centering
\begin{placeholderblock}
\textbf{[INSERTAR BOXPLOT/VIOLINPLOT]}

BNP o NT-proBNP comparando grupos.
Mostrar transformación logarítmica si es necesario.
\end{placeholderblock}
\caption{Distribución de péptidos natriuréticos según outcome}
\label{fig:peptidos_natriureticos}
\end{figure}

\subsection{Análisis de Correlaciones}

\begin{figure}[H]
\centering
\begin{placeholderblock}
\textbf{[INSERTAR HEATMAP DE CORRELACIONES]}

Mapa de calor mostrando correlaciones de Pearson/Spearman entre variables numéricas principales (seleccionar ~20--30 variables más relevantes).

Incluir:
\begin{itemize}
    \item Escala de colores divergente (azul-blanco-rojo)
    \item Valores de correlación en celdas
    \item Clustering jerárquico opcional
\end{itemize}
\end{placeholderblock}
\caption{Matriz de correlaciones entre variables numéricas}
\label{fig:correlaciones}
\end{figure}

\begin{keypoint}
\textbf{Correlaciones relevantes identificadas:}
\begin{itemize}
    \item \placeholder{Variable\_1 vs. Variable\_2: r = X.XX}
    \item \placeholder{Variable\_3 vs. Variable\_4: r = X.XX}
    \item Nota sobre multicolinealidad si aplica
\end{itemize}
\end{keypoint}

\subsection{Análisis de Datos Faltantes}

\begin{figure}[H]
\centering
\begin{placeholderblock}
\textbf{[INSERTAR GRÁFICO DE MISSINGNESS]}

Opciones de visualización:
\begin{itemize}
    \item Barras horizontales con \% de valores faltantes por variable
    \item Heatmap de patrón de missingness (missingno library)
    \item Matriz de correlación de missingness
\end{itemize}
\end{placeholderblock}
\caption{Patrón de datos faltantes}
\label{fig:datos_faltantes}
\end{figure}

\begin{table}[H]
\centering
\caption{Variables con mayor porcentaje de datos faltantes}
\label{tab:datos_faltantes_top}
\begin{tabular}{@{}lcl@{}}
\toprule
\textbf{Variable} & \textbf{\% Faltantes} & \textbf{Posible causa} \\
\midrule
\placeholder{Variable\_1} & \placeholder{XX.X\%} & \placeholder{Motivo} \\
\placeholder{Variable\_2} & \placeholder{XX.X\%} & \placeholder{Motivo} \\
\placeholder{Variable\_3} & \placeholder{XX.X\%} & \placeholder{Motivo} \\
\placeholder{Variable\_4} & \placeholder{XX.X\%} & \placeholder{Motivo} \\
\placeholder{Variable\_5} & \placeholder{XX.X\%} & \placeholder{Motivo} \\
\bottomrule
\end{tabular}
\end{table}

\subsection{Análisis de Tratamientos y Procedimientos}

\begin{figure}[H]
\centering
\begin{placeholderblock}
\textbf{[INSERTAR GRÁFICO DE BARRAS]}

Mostrar frecuencia de:
\begin{itemize}
    \item Fibrinólisis vs. ICP primaria vs. conservador
    \item Tratamientos farmacológicos (antiagregantes, anticoagulantes, etc.)
\end{itemize}
Estratificado por outcome si es informativo.
\end{placeholderblock}
\caption{Distribución de estrategias de reperfusión y tratamientos}
\label{fig:tratamientos}
\end{figure}

\subsection{Análisis Temporal}

\begin{figure}[H]
\centering
\begin{placeholderblock}
\textbf{[INSERTAR GRÁFICO DE LÍNEA/BARRAS TEMPORALES]}

Si aplica:
\begin{itemize}
    \item Número de casos por mes/año
    \item Evolución de la mortalidad en el tiempo
    \item Cambios en prácticas de tratamiento
\end{itemize}
\end{placeholderblock}
\caption{Distribución temporal de casos y mortalidad}
\label{fig:analisis_temporal}
\end{figure}

\subsection{Resumen de Hallazgos del EDA}

\begin{keypoint}
\textbf{Principales hallazgos del análisis exploratorio:}

\begin{enumerate}
    \item \textbf{Desbalance de clases}: Mortalidad del \placeholder{X\%}, confirmando necesidad de técnicas de balanceo.
    
    \item \textbf{Variables con asociación univariada significativa}: 
    \begin{itemize}
        \item \placeholder{Edad}
        \item \placeholder{Clasificación Killip}
        \item \placeholder{Creatinina/TFG}
        \item \placeholder{Otras...}
    \end{itemize}
    
    \item \textbf{Datos faltantes}: \placeholder{N} variables con $>$20\% de datos faltantes, principalmente \placeholder{categoría de variables}.
    
    \item \textbf{Multicolinealidad}: Identificada entre \placeholder{pares de variables}.
    
    \item \textbf{Valores atípicos}: Detectados en \placeholder{variables}, tratados mediante \placeholder{estrategia}.
\end{enumerate}
\end{keypoint}
