% ============================================================================
% APÉNDICE C: CÓDIGO Y RECURSOS
% ============================================================================

\section{Código y Recursos}
\label{app:codigo}

Este apéndice proporciona información sobre el código fuente, recursos y herramientas utilizadas en el proyecto.

\subsection{Estructura del Repositorio}

\begin{verbatim}
mortality-ami-predictor/
+-- DATA/                       # Datos (no incluidos en repositorio publico)
|   +-- raw/                    # Datos crudos
|   +-- processed/              # Datos procesados
+-- Report/                     # Documentacion e informes
|   +-- informe_final/          # Este informe
|   +-- state_of_the_art/       # Estado del arte
+-- Tools/                      # Herramientas principales
|   +-- src/                    # Codigo fuente
|   |   +-- data_load/          # Carga de datos
|   |   +-- cleaning/           # Limpieza y preprocesamiento
|   |   +-- eda/                # Analisis exploratorio
|   |   +-- automl/             # AutoML
|   |   +-- evaluation/         # Evaluacion de modelos
|   |   +-- explainability/     # Explicabilidad (SHAP)
|   +-- dashboard/              # Aplicacion Streamlit
|   +-- notebooks/              # Jupyter notebooks
|   +-- models/                 # Modelos entrenados
|   +-- tests/                  # Tests unitarios
|   +-- docs/                   # Documentacion tecnica
+-- requirements.txt            # Dependencias Python
+-- environment.yml             # Entorno Conda
+-- README.md                   # Documentacion principal
\end{verbatim}

\subsection{Acceso al Código}

\begin{table}[H]
\centering
\caption{Enlaces a recursos del proyecto}
\label{tab:enlaces_recursos}
\begin{tabular}{@{}ll@{}}
\toprule
\textbf{Recurso} & \textbf{Enlace} \\
\midrule
Repositorio GitHub & \placeholder{https://github.com/usuario/mortality-ami-predictor} \\
Documentación & \placeholder{https://usuario.github.io/mortality-ami-predictor} \\
Dashboard (demo) & \placeholder{https://mortality-ami-predictor.streamlit.app} \\
MLflow experiments & \placeholder{Enlace o ``Local''} \\
\bottomrule
\end{tabular}
\end{table}

\subsection{Dependencias Principales}

\begin{table}[H]
\centering
\caption{Principales librerías Python utilizadas}
\label{tab:dependencias}
\begin{tabular}{@{}llp{6cm}@{}}
\toprule
\textbf{Librería} & \textbf{Versión} & \textbf{Uso} \\
\midrule
\multicolumn{3}{l}{\textbf{Manipulación de datos}} \\
pandas & \placeholder{2.x} & DataFrames y manipulación tabular \\
numpy & \placeholder{1.2x} & Arrays y operaciones numéricas \\
\midrule
\multicolumn{3}{l}{\textbf{Visualización}} \\
matplotlib & \placeholder{3.x} & Gráficos base \\
seaborn & \placeholder{0.1x} & Visualización estadística \\
plotly & \placeholder{5.x} & Gráficos interactivos \\
\midrule
\multicolumn{3}{l}{\textbf{Machine Learning}} \\
scikit-learn & \placeholder{1.x} & ML clásico, preprocesamiento, evaluación \\
xgboost & \placeholder{2.x} & Gradient boosting \\
lightgbm & \placeholder{4.x} & Gradient boosting eficiente \\
imbalanced-learn & \placeholder{0.1x} & SMOTE y técnicas de balanceo \\
\midrule
\multicolumn{3}{l}{\textbf{Deep Learning (si aplica)}} \\
tensorflow/keras & \placeholder{2.x} & Redes neuronales \\
autokeras & \placeholder{1.x} & AutoML para deep learning \\
\midrule
\multicolumn{3}{l}{\textbf{Explicabilidad}} \\
shap & \placeholder{0.4x} & Valores SHAP \\
eli5 & \placeholder{0.1x} & Explicación de modelos \\
\midrule
\multicolumn{3}{l}{\textbf{Aplicación}} \\
streamlit & \placeholder{1.x} & Dashboard interactivo \\
mlflow & \placeholder{2.x} & Tracking de experimentos \\
joblib & \placeholder{1.x} & Serialización de modelos \\
\bottomrule
\end{tabular}
\end{table}

\subsection{Instalación}

\begin{lstlisting}[language=bash, caption=Instalacion del entorno]
# Clonar repositorio
git clone https://github.com/usuario/mortality-ami-predictor.git
cd mortality-ami-predictor

# Opcion 1: Usando pip
pip install -r requirements.txt

# Opcion 2: Usando conda
conda env create -f environment.yml
conda activate mortality-ami

# Opcion 3: Usando Docker
docker-compose up -d
\end{lstlisting}

\subsection{Uso Básico}

\subsubsection{Entrenamiento de Modelos}

\begin{lstlisting}[language=Python, caption=Ejemplo de entrenamiento]
from src.data_load import load_data
from src.cleaning import preprocess_pipeline
from src.automl import train_models, evaluate_models

# Cargar datos
df = load_data("DATA/recuima-020425.csv")

# Preprocesamiento
X, y = preprocess_pipeline(df, target="MORTALIDAD_HOSP")

# Entrenar modelos
results = train_models(X, y, models=["lr", "rf", "xgb"])

# Evaluar
metrics = evaluate_models(results, X_test, y_test)
\end{lstlisting}

\subsubsection{Predicción con Modelo Entrenado}

\begin{lstlisting}[language=Python, caption=Ejemplo de prediccion]
import joblib
from src.explainability import explain_prediction

# Cargar modelo
model = joblib.load("models/best_classifier.joblib")
preprocessor = joblib.load("models/preprocessor.joblib")

# Preparar datos del paciente
patient_data = {
    "EDAD": 72,
    "SEXO": 1,
    "KILLIP": 2,
    "PAS": 110,
    "CREATININA": 1.5,
    # ... otras variables
}

# Preprocesar y predecir
X_patient = preprocessor.transform(patient_data)
prob = model.predict_proba(X_patient)[0, 1]

# Explicar prediccion
shap_values = explain_prediction(model, X_patient)
\end{lstlisting}

\subsubsection{Ejecutar Dashboard}

\begin{lstlisting}[language=bash, caption=Ejecucion del dashboard]
cd Tools/dashboard
streamlit run Dashboard.py
\end{lstlisting}

\subsection{Notebooks de Referencia}

\begin{table}[H]
\centering
\caption{Notebooks disponibles}
\label{tab:notebooks}
\begin{tabular}{@{}lp{8cm}@{}}
\toprule
\textbf{Notebook} & \textbf{Contenido} \\
\midrule
eda.ipynb & Análisis exploratorio completo con visualizaciones \\
modeling.ipynb & Desarrollo y comparación de modelos \\
explainability.ipynb & Análisis SHAP y explicabilidad \\
\bottomrule
\end{tabular}
\end{table}

\subsection{Tests}

\begin{lstlisting}[language=bash, caption=Ejecucion de tests]
# Ejecutar todos los tests
pytest tests/ -v

# Con cobertura
pytest tests/ --cov=src --cov-report=html
\end{lstlisting}

\subsection{Documentacion Tecnica}

La documentacion tecnica completa esta disponible en el directorio \texttt{docs/} y puede ser generada localmente:

\begin{lstlisting}[language=bash, caption=Generacion de documentacion]
# Instalar dependencias de documentacion
pip install -r docs-requirements.txt

# Generar documentacion
mkdocs serve  # Para desarrollo
mkdocs build  # Para produccion
\end{lstlisting}

\subsection{Licencia}

\begin{placeholderblock}[ESPECIFICAR LICENCIA]
Este proyecto esta licenciado bajo \placeholder{MIT License / Apache 2.0 / Creative Commons}.

Ver archivo LICENSE en el repositorio para detalles completos.
\end{placeholderblock}

\subsection{Citacion}

Si utilizas este codigo o metodologia en tu investigacion, por favor cita:

\begin{verbatim}
@software{mortalityami2025,
  author = {[AUTORES]},
  title = {Mortality AMI Predictor: Machine Learning para 
           Prediccion de Mortalidad Intrahospitalaria en IAM},
  year = {2025},
  url = {https://github.com/usuario/mortality-ami-predictor}
}
\end{verbatim}

\subsection{Contacto y Soporte}

\begin{placeholderblock}[INFORMACION DE CONTACTO]
Para preguntas, sugerencias o colaboraciones:
\begin{itemize}
    \item Email: \placeholder{equipo@ejemplo.com}
    \item Issues: \placeholder{GitHub Issues del repositorio}
    \item Discusiones: \placeholder{GitHub Discussions}
\end{itemize}
\end{placeholderblock}
