% ============================================================================
% SECCIÓN 00: RESUMEN Y ABSTRACT
% ============================================================================

\begin{abstract}
\noindent
\textbf{Contexto:} El infarto agudo de miocardio (IAM) representa una de las principales causas de mortalidad cardiovascular a nivel mundial. La predicción temprana de la mortalidad intrahospitalaria permite optimizar la estratificación de riesgo y la asignación de recursos terapéuticos.

\vspace{0.3cm}
\noindent
\textbf{Objetivo:} Desarrollar y validar modelos de aprendizaje automático para predecir la mortalidad intrahospitalaria en pacientes con IAM utilizando datos del Registro Cubano de Infarto Agudo de Miocardio (RECUIMA).

\vspace{0.3cm}
\noindent
\textbf{Métodos:} Se analizaron \textbf{3.112} pacientes con diagnóstico de IAM registrados en el RECUIMA. Se implementaron múltiples algoritmos de clasificación incluyendo Regresión Logística, Random Forest, XGBoost, LightGBM y K-Nearest Neighbors. El rendimiento se evaluó mediante validación cruzada estratificada y métricas de discriminación (AUROC, AUPRC), calibración y utilidad clínica. Se desarrollaron dos enfoques: un modelo reducido (10 variables) comparable con la escala GRACE y un modelo extendido (57 variables) como propuesta original.

\vspace{0.3cm}
\noindent
\textbf{Resultados:} El modelo con mejor rendimiento fue \textbf{XGBoost con el conjunto extendido de variables}, alcanzando un AUROC de \textbf{0,938 (IC 95\%: 0,884--0,977)} en el conjunto de prueba. El modelo reducido, comparable con GRACE, obtuvo un AUROC de \textbf{0,901 (IC 95\%: 0,855--0,937)}. Las variables más predictivas identificadas mediante análisis SHAP fueron \textbf{fracción de eyección, edad, filtrado glomerular, frecuencia cardíaca y presión arterial diastólica}. Ambos modelos superaron el rendimiento de las escalas clásicas GRACE (AUROC estimado: 0,820) en esta población.

\vspace{0.3cm}
\noindent
\textbf{Conclusiones:} Los modelos de aprendizaje automático demuestran capacidad superior para predecir mortalidad intrahospitalaria por IAM en comparación con escalas tradicionales. La interpretabilidad mediante SHAP facilita la adopción clínica al identificar los factores de riesgo más relevantes.

\vspace{0.5cm}
\noindent
\textbf{Palabras clave:} infarto agudo de miocardio, mortalidad intrahospitalaria, aprendizaje automático, predicción de riesgo, SHAP, XGBoost

\end{abstract}

\newpage

% ============================================================================
% ABSTRACT EN INGLÉS
% ============================================================================

\begin{center}
{\Large\bfseries\color{primaryblue} Abstract}
\end{center}

\vspace{0.3cm}
\noindent
\textbf{Background:} Acute myocardial infarction (AMI) remains one of the leading causes of cardiovascular mortality worldwide. Early prediction of in-hospital mortality enables optimal risk stratification and therapeutic resource allocation.

\vspace{0.3cm}
\noindent
\textbf{Objective:} To develop and validate machine learning models for predicting in-hospital mortality in AMI patients using data from the Cuban Registry of Acute Myocardial Infarction (RECUIMA).

\vspace{0.3cm}
\noindent
\textbf{Methods:} We analyzed \textbf{3,112} patients diagnosed with AMI from RECUIMA. Multiple classification algorithms were implemented, including Logistic Regression, Random Forest, XGBoost, LightGBM and K-Nearest Neighbors. Performance was evaluated using stratified cross-validation with discrimination metrics (AUROC, AUPRC), calibration, and clinical utility. Two approaches were developed: a reduced model (10 variables) comparable with GRACE score and an extended model (57 variables) as our original proposal.

\vspace{0.3cm}
\noindent
\textbf{Results:} The best-performing model was \textbf{XGBoost with extended variable set}, achieving an AUROC of \textbf{0.938 (95\%CI: 0.884--0.977)} on the test set. The reduced model, comparable with GRACE, achieved an AUROC of \textbf{0.901 (95\%CI: 0.855--0.937)}. The most predictive variables identified through SHAP analysis were \textbf{ejection fraction, age, glomerular filtration rate, heart rate, and diastolic blood pressure}. Both models outperformed traditional GRACE (estimated AUROC: 0.820) scores in this population.

\vspace{0.3cm}
\noindent
\textbf{Conclusions:} Machine learning models demonstrate superior capability for predicting in-hospital mortality from AMI compared to traditional scores. Interpretability through SHAP facilitates clinical adoption by identifying the most relevant risk factors.

\vspace{0.5cm}
\noindent
\textbf{Keywords:} acute myocardial infarction, in-hospital mortality, machine learning, risk prediction, SHAP, XGBoost

\newpage
