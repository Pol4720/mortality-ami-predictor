% ============================================================================
% SECCIÓN 00: RESUMEN Y ABSTRACT
% ============================================================================

\begin{abstract}
\noindent
\textbf{Contexto:} El infarto agudo de miocardio (IAM) representa una de las principales causas de mortalidad cardiovascular a nivel mundial. La predicción temprana de la mortalidad intrahospitalaria permite optimizar la estratificación de riesgo y la asignación de recursos terapéuticos.

\vspace{0.3cm}
\noindent
\textbf{Objetivo:} Desarrollar y validar modelos de aprendizaje automático para predecir la mortalidad intrahospitalaria en pacientes con IAM utilizando datos del Registro Cubano de Infarto Agudo de Miocardio (RECUIMA).

\vspace{0.3cm}
\noindent
\textbf{Métodos:} Se analizaron \textbf{[PLACEHOLDER: N]} pacientes con diagnóstico de IAM registrados entre \textbf{[PLACEHOLDER: período]}. Se implementaron múltiples algoritmos de clasificación incluyendo Regresión Logística, Random Forest, XGBoost, y Redes Neuronales. El rendimiento se evaluó mediante validación cruzada estratificada y métricas de discriminación (AUROC, AUPRC), calibración y utilidad clínica.

\vspace{0.3cm}
\noindent
\textbf{Resultados:} El modelo con mejor rendimiento fue \textbf{[PLACEHOLDER: nombre del modelo]}, alcanzando un AUROC de \textbf{[PLACEHOLDER: valor $\pm$ IC95\%]} en el conjunto de prueba. Las variables más predictivas identificadas mediante análisis SHAP fueron \textbf{[PLACEHOLDER: top 5 variables]}. El modelo superó el rendimiento de las escalas clásicas GRACE y TIMI en esta población.

\vspace{0.3cm}
\noindent
\textbf{Conclusiones:} Los modelos de aprendizaje automático demuestran capacidad superior para predecir mortalidad intrahospitalaria por IAM en comparación con escalas tradicionales. La interpretabilidad mediante SHAP facilita la adopción clínica al identificar los factores de riesgo más relevantes.

\vspace{0.5cm}
\noindent
\textbf{Palabras clave:} infarto agudo de miocardio, mortalidad intrahospitalaria, aprendizaje automático, predicción de riesgo, SHAP, XGBoost

\end{abstract}

\newpage

% ============================================================================
% ABSTRACT EN INGLÉS
% ============================================================================

\begin{center}
{\Large\bfseries\color{primaryblue} Abstract}
\end{center}

\vspace{0.3cm}
\noindent
\textbf{Background:} Acute myocardial infarction (AMI) remains one of the leading causes of cardiovascular mortality worldwide. Early prediction of in-hospital mortality enables optimal risk stratification and therapeutic resource allocation.

\vspace{0.3cm}
\noindent
\textbf{Objective:} To develop and validate machine learning models for predicting in-hospital mortality in AMI patients using data from the Cuban Registry of Acute Myocardial Infarction (RECUIMA).

\vspace{0.3cm}
\noindent
\textbf{Methods:} We analyzed \textbf{[PLACEHOLDER: N]} patients diagnosed with AMI between \textbf{[PLACEHOLDER: period]}. Multiple classification algorithms were implemented, including Logistic Regression, Random Forest, XGBoost, and Neural Networks. Performance was evaluated using stratified cross-validation with discrimination metrics (AUROC, AUPRC), calibration, and clinical utility.

\vspace{0.3cm}
\noindent
\textbf{Results:} The best-performing model was \textbf{[PLACEHOLDER: model name]}, achieving an AUROC of \textbf{[PLACEHOLDER: value $\pm$ 95\%CI]} on the test set. The most predictive variables identified through SHAP analysis were \textbf{[PLACEHOLDER: top 5 variables]}. The model outperformed traditional GRACE and TIMI scores in this population.

\vspace{0.3cm}
\noindent
\textbf{Conclusions:} Machine learning models demonstrate superior capability for predicting in-hospital mortality from AMI compared to traditional scores. Interpretability through SHAP facilitates clinical adoption by identifying the most relevant risk factors.

\vspace{0.5cm}
\noindent
\textbf{Keywords:} acute myocardial infarction, in-hospital mortality, machine learning, risk prediction, SHAP, XGBoost

\newpage
