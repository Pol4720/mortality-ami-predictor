% ============================================================================
% SECCIÓN 02: ESTADO DEL ARTE
% ============================================================================

\section{Estado del Arte}
\label{sec:estado_del_arte}

% NOTA: Esta sección integra y expande el contenido del archivo 
% estado_del_arte_ml_infarcto.tex proporcionado

\subsection{Modelos Clásicos de Predicción de Mortalidad en IAM}

Las escalas de riesgo clásicas constituyen la base sobre la que se ha desarrollado la predicción pronóstica en el \gls{iam}. Entre las más utilizadas destacan:

\subsubsection{Escala GRACE}

El \textit{Global Registry of Acute Coronary Events} \citep{fox2006grace} representa el estándar de referencia actual para la estratificación de riesgo en síndromes coronarios agudos. La escala GRACE incorpora las siguientes variables:

\begin{table}[H]
\centering
\caption{Variables incluidas en la escala GRACE}
\label{tab:grace_variables}
\begin{tabular}{@{}lll@{}}
\toprule
\textbf{Variable} & \textbf{Tipo} & \textbf{Rango de puntos} \\
\midrule
Edad & Continua & 0--100 \\
Frecuencia cardíaca & Continua & 0--46 \\
Presión arterial sistólica & Continua & 0--58 \\
Creatinina sérica & Continua & 0--28 \\
Clase Killip & Categórica & 0--59 \\
Parada cardiorrespiratoria al ingreso & Binaria & 0--39 \\
Desviación del segmento ST & Binaria & 0--28 \\
Elevación de biomarcadores & Binaria & 0--14 \\
\bottomrule
\end{tabular}
\end{table}

\subsubsection{Escala TIMI}

La escala \textit{Thrombolysis In Myocardial Infarction} \citep{antman2000timi} ofrece una alternativa más sencilla, derivada de ensayos clínicos controlados:

\begin{table}[H]
\centering
\caption{Variables incluidas en la escala TIMI para IAM con elevación del ST}
\label{tab:timi_variables}
\begin{tabular}{@{}ll@{}}
\toprule
\textbf{Variable} & \textbf{Puntos} \\
\midrule
Edad $\geq$ 75 años & 3 \\
Edad 65--74 años & 2 \\
Diabetes, hipertensión o angina & 1 \\
PAS $<$ 100 mmHg & 3 \\
FC $>$ 100 lpm & 2 \\
Killip II--IV & 2 \\
Peso $<$ 67 kg & 1 \\
Elevación ST anterior o BRIHH & 1 \\
Tiempo hasta reperfusión $>$ 4 horas & 1 \\
\bottomrule
\end{tabular}
\end{table}

\subsubsection{Otras Escalas Contemporáneas}

\begin{itemize}
    \item \textbf{ACTION Registry-GWTG}: Incorpora datos contemporáneos incluyendo biomarcadores y tratamientos de reperfusión, mejorando la discriminación en poblaciones modernas \citep{morrow2013action}.
    
    \item \textbf{ProACS}: Desarrollada específicamente para poblaciones europeas, con énfasis en la presentación clínica inicial \citep{pinto2015proacs}.
\end{itemize}

\subsection{Limitaciones de los Modelos Tradicionales}

Los modelos basados en regresión logística presentan limitaciones inherentes:

\begin{enumerate}
    \item \textbf{Asunciones de linealidad}: Relaciones lineales entre variables independientes y el logit del resultado.
    
    \item \textbf{Interacciones predefinidas}: Requieren especificación explícita de términos de interacción.
    
    \item \textbf{Distribuciones asimétricas}: Dificultad para manejar variables con distribuciones no gaussianas.
    
    \item \textbf{Colinealidad}: Sensibilidad a correlaciones entre predictores.
    
    \item \textbf{Falta de actualización}: Modelos estáticos que no se adaptan a cambios en la práctica clínica.
\end{enumerate}

\subsection{Aprendizaje Automático en Predicción Cardiovascular}

\subsubsection{Evolución Histórica}

El uso de técnicas de \gls{ml} en cardiología ha experimentado un crecimiento exponencial en la última década, impulsado por:

\begin{itemize}
    \item Disponibilidad de grandes bases de datos clínicos (EHR, registros).
    \item Avances en capacidad computacional.
    \item Desarrollo de algoritmos más sofisticados.
    \item Técnicas de explicabilidad que facilitan la interpretación.
\end{itemize}

\subsubsection{Estudios Relevantes}

\begin{keypoint}
\textbf{Principales hallazgos en la literatura:}
\begin{itemize}
    \item Los modelos de gradient boosting (XGBoost, LightGBM) consistentemente superan a la regresión logística en predicción de mortalidad por IAM.
    \item El AUROC de modelos de ML típicamente oscila entre 0.85--0.93, comparado con 0.75--0.82 para escalas tradicionales.
    \item Variables como edad, presión arterial, función renal y biomarcadores cardíacos emergen consistentemente como los predictores más importantes.
\end{itemize}
\end{keypoint}

\paragraph{Zhu et al. (2024)}
Estudio multicéntrico con más de 20,000 pacientes con IAM. El modelo XGBoost alcanzó un AUROC de 0.93, significativamente superior a GRACE (0.81) y TIMI (0.78). Los predictores más importantes fueron: edad, presión arterial sistólica, fracción de eyección, NT-proBNP y creatinina \citep{zhu2024ml}.

\paragraph{Oliveira et al. (2023)}
Comparación sistemática de algoritmos (regresión logística penalizada, random forest, XGBoost, redes neuronales) en 5,000 pacientes. Los modelos basados en árboles mostraron mejor rendimiento global (AUROC = 0.89) y mejor calibración \citep{oliveira2023ml}.

\paragraph{Wang et al. (2022)}
Integración de biomarcadores inflamatorios (proteína C reactiva, leucocitos) y función renal (filtrado glomerular estimado) en modelos de ML. La combinación de variables clínicas y analíticas mejoró significativamente la predicción \citep{wang2022ami}.

\subsubsection{Comparación de Algoritmos}

\begin{table}[H]
\centering
\caption{Comparación de algoritmos de ML en predicción de mortalidad por IAM (revisión de literatura)}
\label{tab:comparacion_algoritmos_literatura}
\begin{tabular}{@{}lcccl@{}}
\toprule
\textbf{Algoritmo} & \textbf{AUROC} & \textbf{Sensibilidad} & \textbf{Especificidad} & \textbf{Ventajas} \\
\midrule
Regresión Logística & 0.75--0.82 & 0.65--0.75 & 0.70--0.80 & Interpretable, baseline \\
Random Forest & 0.82--0.88 & 0.70--0.82 & 0.75--0.85 & Robusto, no lineal \\
XGBoost & 0.85--0.93 & 0.75--0.88 & 0.80--0.90 & Alto rendimiento \\
Redes Neuronales & 0.80--0.90 & 0.70--0.85 & 0.75--0.88 & Patrones complejos \\
Ensambles & 0.87--0.94 & 0.78--0.90 & 0.82--0.92 & Combina fortalezas \\
\bottomrule
\end{tabular}
\end{table}

\subsection{Predicción de Arritmias Ventriculares}

Las arritmias ventriculares (taquicardia y fibrilación ventricular) constituyen una causa principal de muerte súbita intrahospitalaria en pacientes con IAM. Los enfoques de ML se han centrado en:

\begin{itemize}
    \item \textbf{Análisis de señales ECG}: Redes neuronales convolucionales (CNN) para detección de patrones arritmogénicos.
    \item \textbf{Variables clínicas tabulares}: Random Forest y XGBoost usando variables como intervalo QTc, bloqueo de rama, número de derivaciones afectadas.
    \item \textbf{Modelos híbridos}: Combinación de datos estructurados y señales electrocardiográficas.
\end{itemize}

\subsection{Desafíos y Vacíos en la Literatura}

\begin{enumerate}
    \item \textbf{Validación externa limitada}: Mayoría de estudios validan en la misma cohorte de desarrollo.
    
    \item \textbf{Desbalance de clases}: Mortalidad intrahospitalaria típicamente $<$10\%, generando sesgos hacia la clase mayoritaria.
    
    \item \textbf{Interpretabilidad}: Modelos de caja negra dificultan la adopción clínica; técnicas SHAP y LIME mitigan este problema.
    
    \item \textbf{Calibración}: Modelos con buena discriminación pero mala calibración pueden inducir decisiones erróneas.
    
    \item \textbf{Heterogeneidad poblacional}: Modelos desarrollados en un contexto pueden no generalizar a otras poblaciones.
    
    \item \textbf{Datos faltantes}: Estrategias de imputación pueden introducir sesgos.
\end{enumerate}

\subsection{Oportunidades de Investigación}

\begin{itemize}
    \item Desarrollo de modelos específicos para poblaciones latinoamericanas y caribeñas.
    \item Integración multimodal de datos clínicos, analíticos, electrocardiográficos e imagenológicos.
    \item Actualización dinámica del riesgo durante la hospitalización.
    \item Evaluación de impacto clínico mediante ensayos de implementación.
    \item Modelos federados que preserven la privacidad de datos.
\end{itemize}

\subsection{Marco Teórico del Presente Estudio}

El presente trabajo se fundamenta en:

\begin{enumerate}
    \item \textbf{Teoría de aprendizaje estadístico}: Principios de generalización, sesgo-varianza, regularización.
    
    \item \textbf{Medicina basada en evidencia}: Evaluación rigurosa de rendimiento predictivo.
    
    \item \textbf{Explicabilidad de IA} (\textit{Explainable AI}): Técnicas SHAP para interpretación de modelos complejos.
    
    \item \textbf{Guías TRIPOD+AI}: Estándares para reporte transparente de modelos de predicción.
\end{enumerate}
