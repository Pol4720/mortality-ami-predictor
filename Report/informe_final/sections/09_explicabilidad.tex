% ============================================================================
% SECCIÓN 09: EXPLICABILIDAD E INTERPRETABILIDAD
% ============================================================================

\section{Explicabilidad del Modelo}
\label{sec:explicabilidad}

La interpretabilidad del modelo es fundamental para su adopción clínica. Esta sección presenta los análisis de explicabilidad realizados mediante técnicas de \textit{Explainable AI} (XAI), principalmente valores \gls{shap}.

\subsection{Importancia Global de Variables}

\subsubsection{Feature Importance Nativa}

\begin{figure}[H]
\centering
\begin{placeholderblock}
\textbf{[INSERTAR GRÁFICO DE FEATURE IMPORTANCE]}

Gráfico de barras horizontales mostrando las top 20 variables más importantes según:
\begin{itemize}
    \item Gain (para XGBoost/LightGBM)
    \item Impurity importance (para Random Forest)
\end{itemize}

Ordenado de mayor a menor importancia.
\end{placeholderblock}
\caption{Importancia de características según el modelo (gain/impurity)}
\label{fig:feature_importance_nativa}
\end{figure}

\subsubsection{SHAP Summary Plot}

\begin{figure}[H]
\centering
\begin{placeholderblock}
\textbf{[INSERTAR SHAP SUMMARY PLOT]}

SHAP summary plot (beeswarm) mostrando:
\begin{itemize}
    \item Top 20 variables en eje Y (ordenadas por importancia SHAP)
    \item Valores SHAP en eje X
    \item Color: valor de la variable (azul = bajo, rojo = alto)
    \item Cada punto = un paciente
\end{itemize}

\textit{Este gráfico muestra tanto la importancia como la dirección del efecto.}
\end{placeholderblock}
\caption{SHAP Summary Plot -- Impacto global de variables}
\label{fig:shap_summary}
\end{figure}

\begin{keypoint}
\textbf{Variables más influyentes según SHAP (top 10):}
\begin{enumerate}
    \item \placeholder{Variable\_1}: \placeholder{Interpretación clínica}
    \item \placeholder{Variable\_2}: \placeholder{Interpretación clínica}
    \item \placeholder{Variable\_3}: \placeholder{Interpretación clínica}
    \item \placeholder{Variable\_4}: \placeholder{Interpretación clínica}
    \item \placeholder{Variable\_5}: \placeholder{Interpretación clínica}
    \item \placeholder{Variable\_6}: \placeholder{Interpretación clínica}
    \item \placeholder{Variable\_7}: \placeholder{Interpretación clínica}
    \item \placeholder{Variable\_8}: \placeholder{Interpretación clínica}
    \item \placeholder{Variable\_9}: \placeholder{Interpretación clínica}
    \item \placeholder{Variable\_10}: \placeholder{Interpretación clínica}
\end{enumerate}
\end{keypoint}

\subsubsection{SHAP Bar Plot}

\begin{figure}[H]
\centering
\begin{placeholderblock}
\textbf{[INSERTAR SHAP BAR PLOT]}

Gráfico de barras mostrando el valor SHAP medio absoluto para las top 15-20 variables.
\end{placeholderblock}
\caption{Importancia media de variables según valores SHAP}
\label{fig:shap_bar}
\end{figure}

\subsection{Efectos Parciales de Variables Individuales}

\subsubsection{SHAP Dependence Plots}

\begin{figure}[H]
\centering
\begin{placeholderblock}
\textbf{[INSERTAR GRID DE SHAP DEPENDENCE PLOTS]}

Panel de 6-9 gráficos (2x3 o 3x3) para las variables más importantes:
\begin{itemize}
    \item Eje X: Valor de la variable
    \item Eje Y: Valor SHAP
    \item Color: Variable de interacción más relevante
\end{itemize}

Variables sugeridas:
\begin{enumerate}
    \item Edad
    \item Presión arterial sistólica
    \item Frecuencia cardíaca
    \item Creatinina / TFG
    \item Troponina
    \item Killip class
    \item FEVI
    \item Glucemia
\end{enumerate}
\end{placeholderblock}
\caption{SHAP Dependence Plots para variables clave}
\label{fig:shap_dependence}
\end{figure}

\subsubsection{Interpretación de Efectos}

\paragraph{Edad}
\begin{placeholderblock}
Describir:
\begin{itemize}
    \item Forma de la relación (lineal, no lineal, umbral)
    \item Punto de inflexión si existe
    \item Consistencia con literatura clínica
\end{itemize}

\textit{Ejemplo: ``La edad muestra un efecto no lineal con aumento del riesgo más pronunciado a partir de los 75 años, consistente con la literatura sobre fragilidad en pacientes con IAM.''}
\end{placeholderblock}

\paragraph{Presión Arterial Sistólica}
\begin{placeholderblock}
Describir la relación PAS-riesgo, incluyendo el efecto de hipotensión y posible efecto en J.
\end{placeholderblock}

\paragraph{Función Renal (Creatinina/TFG)}
\begin{placeholderblock}
Describir el impacto de la función renal en el riesgo predicho.
\end{placeholderblock}

\paragraph{Clasificación Killip}
\begin{placeholderblock}
Describir el gradiente de riesgo según clasificación Killip.
\end{placeholderblock}

\subsection{Interacciones entre Variables}

\subsubsection{SHAP Interaction Values}

\begin{figure}[H]
\centering
\begin{placeholderblock}
\textbf{[INSERTAR HEATMAP DE INTERACCIONES SHAP]}

Mapa de calor mostrando la magnitud de las interacciones entre las top 10-15 variables.
\end{placeholderblock}
\caption{Matriz de interacciones SHAP}
\label{fig:shap_interactions}
\end{figure}

\subsubsection{Interacciones Clínicamente Relevantes}

\begin{placeholderblock}
Describir las interacciones más importantes:
\begin{itemize}
    \item \textbf{Edad $\times$ Killip}: \placeholder{Descripción del efecto sinérgico}
    \item \textbf{Diabetes $\times$ TFG}: \placeholder{Descripción}
    \item \textbf{PAS $\times$ FC}: \placeholder{Descripción (ej. shock index)}
\end{itemize}
\end{placeholderblock}

\subsection{Explicaciones Locales (Individuales)}

\subsubsection{SHAP Force Plots}

\begin{figure}[H]
\centering
\begin{placeholderblock}
\textbf{[INSERTAR SHAP FORCE PLOT -- CASO ALTO RIESGO]}

Force plot para un paciente con alta probabilidad de mortalidad mostrando:
\begin{itemize}
    \item Valor base (probabilidad media)
    \item Factores que aumentan el riesgo (rojo)
    \item Factores que disminuyen el riesgo (azul)
    \item Probabilidad final predicha
\end{itemize}
\end{placeholderblock}
\caption{SHAP Force Plot -- Paciente de alto riesgo}
\label{fig:shap_force_alto}
\end{figure}

\begin{figure}[H]
\centering
\begin{placeholderblock}
\textbf{[INSERTAR SHAP FORCE PLOT -- CASO BAJO RIESGO]}

Force plot para un paciente con baja probabilidad de mortalidad.
\end{placeholderblock}
\caption{SHAP Force Plot -- Paciente de bajo riesgo}
\label{fig:shap_force_bajo}
\end{figure}

\subsubsection{SHAP Waterfall Plots}

\begin{figure}[H]
\centering
\begin{placeholderblock}
\textbf{[INSERTAR SHAP WATERFALL PLOTS]}

Panel con 2-3 waterfall plots para casos representativos:
\begin{itemize}
    \item Caso de alto riesgo correctamente clasificado (VP)
    \item Caso de bajo riesgo correctamente clasificado (VN)
    \item Caso de error (FN o FP) -- análisis de por qué falló
\end{itemize}
\end{placeholderblock}
\caption{SHAP Waterfall Plots -- Casos representativos}
\label{fig:shap_waterfall}
\end{figure}

\subsubsection{Ejemplos de Explicaciones Individuales}

\begin{table}[H]
\centering
\caption{Explicación de predicciones para casos representativos}
\label{tab:explicaciones_individuales}
\begin{tabular}{@{}p{2cm}p{3cm}p{3cm}p{5cm}@{}}
\toprule
\textbf{Caso} & \textbf{Prob. predicha} & \textbf{Outcome real} & \textbf{Principales factores} \\
\midrule
Paciente A & \placeholder{0.XX (Alto)} & Fallecido & \placeholder{Edad 82, Killip IV, Creatinina 3.2} \\
Paciente B & \placeholder{0.XX (Bajo)} & Superviviente & \placeholder{Edad 55, Killip I, FEVI 50\%} \\
Paciente C & \placeholder{0.XX (Intermedio)} & Superviviente & \placeholder{Edad 70, Diabetes, FEVI 35\%} \\
\bottomrule
\end{tabular}
\end{table}

\subsection{Análisis de Grupos de Riesgo}

\subsubsection{Estratificación por Probabilidad Predicha}

\begin{table}[H]
\centering
\caption{Características por tercil de riesgo predicho}
\label{tab:terciles_riesgo}
\begin{tabular}{@{}lccc@{}}
\toprule
\textbf{Variable} & \textbf{Bajo riesgo} & \textbf{Riesgo medio} & \textbf{Alto riesgo} \\
 & ($<$\placeholder{XX}\%) & (\placeholder{XX}--\placeholder{XX}\%) & ($>$\placeholder{XX}\%) \\
\midrule
N pacientes & \placeholder{n} & \placeholder{n} & \placeholder{n} \\
Mortalidad observada & \placeholder{X.X\%} & \placeholder{X.X\%} & \placeholder{XX.X\%} \\
Edad media & \placeholder{XX.X} & \placeholder{XX.X} & \placeholder{XX.X} \\
Killip III-IV & \placeholder{X.X\%} & \placeholder{XX.X\%} & \placeholder{XX.X\%} \\
SHAP medio (top 5 vars) & \placeholder{-X.XX} & \placeholder{X.XX} & \placeholder{+X.XX} \\
\bottomrule
\end{tabular}
\end{table}

\subsubsection{Perfiles SHAP por Grupo}

\begin{figure}[H]
\centering
\begin{placeholderblock}
\textbf{[INSERTAR SHAP PROFILES POR GRUPO]}

Panel mostrando distribución de valores SHAP para las top 10 variables, separado por grupo de riesgo (bajo/medio/alto).
\end{placeholderblock}
\caption{Distribución de valores SHAP por grupo de riesgo}
\label{fig:shap_por_grupo}
\end{figure}

\subsection{Validación Clínica de la Explicabilidad}

\begin{placeholderblock}
\textbf{[COMPLETAR SI SE REALIZÓ VALIDACIÓN CON EXPERTOS]}

Describir:
\begin{itemize}
    \item Número de expertos clínicos consultados
    \item Casos revisados
    \item Concordancia con juicio clínico
    \item Feedback sobre utilidad de las explicaciones
    \item Sugerencias de mejora
\end{itemize}
\end{placeholderblock}

\subsection{Consistencia con Conocimiento Clínico}

\begin{table}[H]
\centering
\caption{Consistencia de hallazgos SHAP con literatura clínica}
\label{tab:consistencia_clinica}
\begin{tabular}{@{}llp{5cm}@{}}
\toprule
\textbf{Variable} & \textbf{Efecto SHAP} & \textbf{Evidencia clínica} \\
\midrule
Edad avanzada & $\uparrow$ riesgo & Consistente: mayor fragilidad, comorbilidades \\
Killip alto & $\uparrow$ riesgo & Consistente: insuficiencia cardíaca \\
TFG baja & $\uparrow$ riesgo & Consistente: síndrome cardiorrenal \\
PAS baja & $\uparrow$ riesgo & Consistente: hipoperfusión, shock \\
FEVI baja & $\uparrow$ riesgo & Consistente: disfunción ventricular \\
\placeholder{Variable} & \placeholder{Efecto} & \placeholder{Interpretación} \\
\bottomrule
\end{tabular}
\end{table}

\subsection{Implicaciones para la Práctica Clínica}

\begin{keypoint}
\textbf{Puntos clave para la interpretación clínica:}

\begin{enumerate}
    \item \textbf{Factores modificables}: Las variables \placeholder{X, Y, Z} son potencialmente modificables y podrían ser objetivos terapéuticos.
    
    \item \textbf{Identificación de pacientes vulnerables}: El modelo identifica pacientes con combinaciones de factores de alto riesgo que podrían beneficiarse de monitorización intensiva.
    
    \item \textbf{Explicaciones individualizadas}: Las explicaciones SHAP permiten comunicar al equipo clínico los principales factores de riesgo de cada paciente.
    
    \item \textbf{Limitaciones}: El modelo captura asociaciones, no causalidad. Los factores identificados no necesariamente son objetivos de intervención.
\end{enumerate}
\end{keypoint}

\subsection{Integración en la Herramienta Dashboard}

\begin{placeholderblock}
\textbf{[DESCRIBIR IMPLEMENTACIÓN EN DASHBOARD]}

Describir cómo se integran las explicaciones en la interfaz de usuario:
\begin{itemize}
    \item Visualización de SHAP force plot para cada predicción
    \item Identificación de factores de riesgo principales
    \item Comparación con población de referencia
    \item Generación de reportes explicativos
\end{itemize}

\textit{Referencias al Manual de Usuario y Documentación Técnica.}
\end{placeholderblock}
