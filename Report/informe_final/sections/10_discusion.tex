% ============================================================================
% SECCIÓN 10: DISCUSIÓN
% ============================================================================

\section{Discusión}
\label{sec:discusion}

\subsection{Resumen de Hallazgos Principales}

\begin{placeholderblock}
Este estudio desarrolló y validó un modelo de aprendizaje automático para la predicción de mortalidad intrahospitalaria en pacientes con \gls{iam}. Los principales hallazgos fueron:

\begin{enumerate}
    \item El modelo \placeholder{XGBoost/RF/Ensemble} alcanzó un AUROC de \placeholder{0.XXX} (IC 95\%: \placeholder{0.XXX--0.XXX}), superior al score GRACE (\placeholder{0.XXX}, p \placeholder{$<$0.001}).
    
    \item Las variables más predictivas fueron \placeholder{edad, clasificación Killip, función renal, troponinas, FEVI}, consistentes con el conocimiento fisiopatológico.
    
    \item El modelo mostró buena calibración (Brier Score = \placeholder{0.XXX}) y utilidad clínica demostrada mediante Decision Curve Analysis.
    
    \item El análisis SHAP reveló relaciones no lineales e interacciones entre variables que los modelos tradicionales no capturan.
\end{enumerate}
\end{placeholderblock}

\subsection{Comparación con la Literatura}

\subsubsection{Rendimiento Comparativo}

\begin{table}[H]
\centering
\caption{Comparación con estudios previos de ML en predicción de mortalidad por IAM}
\label{tab:comparacion_literatura}
\begin{tabular}{@{}llllc@{}}
\toprule
\textbf{Estudio} & \textbf{N} & \textbf{Modelo} & \textbf{AUROC} & \textbf{Validación} \\
\midrule
Zhu et al. 2024 & 20,000 & XGBoost & 0.93 & Externa \\
Oliveira et al. 2023 & 5,000 & Ensemble & 0.89 & Interna \\
Wang et al. 2022 & 8,500 & Random Forest & 0.87 & Temporal \\
\midrule
\rowcolor{accentgold!20}
\textbf{Presente estudio} & \placeholder{N} & \placeholder{Modelo} & \placeholder{0.XX} & \placeholder{Tipo} \\
\bottomrule
\end{tabular}
\end{table}

\begin{placeholderblock}
\textbf{[DISCUTIR COMPARACIÓN]}

Analizar:
\begin{itemize}
    \item Similitudes y diferencias con estudios previos
    \item Posibles explicaciones de diferencias en rendimiento
    \item Características específicas de la población estudiada
    \item Ventajas/limitaciones del enfoque utilizado
\end{itemize}
\end{placeholderblock}

\subsubsection{Variables Predictoras}

\begin{placeholderblock}
\textbf{[DISCUTIR VARIABLES IDENTIFICADAS]}

Comparar las variables más importantes con literatura previa:
\begin{itemize}
    \item Variables consistentes con escalas clásicas (GRACE, TIMI)
    \item Variables adicionales identificadas por ML
    \item Variables esperadas que no resultaron importantes
    \item Posibles explicaciones fisiopatológicas
\end{itemize}
\end{placeholderblock}

\subsection{Fortalezas del Estudio}

\begin{enumerate}
    \item \textbf{Metodología rigurosa}: 
    \begin{placeholderblock}
    Describir aspectos metodológicos sólidos:
    \begin{itemize}
        \item Seguimiento de guías TRIPOD+AI
        \item Separación estricta de conjuntos train/validation/test
        \item Validación cruzada estratificada
        \item Múltiples métricas de evaluación
    \end{itemize}
    \end{placeholderblock}
    
    \item \textbf{Análisis de explicabilidad}: 
    \begin{placeholderblock}
    El uso de técnicas SHAP permite interpretar las predicciones del modelo, facilitando la confianza clínica y la identificación de factores de riesgo modificables.
    \end{placeholderblock}
    
    \item \textbf{Comparación con benchmarks establecidos}: 
    \begin{placeholderblock}
    La comparación directa con el score GRACE proporciona contexto clínico relevante para la adopción del modelo.
    \end{placeholderblock}
    
    \item \textbf{Evaluación de utilidad clínica}: 
    \begin{placeholderblock}
    El análisis de curva de decisión va más allá de métricas estadísticas para evaluar el beneficio práctico del modelo.
    \end{placeholderblock}
    
    \item \textbf{Reproducibilidad}: 
    \begin{placeholderblock}
    El código fuente, pipelines y modelos están documentados y disponibles para replicación.
    \end{placeholderblock}
\end{enumerate}

\subsection{Limitaciones del Estudio}

\subsubsection{Limitaciones Metodológicas}

\begin{enumerate}
    \item \textbf{Diseño retrospectivo}: 
    \begin{placeholderblock}
    Los datos fueron recolectados retrospectivamente, lo que puede introducir sesgos de selección y de información.
    \end{placeholderblock}
    
    \item \textbf{Centro único}: 
    \begin{placeholderblock}
    [Si aplica] El modelo fue desarrollado con datos de un solo centro/región, lo que puede limitar la generalización a otras poblaciones.
    \end{placeholderblock}
    
    \item \textbf{Validación interna}: 
    \begin{placeholderblock}
    [Si aplica] La validación se realizó mediante división interna de datos. Se requiere validación externa en cohortes independientes.
    \end{placeholderblock}
    
    \item \textbf{Datos faltantes}: 
    \begin{placeholderblock}
    Algunas variables presentaron alto porcentaje de datos faltantes, requiriendo imputación que puede introducir sesgos.
    \end{placeholderblock}
\end{enumerate}

\subsubsection{Limitaciones de los Datos}

\begin{enumerate}
    \item \textbf{Periodo temporal}: 
    \begin{placeholderblock}
    Los datos corresponden al período [fechas], y los patrones de práctica clínica pueden haber cambiado desde entonces.
    \end{placeholderblock}
    
    \item \textbf{Variables no disponibles}: 
    \begin{placeholderblock}
    Algunas variables potencialmente relevantes (ej. \placeholder{biomarcadores específicos, datos genéticos, ECG continuo}) no estaban disponibles en el dataset.
    \end{placeholderblock}
    
    \item \textbf{Definición del outcome}: 
    \begin{placeholderblock}
    La mortalidad intrahospitalaria no captura eventos post-alta que pueden ser clínicamente relevantes.
    \end{placeholderblock}
\end{enumerate}

\subsubsection{Limitaciones del Modelo}

\begin{enumerate}
    \item \textbf{Interpretabilidad vs. rendimiento}: 
    \begin{placeholderblock}
    Aunque las técnicas SHAP mejoran la interpretabilidad, los modelos de ensemble siguen siendo más complejos que la regresión logística tradicional.
    \end{placeholderblock}
    
    \item \textbf{Causalidad}: 
    \begin{placeholderblock}
    El modelo identifica asociaciones predictivas, no relaciones causales. La intervención sobre variables identificadas no garantiza mejora en outcomes.
    \end{placeholderblock}
    
    \item \textbf{Drift temporal}: 
    \begin{placeholderblock}
    El rendimiento del modelo puede degradarse con el tiempo si cambian las características de la población o las prácticas clínicas.
    \end{placeholderblock}
\end{enumerate}

\subsection{Implicaciones Clínicas}

\subsubsection{Potencial de Implementación}

\begin{placeholderblock}
\textbf{[DISCUTIR APLICABILIDAD CLÍNICA]}

\begin{itemize}
    \item Escenarios de uso: triaje, monitorización, decisiones de tratamiento
    \item Integración con flujos de trabajo existentes
    \item Recursos necesarios para implementación
    \item Aceptabilidad por parte del personal clínico
\end{itemize}
\end{placeholderblock}

\subsubsection{Beneficios Potenciales}

\begin{enumerate}
    \item \textbf{Identificación temprana de alto riesgo}: 
    \begin{placeholderblock}
    Permite identificar pacientes que requieren monitorización intensiva o intervenciones agresivas.
    \end{placeholderblock}
    
    \item \textbf{Optimización de recursos}: 
    \begin{placeholderblock}
    La estratificación de riesgo puede guiar la asignación de camas en UCI/UCO.
    \end{placeholderblock}
    
    \item \textbf{Comunicación con pacientes/familias}: 
    \begin{placeholderblock}
    Proporciona información pronóstica objetiva para la toma de decisiones compartida.
    \end{placeholderblock}
    
    \item \textbf{Apoyo a la decisión}: 
    \begin{placeholderblock}
    Complementa (no reemplaza) el juicio clínico con información cuantitativa.
    \end{placeholderblock}
\end{enumerate}

\subsubsection{Consideraciones de Implementación}

\begin{placeholderblock}
\textbf{[DISCUTIR BARRERAS Y FACILITADORES]}

\begin{itemize}
    \item Integración con sistemas de historia clínica electrónica
    \item Entrenamiento del personal
    \item Validación local antes de implementación
    \item Monitorización continua del rendimiento
    \item Aspectos regulatorios (certificación de software médico)
\end{itemize}
\end{placeholderblock}

\subsection{Direcciones Futuras}

\subsubsection{Validación y Extensión}

\begin{enumerate}
    \item \textbf{Validación externa}: 
    \begin{placeholderblock}
    Evaluar el modelo en cohortes independientes de otras instituciones/países.
    \end{placeholderblock}
    
    \item \textbf{Validación temporal prospectiva}: 
    \begin{placeholderblock}
    Implementar el modelo en tiempo real y evaluar rendimiento prospectivo.
    \end{placeholderblock}
    
    \item \textbf{Estudio de impacto}: 
    \begin{placeholderblock}
    Ensayo clínico evaluando si el uso del modelo mejora los outcomes de los pacientes.
    \end{placeholderblock}
\end{enumerate}

\subsubsection{Mejoras del Modelo}

\begin{enumerate}
    \item \textbf{Incorporación de nuevas variables}: 
    \begin{placeholderblock}
    Explorar la adición de biomarcadores emergentes, datos de imagen, o señales de ECG continuo.
    \end{placeholderblock}
    
    \item \textbf{Predicción dinámica}: 
    \begin{placeholderblock}
    Desarrollar modelos que actualicen el riesgo a medida que evoluciona el paciente durante la hospitalización.
    \end{placeholderblock}
    
    \item \textbf{Outcomes adicionales}: 
    \begin{placeholderblock}
    Extender a predicción de arritmias ventriculares, shock cardiogénico, reinfartos.
    \end{placeholderblock}
    
    \item \textbf{Aprendizaje federado}: 
    \begin{placeholderblock}
    Desarrollar modelos multicéntricos preservando la privacidad de datos.
    \end{placeholderblock}
\end{enumerate}

\subsubsection{Aspectos de Implementación}

\begin{enumerate}
    \item \textbf{Integración tecnológica}: 
    \begin{placeholderblock}
    Desarrollar APIs para integración con sistemas hospitalarios.
    \end{placeholderblock}
    
    \item \textbf{Interfaz de usuario}: 
    \begin{placeholderblock}
    Refinar el dashboard basándose en feedback de usuarios clínicos.
    \end{placeholderblock}
    
    \item \textbf{Mantenimiento del modelo}: 
    \begin{placeholderblock}
    Establecer procesos para re-entrenamiento y monitorización de drift.
    \end{placeholderblock}
\end{enumerate}

\subsection{Consideraciones Éticas}

\begin{placeholderblock}
\textbf{[DISCUTIR ASPECTOS ÉTICOS]}

\begin{itemize}
    \item \textbf{Equidad}: El modelo debe ser evaluado por subgrupos para evitar sesgos discriminatorios.
    \item \textbf{Transparencia}: Las explicaciones deben ser comunicables a pacientes y familias.
    \item \textbf{Autonomía}: El modelo es una herramienta de apoyo, no un sustituto del juicio clínico.
    \item \textbf{Privacidad}: Los datos deben ser manejados según normativas de protección de datos.
    \item \textbf{Responsabilidad}: Definir responsabilidades cuando las predicciones del modelo influyan en decisiones clínicas.
\end{itemize}
\end{placeholderblock}
