% ============================================================================
% SECCIÓN 06: PREPROCESAMIENTO DE DATOS
% ============================================================================

\section{Preprocesamiento de Datos}
\label{sec:preprocesamiento}

Esta sección detalla los procedimientos aplicados para preparar los datos crudos para el modelado predictivo.

\subsection{Flujo de Preprocesamiento}

\begin{figure}[H]
\centering
\begin{placeholderblock}
\textbf{[INSERTAR DIAGRAMA DE FLUJO DE PREPROCESAMIENTO]}

Diagrama mostrando:
\begin{enumerate}
    \item Datos crudos (N inicial)
    \item Exclusiones y filtrado
    \item Imputación de faltantes
    \item Tratamiento de outliers
    \item Codificación de variables
    \item Normalización/Transformación
    \item Datos procesados (N final)
\end{enumerate}
\end{placeholderblock}
\caption{Pipeline de preprocesamiento de datos}
\label{fig:pipeline_preprocesamiento}
\end{figure}

\subsection{Exclusión de Observaciones y Variables}

\subsubsection{Exclusiones Realizadas}

\begin{table}[H]
\centering
\caption{Resumen de exclusiones de observaciones}
\label{tab:exclusiones_observaciones}
\begin{tabular}{@{}lcc@{}}
\toprule
\textbf{Motivo de exclusión} & \textbf{N excluidos} & \textbf{N restante} \\
\midrule
Registros iniciales & -- & \placeholder{N total} \\
Duplicados & \placeholder{n} & \placeholder{N} \\
Diagnóstico no confirmado de IAM & \placeholder{n} & \placeholder{N} \\
Datos de outcome faltantes & \placeholder{n} & \placeholder{N} \\
$>$\placeholder{X}\% variables faltantes & \placeholder{n} & \placeholder{N} \\
\midrule
\textbf{Muestra final de análisis} & -- & \placeholder{N final} \\
\bottomrule
\end{tabular}
\end{table}

\subsubsection{Variables Excluidas}

\begin{table}[H]
\centering
\caption{Variables excluidas del análisis}
\label{tab:variables_excluidas}
\begin{tabular}{@{}llp{6cm}@{}}
\toprule
\textbf{Variable} & \textbf{\% Faltantes} & \textbf{Motivo de exclusión} \\
\midrule
\placeholder{Variable\_1} & \placeholder{XX\%} & Exceso de datos faltantes \\
\placeholder{Variable\_2} & \placeholder{XX\%} & Exceso de datos faltantes \\
\placeholder{Variable\_3} & -- & Información post-outcome (data leakage) \\
\placeholder{Variable\_4} & -- & Identificador único (no predictivo) \\
\placeholder{Variable\_5} & -- & Redundante con Variable\_X \\
\bottomrule
\end{tabular}
\end{table}

\subsection{Imputación de Datos Faltantes}

\subsubsection{Análisis del Patrón de Missingness}

\begin{figure}[H]
\centering
\begin{placeholderblock}
\textbf{[INSERTAR ANÁLISIS DE PATRÓN DE MISSINGNESS]}

Mostrar:
\begin{itemize}
    \item Test de Little (MCAR)
    \item Correlación de indicadores de missingness
    \item Conclusión sobre mecanismo predominante (MCAR/MAR/MNAR)
\end{itemize}
\end{placeholderblock}
\caption{Análisis del mecanismo de datos faltantes}
\label{fig:mecanismo_missingness}
\end{figure}

\subsubsection{Estrategias de Imputación Aplicadas}

\begin{table}[H]
\centering
\caption{Métodos de imputación por tipo de variable}
\label{tab:metodos_imputacion}
\begin{tabular}{@{}llll@{}}
\toprule
\textbf{Tipo} & \textbf{Método} & \textbf{Parámetros} & \textbf{Variables} \\
\midrule
Numéricas continuas & \placeholder{KNN Imputer} & \placeholder{k=5} & \placeholder{Lista} \\
Numéricas con sesgo & \placeholder{Mediana} & -- & \placeholder{Lista} \\
Categóricas & \placeholder{Moda} & -- & \placeholder{Lista} \\
Categóricas especiales & \placeholder{``Desconocido''} & -- & \placeholder{Lista} \\
\bottomrule
\end{tabular}
\end{table}

\begin{placeholderblock}
\textbf{[OPCIONAL: VALIDACIÓN DE IMPUTACIÓN]}

Incluir si se realizó:
\begin{itemize}
    \item Comparación de distribuciones pre/post imputación
    \item Análisis de sensibilidad con diferentes métodos
    \item Métricas de calidad de imputación
\end{itemize}
\end{placeholderblock}

\subsection{Tratamiento de Valores Atípicos}

\subsubsection{Detección de Outliers}

\begin{table}[H]
\centering
\caption{Outliers detectados por variable}
\label{tab:outliers_detectados}
\begin{tabular}{@{}lcccc@{}}
\toprule
\textbf{Variable} & \textbf{Método} & \textbf{N outliers} & \textbf{Límite inf.} & \textbf{Límite sup.} \\
\midrule
\placeholder{Edad} & \placeholder{IQR} & \placeholder{n} & \placeholder{X} & \placeholder{X} \\
\placeholder{PAS} & \placeholder{IQR} & \placeholder{n} & \placeholder{X} & \placeholder{X} \\
\placeholder{Creatinina} & \placeholder{IQR} & \placeholder{n} & \placeholder{X} & \placeholder{X} \\
\placeholder{Troponina} & \placeholder{Z-score} & \placeholder{n} & \placeholder{X} & \placeholder{X} \\
\bottomrule
\end{tabular}
\end{table}

\subsubsection{Tratamiento Aplicado}

\begin{itemize}
    \item \textbf{Estrategia general}: \placeholder{Winsorización/Capping/Exclusión/Transformación}
    \item \textbf{Percentiles de corte}: \placeholder{1\%--99\% / 5\%--95\%}
    \item \textbf{Justificación clínica}: \placeholder{Describir si se consultó con expertos clínicos}
\end{itemize}

\begin{figure}[H]
\centering
\begin{placeholderblock}
\textbf{[INSERTAR COMPARACIÓN ANTES/DESPUÉS]}

Panel de boxplots mostrando distribución de variables con outliers antes y después del tratamiento.
\end{placeholderblock}
\caption{Distribución de variables antes y después del tratamiento de outliers}
\label{fig:antes_despues_outliers}
\end{figure}

\subsection{Codificación de Variables Categóricas}

\subsubsection{Variables Binarias}

\begin{table}[H]
\centering
\caption{Codificación de variables binarias}
\label{tab:codificacion_binarias}
\begin{tabular}{@{}llll@{}}
\toprule
\textbf{Variable} & \textbf{Categoría 0} & \textbf{Categoría 1} & \textbf{N en cat. 1} \\
\midrule
SEXO & Femenino & Masculino & \placeholder{n (\%)} \\
APP\_HTA & No & Sí & \placeholder{n (\%)} \\
APP\_DM & No & Sí & \placeholder{n (\%)} \\
\placeholder{...} & & & \\
\bottomrule
\end{tabular}
\end{table}

\subsubsection{Variables Nominales (One-Hot Encoding)}

\begin{table}[H]
\centering
\caption{Variables con One-Hot Encoding}
\label{tab:one_hot}
\begin{tabular}{@{}lll@{}}
\toprule
\textbf{Variable original} & \textbf{Categorías} & \textbf{Variables dummy creadas} \\
\midrule
\placeholder{ECG\_LOCALIZACION} & \placeholder{Anterior, Inferior, Lateral} & \placeholder{LOC\_anterior, LOC\_inferior, LOC\_lateral} \\
\placeholder{ECG\_RITMO} & \placeholder{Sinusal, FA, Flutter} & \placeholder{RITMO\_sinusal, RITMO\_fa, RITMO\_flutter} \\
\bottomrule
\end{tabular}
\end{table}

\subsubsection{Variables Ordinales}

\begin{table}[H]
\centering
\caption{Codificación de variables ordinales}
\label{tab:codificacion_ordinales}
\begin{tabular}{@{}llll@{}}
\toprule
\textbf{Variable} & \textbf{Categorías (orden)} & \textbf{Valores asignados} \\
\midrule
KILLIP & I $<$ II $<$ III $<$ IV & 1, 2, 3, 4 \\
\placeholder{TABAQUISMO} & \placeholder{Nunca $<$ Ex $<$ Activo} & \placeholder{0, 1, 2} \\
\bottomrule
\end{tabular}
\end{table}

\subsection{Transformaciones de Variables Numéricas}

\subsubsection{Variables con Transformación Logarítmica}

\begin{table}[H]
\centering
\caption{Variables transformadas logarítmicamente}
\label{tab:transformacion_log}
\begin{tabular}{@{}lccc@{}}
\toprule
\textbf{Variable} & \textbf{Skewness original} & \textbf{Skewness post-log} & \textbf{Justificación} \\
\midrule
\placeholder{TROPONINA} & \placeholder{X.XX} & \placeholder{X.XX} & Alta asimetría positiva \\
\placeholder{NT\_PROBNP} & \placeholder{X.XX} & \placeholder{X.XX} & Distribución log-normal \\
\placeholder{CREATININA} & \placeholder{X.XX} & \placeholder{X.XX} & Cola derecha extendida \\
\bottomrule
\end{tabular}
\end{table}

\subsubsection{Normalización/Estandarización}

\begin{table}[H]
\centering
\caption{Métodos de escalado aplicados}
\label{tab:escalado}
\begin{tabular}{@{}lll@{}}
\toprule
\textbf{Método} & \textbf{Fórmula} & \textbf{Variables aplicadas} \\
\midrule
StandardScaler & $z = \frac{x - \mu}{\sigma}$ & \placeholder{Variables para modelos lineales/NN} \\
MinMaxScaler & $x' = \frac{x - x_{min}}{x_{max} - x_{min}}$ & \placeholder{Variables para NN (opcional)} \\
RobustScaler & $z = \frac{x - Q_2}{Q_3 - Q_1}$ & \placeholder{Variables con outliers residuales} \\
\bottomrule
\end{tabular}
\end{table}

\begin{keypoint}
\textbf{Nota importante:} Los parámetros de escalado (media, desviación estándar, etc.) se calcularon \textbf{únicamente} en el conjunto de entrenamiento y se aplicaron a los conjuntos de validación y test para evitar fuga de información.
\end{keypoint}

\subsection{Ingeniería de Características}

\subsubsection{Variables Derivadas Creadas}

\begin{table}[H]
\centering
\caption{Nuevas variables creadas}
\label{tab:nuevas_variables}
\begin{tabular}{@{}llp{5cm}@{}}
\toprule
\textbf{Variable} & \textbf{Fórmula/Definición} & \textbf{Justificación} \\
\midrule
\placeholder{IMC} & $\frac{\text{peso (kg)}}{\text{altura (m)}^2}$ & Indicador de obesidad \\
\placeholder{PAM} & $\frac{\text{PAS} + 2 \times \text{PAD}}{3}$ & Presión arterial media \\
\placeholder{PP} & $\text{PAS} - \text{PAD}$ & Presión de pulso (rigidez arterial) \\
\placeholder{SHOCK\_INDEX} & $\frac{\text{FC}}{\text{PAS}}$ & Índice de shock \\
\placeholder{ANEMIA} & Hb $<$ 12 (M) / $<$ 13 (H) & Indicador binario de anemia \\
\placeholder{ERC\_STAGE} & Según TFG: G1--G5 & Estadio de enfermedad renal \\
\bottomrule
\end{tabular}
\end{table}

\subsubsection{Interacciones y Términos No Lineales}

\begin{placeholderblock}
\textbf{[COMPLETAR SI SE CREARON INTERACCIONES]}

Describir:
\begin{itemize}
    \item Interacciones basadas en conocimiento clínico
    \item Términos polinómicos (si aplica)
    \item Justificación de cada interacción
\end{itemize}
\end{placeholderblock}

\subsection{Selección de Características}

\subsubsection{Filtrado Univariado}

\begin{table}[H]
\centering
\caption{Resultados de pruebas de asociación univariada}
\label{tab:filtrado_univariado}
\begin{tabular}{@{}lllc@{}}
\toprule
\textbf{Variable} & \textbf{Test} & \textbf{Estadístico} & \textbf{p-valor} \\
\midrule
\placeholder{EDAD} & t-test & \placeholder{t = X.XX} & \placeholder{$<$0.001} \\
\placeholder{KILLIP} & $\chi^2$ & \placeholder{$\chi^2$ = X.XX} & \placeholder{$<$0.001} \\
\placeholder{CREATININA} & Mann-Whitney & \placeholder{U = XXXX} & \placeholder{$<$0.001} \\
\multicolumn{4}{c}{\textit{... (mostrar variables más significativas)}} \\
\bottomrule
\end{tabular}
\end{table}

\subsubsection{Importancia de Características (Feature Importance)}

\begin{figure}[H]
\centering
\begin{placeholderblock}
\textbf{[INSERTAR GRÁFICO DE FEATURE IMPORTANCE]}

Gráfico de barras horizontales mostrando importancia de las top 20--30 variables según:
\begin{itemize}
    \item Random Forest (importancia por impureza o permutación)
    \item XGBoost (gain, cover, o frequency)
\end{itemize}
\end{placeholderblock}
\caption{Importancia de características según modelos de ensemble}
\label{fig:feature_importance_preproceso}
\end{figure}

\subsubsection{Conjunto Final de Variables}

\begin{table}[H]
\centering
\caption{Resumen de variables incluidas en el modelo final}
\label{tab:variables_finales}
\begin{tabular}{@{}lc@{}}
\toprule
\textbf{Categoría} & \textbf{N variables} \\
\midrule
Demográficas & \placeholder{X} \\
Antecedentes & \placeholder{X} \\
Presentación clínica & \placeholder{X} \\
Electrocardiograma & \placeholder{X} \\
Biomarcadores & \placeholder{X} \\
Tratamientos & \placeholder{X} \\
Variables derivadas & \placeholder{X} \\
\midrule
\textbf{Total variables finales} & \placeholder{XX} \\
\bottomrule
\end{tabular}
\end{table}

\subsection{División de Datos}

\begin{figure}[H]
\centering
\begin{placeholderblock}
\textbf{[INSERTAR DIAGRAMA DE PARTICIÓN]}

Diagrama mostrando la división de datos:
\begin{itemize}
    \item N total $\rightarrow$ Train (70\%) + Validation (15\%) + Test (15\%)
    \item Indicar estratificación por variable objetivo
    \item Mostrar distribución de clases en cada partición
\end{itemize}
\end{placeholderblock}
\caption{Esquema de partición de datos}
\label{fig:particion_datos}
\end{figure}

\begin{table}[H]
\centering
\caption{Distribución de datos en cada partición}
\label{tab:distribucion_particiones}
\begin{tabular}{@{}lcccc@{}}
\toprule
\textbf{Conjunto} & \textbf{N total} & \textbf{Eventos (n)} & \textbf{No eventos (n)} & \textbf{Tasa eventos} \\
\midrule
Entrenamiento & \placeholder{N} & \placeholder{n} & \placeholder{n} & \placeholder{X.X\%} \\
Validación & \placeholder{N} & \placeholder{n} & \placeholder{n} & \placeholder{X.X\%} \\
Test & \placeholder{N} & \placeholder{n} & \placeholder{n} & \placeholder{X.X\%} \\
\midrule
Total & \placeholder{N} & \placeholder{n} & \placeholder{n} & \placeholder{X.X\%} \\
\bottomrule
\end{tabular}
\end{table}

\subsection{Resumen del Preprocesamiento}

\begin{keypoint}
\textbf{Resumen de transformaciones aplicadas:}

\begin{enumerate}
    \item \textbf{Observaciones}: De \placeholder{N inicial} a \placeholder{N final} tras exclusiones.
    \item \textbf{Variables}: De \placeholder{185 originales} a \placeholder{XX finales}.
    \item \textbf{Imputación}: \placeholder{X} variables imputadas mediante \placeholder{método(s)}.
    \item \textbf{Outliers}: \placeholder{X} variables con outliers tratados.
    \item \textbf{Transformaciones}: \placeholder{X} variables transformadas logarítmicamente.
    \item \textbf{Codificación}: \placeholder{X} variables dummy creadas.
    \item \textbf{Variables nuevas}: \placeholder{X} variables derivadas añadidas.
\end{enumerate}
\end{keypoint}
