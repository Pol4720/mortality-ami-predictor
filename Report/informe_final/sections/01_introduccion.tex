% ============================================================================
% SECCIÓN 01: INTRODUCCIÓN
% ============================================================================

\section{Introducción}
\label{sec:introduccion}

\subsection{Contexto y Motivación}

El infarto agudo de miocardio (\gls{iam}) constituye una emergencia cardiovascular que representa una de las principales causas de morbilidad y mortalidad a nivel mundial \citep{fox2006grace}. Según la Organización Mundial de la Salud, las enfermedades cardiovasculares causan aproximadamente 17,9 millones de muertes anuales, de las cuales una proporción significativa corresponde a eventos coronarios agudos.

La identificación temprana de pacientes con alto riesgo de mortalidad intrahospitalaria resulta fundamental para:

\begin{itemize}
    \item \textbf{Optimizar la estratificación de riesgo}: Clasificar pacientes según su probabilidad de eventos adversos.
    \item \textbf{Guiar decisiones terapéuticas}: Seleccionar estrategias de reperfusión y tratamiento farmacológico.
    \item \textbf{Asignar recursos}: Determinar niveles de monitorización y cuidados intensivos.
    \item \textbf{Informar al paciente y familia}: Comunicar pronóstico de manera fundamentada.
\end{itemize}

\subsection{Escalas de Riesgo Tradicionales}

Históricamente, la predicción de mortalidad en el \gls{iam} se ha basado en escalas clínicas derivadas de modelos de regresión logística:

\begin{itemize}
    \item \textbf{Escala GRACE} (\textit{Global Registry of Acute Coronary Events}): Desarrollada a partir de un registro multinacional, incorpora variables como edad, frecuencia cardíaca, presión arterial sistólica, creatinina sérica, clase Killip, elevación de biomarcadores cardíacos, desviación del segmento ST y parada cardiorrespiratoria \citep{fox2006grace}.
    
    \item \textbf{Escala TIMI} (\textit{Thrombolysis In Myocardial Infarction}): Derivada de ensayos clínicos, utiliza un conjunto reducido de variables fácilmente disponibles \citep{antman2000timi}.
    
    \item \textbf{ACTION Registry-GWTG}: Incorpora datos contemporáneos incluyendo tratamientos de reperfusión \citep{morrow2013action}.
\end{itemize}

\subsection{Limitaciones de los Enfoques Tradicionales}

A pesar de su amplia validación, las escalas clásicas presentan limitaciones relevantes:

\begin{enumerate}
    \item \textbf{Asunciones de linealidad}: Los modelos de regresión logística asumen relaciones lineales entre predictores y el logit de la probabilidad, lo cual puede no reflejar la complejidad biológica real.
    
    \item \textbf{Interacciones no capturadas}: Las interacciones de alto orden entre variables (ej. edad $\times$ función renal $\times$ biomarcadores) no se modelan adecuadamente.
    
    \item \textbf{Dependencia del contexto}: Escalas desarrolladas en poblaciones específicas pueden perder rendimiento en cohortes diferentes.
    
    \item \textbf{Variables faltantes}: Muchas escalas requieren determinaciones analíticas que no siempre están disponibles en entornos con recursos limitados.
\end{enumerate}

\subsection{Aprendizaje Automático en Predicción Clínica}

El \gls{ml} ofrece un paradigma alternativo que permite:

\begin{itemize}
    \item Capturar relaciones no lineales y de alta complejidad entre variables.
    \item Manejar grandes volúmenes de datos heterogéneos.
    \item Identificar patrones predictivos no evidentes mediante análisis convencional.
    \item Proporcionar interpretabilidad mediante técnicas de explicabilidad (\gls{shap}).
\end{itemize}

Estudios recientes han demostrado que algoritmos como \gls{xgb}, \gls{rf} y redes neuronales pueden superar el rendimiento de escalas tradicionales en la predicción de mortalidad por \gls{iam} \citep{zhu2024ml, oliveira2023ml}.

\subsection{Objetivos del Estudio}

\subsubsection{Objetivo General}

Desarrollar y validar modelos de aprendizaje automático para predecir la mortalidad intrahospitalaria en pacientes con infarto agudo de miocardio utilizando datos del Registro Cubano de Infarto Agudo de Miocardio (RECUIMA).

\subsubsection{Objetivos Específicos}

\begin{enumerate}
    \item Caracterizar el perfil clínico y demográfico de los pacientes con \gls{iam} en el registro RECUIMA.
    
    \item Implementar y comparar múltiples algoritmos de clasificación:
    \begin{itemize}
        \item Regresión Logística (baseline)
        \item Random Forest
        \item Gradient Boosting (XGBoost, LightGBM)
        \item Redes Neuronales Artificiales
        \item Modelos de ensamble
    \end{itemize}
    
    \item Evaluar el rendimiento predictivo mediante métricas de:
    \begin{itemize}
        \item Discriminación (AUROC, AUPRC)
        \item Calibración (curva de calibración, Brier score)
        \item Utilidad clínica (curvas de decisión)
    \end{itemize}
    
    \item Comparar el rendimiento de los modelos de ML con las escalas GRACE y TIMI.
    
    \item Identificar los predictores más importantes mediante análisis de explicabilidad \gls{shap}.
    
    \item Desarrollar una herramienta de predicción integrada en un dashboard interactivo.
\end{enumerate}

\subsection{Estructura del Informe}

El presente informe se organiza siguiendo las recomendaciones de las guías TRIPOD+AI para el reporte transparente de modelos de predicción clínica:

\begin{itemize}
    \item \textbf{Sección 2}: Estado del arte en predicción de mortalidad por IAM.
    \item \textbf{Sección 3}: Descripción detallada del dataset RECUIMA.
    \item \textbf{Sección 4}: Metodología de desarrollo y validación.
    \item \textbf{Sección 5}: Análisis exploratorio de datos.
    \item \textbf{Sección 6}: Preprocesamiento y preparación de datos.
    \item \textbf{Sección 7}: Desarrollo de modelos predictivos.
    \item \textbf{Sección 8}: Resultados y evaluación de rendimiento.
    \item \textbf{Sección 9}: Análisis de explicabilidad.
    \item \textbf{Sección 10}: Discusión de hallazgos.
    \item \textbf{Sección 11}: Conclusiones.
    \item \textbf{Sección 12}: Limitaciones y trabajo futuro.
\end{itemize}
