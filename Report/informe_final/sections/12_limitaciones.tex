% ============================================================================
% SECCIÓN 12: LIMITACIONES Y TRABAJO FUTURO
% ============================================================================

\section{Limitaciones y Trabajo Futuro}
\label{sec:limitaciones}

\subsection{Limitaciones del Estudio}

\subsubsection{Limitaciones del Diseño}

\begin{table}[H]
\centering
\caption{Resumen de limitaciones metodológicas}
\label{tab:limitaciones_metodologicas}
{\footnotesize
\setlength{\tabcolsep}{3pt}
\begin{tabular}{@{}p{2.2cm}p{4.2cm}p{4.2cm}@{}}
\toprule
\textbf{Limitación} & \textbf{Descripción} & \textbf{Impacto} \\
\midrule
Diseño retrosp. & \placeholder{Registros clínicos} & \placeholder{Sesgo selección} \\
Centro único & \placeholder{Una institución} & \placeholder{Generalización} \\
Periodo temp. & \placeholder{Datos [periodo]} & \placeholder{Drift temporal} \\
Valid. interna & \placeholder{División datos} & \placeholder{Optimismo} \\
\bottomrule
\end{tabular}
}
\end{table}

\subsubsection{Limitaciones de los Datos}

\begin{enumerate}
    \item \textbf{Datos faltantes}:
    \begin{placeholderblock}
    \begin{itemize}
        \item \placeholder{X} variables con $>$20\% de datos faltantes
        \item Imputación puede introducir sesgos
        \item Patrón de missingness posiblemente MNAR en algunas variables
    \end{itemize}
    \end{placeholderblock}
    
    \item \textbf{Calidad de datos}:
    \begin{placeholderblock}
    \begin{itemize}
        \item Errores de transcripción posibles en registros clínicos
        \item Variabilidad en protocolos de laboratorio
        \item Inconsistencias en definiciones de variables
    \end{itemize}
    \end{placeholderblock}
    
    \item \textbf{Variables no disponibles}:
    \begin{placeholderblock}
    \begin{itemize}
        \item Biomarcadores emergentes (ej. \placeholder{copeptina, GDF-15})
        \item Datos genéticos/farmacogenómicos
        \item Variables socioeconómicas
        \item Señales de ECG continuo
        \item Datos de imagen (ecocardiografía cuantitativa)
    \end{itemize}
    \end{placeholderblock}
    
    \item \textbf{Definición del outcome}:
    \begin{placeholderblock}
    \begin{itemize}
        \item Mortalidad intrahospitalaria no captura eventos post-alta
        \item Duración de estancia variable afecta la ventana de observación
        \item Causa de muerte no siempre documentada
    \end{itemize}
    \end{placeholderblock}
\end{enumerate}

\subsubsection{Limitaciones del Modelo}

\begin{enumerate}
    \item \textbf{Complejidad vs. interpretabilidad}:
    \begin{placeholderblock}
    Aunque SHAP mejora la interpretabilidad, los modelos de ensemble son inherentemente más complejos que la regresión logística, lo que puede dificultar la adopción clínica.
    \end{placeholderblock}
    
    \item \textbf{Predicción estática}:
    \begin{placeholderblock}
    El modelo realiza predicción en un punto temporal (ingreso), sin actualización dinámica durante la evolución del paciente.
    \end{placeholderblock}
    
    \item \textbf{Umbral de decisión}:
    \begin{placeholderblock}
    La selección del umbral óptimo depende del contexto clínico específico y los costos de errores de clasificación, que no fueron formalmente evaluados.
    \end{placeholderblock}
    
    \item \textbf{Asunción de estacionariedad}:
    \begin{placeholderblock}
    El modelo asume que las relaciones entre variables y outcome se mantienen estables, lo cual puede no ser cierto ante cambios en tratamientos o características poblacionales.
    \end{placeholderblock}
\end{enumerate}

\subsubsection{Limitaciones de Validación}

\begin{enumerate}
    \item \textbf{Ausencia de validación externa}:
    \begin{placeholderblock}
    El modelo no ha sido validado en cohortes de otras instituciones, regiones o países.
    \end{placeholderblock}
    
    \item \textbf{Ausencia de validación prospectiva}:
    \begin{placeholderblock}
    No se ha evaluado el rendimiento del modelo en un escenario de uso en tiempo real.
    \end{placeholderblock}
    
    \item \textbf{Ausencia de estudio de impacto}:
    \begin{placeholderblock}
    No se ha evaluado si el uso del modelo mejora los outcomes clínicos de los pacientes.
    \end{placeholderblock}
\end{enumerate}

\subsection{Sesgos Potenciales}

\begin{table}[H]
\centering
\caption{Análisis de sesgos potenciales}
\label{tab:sesgos}
{\footnotesize
\begin{tabular}{@{}p{2.5cm}p{1.5cm}p{5.5cm}@{}}
\toprule
\textbf{Tipo de sesgo} & \textbf{Riesgo} & \textbf{Mitigación aplicada} \\
\midrule
Sesgo de selección & \placeholder{Medio} & \placeholder{Criterios de inclusión claros} \\
Sesgo de información & \placeholder{Bajo} & \placeholder{Uso de variables objetivas} \\
Sesgo de confusión & \placeholder{Medio} & \placeholder{Ajuste multivariable, ML} \\
Sesgo de publicación & N/A & Reporte de todos los modelos \\
Overfitting & \placeholder{Bajo} & \placeholder{CV, regularización} \\
Data leakage & \placeholder{Bajo} & \placeholder{Separación temporal} \\
\bottomrule
\end{tabular}
}
\end{table}

\subsection{Trabajo Futuro}

\subsubsection{Corto Plazo (6--12 meses)}

\begin{enumerate}
    \item \textbf{Validación externa}:
    \begin{placeholderblock}
    \begin{itemize}
        \item Colaboración con otras instituciones para obtener cohortes de validación
        \item Evaluación de rendimiento en diferentes poblaciones
        \item Análisis de transportabilidad del modelo
    \end{itemize}
    \end{placeholderblock}
    
    \item \textbf{Refinamiento del modelo}:
    \begin{placeholderblock}
    \begin{itemize}
        \item Incorporación de feedback clínico
        \item Optimización de la calibración
        \item Ajuste de umbrales según contexto local
    \end{itemize}
    \end{placeholderblock}
    
    \item \textbf{Mejora de la interfaz}:
    \begin{placeholderblock}
    \begin{itemize}
        \item Usability testing con usuarios clínicos
        \item Integración con sistemas de historia clínica
        \item Generación automática de reportes
    \end{itemize}
    \end{placeholderblock}
\end{enumerate}

\subsubsection{Mediano Plazo (1--2 años)}

\begin{enumerate}
    \item \textbf{Estudio prospectivo}:
    \begin{placeholderblock}
    \begin{itemize}
        \item Implementación piloto en entorno clínico real
        \item Evaluación de rendimiento prospectivo
        \item Análisis de aceptabilidad y usabilidad
    \end{itemize}
    \end{placeholderblock}
    
    \item \textbf{Extensión del modelo}:
    \begin{placeholderblock}
    \begin{itemize}
        \item Predicción de otros outcomes (arritmias, shock, reingresos)
        \item Predicción dinámica con actualización durante hospitalización
        \item Incorporación de variables adicionales (imágenes, señales)
    \end{itemize}
    \end{placeholderblock}
    
    \item \textbf{Aspectos regulatorios}:
    \begin{placeholderblock}
    \begin{itemize}
        \item Certificación como dispositivo médico de software (si aplica)
        \item Documentación de conformidad regulatoria
        \item Evaluación de aspectos legales de responsabilidad
    \end{itemize}
    \end{placeholderblock}
\end{enumerate}

\subsubsection{Largo Plazo (2--5 años)}

\begin{enumerate}
    \item \textbf{Estudio de impacto clínico}:
    \begin{placeholderblock}
    \begin{itemize}
        \item Ensayo clínico aleatorizado evaluando si el uso del modelo mejora outcomes
        \item Análisis de costo-efectividad
        \item Evaluación de implementación a escala
    \end{itemize}
    \end{placeholderblock}
    
    \item \textbf{Aprendizaje federado}:
    \begin{placeholderblock}
    \begin{itemize}
        \item Desarrollo de modelos multicéntricos preservando privacidad
        \item Colaboración internacional
        \item Modelos adaptativos por región/población
    \end{itemize}
    \end{placeholderblock}
    
    \item \textbf{Medicina personalizada}:
    \begin{placeholderblock}
    \begin{itemize}
        \item Integración de datos genómicos y ómicos
        \item Predicción de respuesta a tratamientos específicos
        \item Optimización terapéutica individualizada
    \end{itemize}
    \end{placeholderblock}
    
    \item \textbf{Mantenimiento continuo}:
    \begin{placeholderblock}
    \begin{itemize}
        \item Sistemas de monitorización de drift
        \item Re-entrenamiento periódico
        \item Actualización con nuevas evidencias y variables
    \end{itemize}
    \end{placeholderblock}
\end{enumerate}

\subsection{Consideraciones para Implementación}

\begin{table}[H]
\centering
\caption{Checklist para implementación clínica}
\label{tab:checklist_implementacion}
\begin{tabular}{@{}lcc@{}}
\toprule
\textbf{Requisito} & \textbf{Estado actual} & \textbf{Prioridad} \\
\midrule
Validación externa en $\geq$1 cohorte independiente & \placeholder{Pendiente} & Alta \\
Validación prospectiva & \placeholder{Pendiente} & Alta \\
Integración con HCE/EHR & \placeholder{Pendiente/Parcial} & Alta \\
Evaluación de equidad por subgrupos & \placeholder{Completado/Parcial} & Alta \\
Documentación técnica completa & \placeholder{Completado} & Media \\
Manual de usuario & \placeholder{Completado} & Media \\
Proceso de re-entrenamiento definido & \placeholder{Pendiente} & Media \\
Marco de gobernanza establecido & \placeholder{Pendiente} & Media \\
Evaluación regulatoria (si aplica) & \placeholder{Pendiente} & Variable \\
Estudio de impacto clínico & \placeholder{Pendiente} & Largo plazo \\
\bottomrule
\end{tabular}
\end{table}

\subsection{Propuesta de Colaboración}

\begin{placeholderblock}
\textbf{[INVITACIÓN A COLABORACIÓN]}

Invitamos a investigadores, clínicos e instituciones interesadas en:
\begin{itemize}
    \item Participar en estudios de validación externa
    \item Colaborar en el desarrollo de modelos multicéntricos
    \item Contribuir al refinamiento de la herramienta
    \item Implementar y evaluar el modelo en sus contextos locales
\end{itemize}

Para más información:
\begin{itemize}
    \item Repositorio: \placeholder{URL del repositorio}
    \item Contacto: \placeholder{email del equipo}
    \item Documentación: \placeholder{URL de documentación}
\end{itemize}
\end{placeholderblock}
