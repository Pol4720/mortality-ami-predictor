% ============================================================================
% SECCIÓN 08: RESULTADOS
% ============================================================================

\section{Resultados}
\label{sec:resultados}

Esta sección presenta los resultados de la evaluación del modelo final en el conjunto de test, que permaneció completamente reservado durante el desarrollo.

\subsection{Características de la Cohorte de Test}

\begin{table}[H]
\centering
\caption{Características basales del conjunto de test}
\label{tab:caracteristicas_test}
\begin{tabular}{@{}lccc@{}}
\toprule
\textbf{Variable} & \textbf{Supervivientes} & \textbf{Fallecidos} & \textbf{p-valor} \\
 & (n = \placeholder{XX}) & (n = \placeholder{XX}) & \\
\midrule
Edad, años (media ± DE) & \placeholder{XX.X ± XX.X} & \placeholder{XX.X ± XX.X} & \placeholder{X.XXX} \\
Sexo masculino, n (\%) & \placeholder{n (XX.X\%)} & \placeholder{n (XX.X\%)} & \placeholder{X.XXX} \\
Hipertensión, n (\%) & \placeholder{n (XX.X\%)} & \placeholder{n (XX.X\%)} & \placeholder{X.XXX} \\
Diabetes, n (\%) & \placeholder{n (XX.X\%)} & \placeholder{n (XX.X\%)} & \placeholder{X.XXX} \\
Killip III-IV, n (\%) & \placeholder{n (XX.X\%)} & \placeholder{n (XX.X\%)} & \placeholder{X.XXX} \\
\bottomrule
\end{tabular}
\end{table}

\subsection{Métricas de Discriminación}

\subsubsection{Rendimiento Global}

\begin{table}[H]
\centering
\caption{Métricas de discriminación del modelo final en conjunto de test}
\label{tab:metricas_discriminacion_test}
\begin{tabular}{@{}lcc@{}}
\toprule
\textbf{Métrica} & \textbf{Valor} & \textbf{IC 95\%} \\
\midrule
\rowcolor{accentgold!20}
\textbf{AUROC} & \placeholder{\textbf{0.XXX}} & \placeholder{[0.XXX -- 0.XXX]} \\
AUPRC & \placeholder{0.XXX} & \placeholder{[0.XXX -- 0.XXX]} \\
\midrule
\multicolumn{3}{l}{\textit{Al umbral óptimo (Youden): \placeholder{0.XXX}}} \\
Sensibilidad & \placeholder{0.XXX} & \placeholder{[0.XXX -- 0.XXX]} \\
Especificidad & \placeholder{0.XXX} & \placeholder{[0.XXX -- 0.XXX]} \\
VPP (Precisión) & \placeholder{0.XXX} & \placeholder{[0.XXX -- 0.XXX]} \\
VPN & \placeholder{0.XXX} & \placeholder{[0.XXX -- 0.XXX]} \\
F1-Score & \placeholder{0.XXX} & \placeholder{[0.XXX -- 0.XXX]} \\
Accuracy & \placeholder{0.XXX} & \placeholder{[0.XXX -- 0.XXX]} \\
\bottomrule
\end{tabular}
\end{table}

\begin{keypoint}
\textbf{Resultado principal:} El modelo \placeholder{nombre del modelo} alcanzó un AUROC de \placeholder{0.XXX} (IC 95\%: \placeholder{0.XXX--0.XXX}) en el conjunto de test independiente, superando al score GRACE (\placeholder{0.XXX}) en \placeholder{XX} puntos porcentuales.
\end{keypoint}

\subsubsection{Curva ROC}

\begin{figure}[H]
\centering
\begin{placeholderblock}
\textbf{[INSERTAR CURVA ROC]}

Gráfico con:
\begin{itemize}
    \item Curva ROC del modelo final
    \item Curva ROC del modelo baseline (regresión logística)
    \item Curva ROC del score GRACE (si calculado)
    \item Línea diagonal de referencia (AUC = 0.5)
    \item Leyenda con AUC de cada modelo
    \item Intervalo de confianza sombreado (bootstrap)
    \item Punto óptimo marcado (umbral de Youden)
\end{itemize}
\end{placeholderblock}
\caption{Curvas ROC comparativas en conjunto de test}
\label{fig:curva_roc}
\end{figure}

\subsubsection{Curva Precision-Recall}

\begin{figure}[H]
\centering
\begin{placeholderblock}
\textbf{[INSERTAR CURVA PRECISION-RECALL]}

Gráfico mostrando:
\begin{itemize}
    \item Curva PR del modelo final
    \item Curva PR del baseline
    \item Línea de referencia (prevalencia de eventos)
    \item AUPRC en leyenda
\end{itemize}

\textit{Esta curva es especialmente relevante dado el desbalance de clases.}
\end{placeholderblock}
\caption{Curvas Precision-Recall comparativas}
\label{fig:curva_pr}
\end{figure}

\subsubsection{Matriz de Confusión}

\begin{figure}[H]
\centering
\begin{placeholderblock}
\textbf{[INSERTAR MATRIZ DE CONFUSIÓN]}

Matriz 2x2 con:
\begin{itemize}
    \item Verdaderos Positivos (VP)
    \item Verdaderos Negativos (VN)
    \item Falsos Positivos (FP)
    \item Falsos Negativos (FN)
\end{itemize}

Mostrar valores absolutos y porcentajes.
Al umbral óptimo seleccionado.
\end{placeholderblock}
\caption{Matriz de confusión del modelo final (umbral = \placeholder{0.XXX})}
\label{fig:matriz_confusion}
\end{figure}

\begin{table}[H]
\centering
\caption{Matriz de confusión -- Valores numéricos}
\label{tab:matriz_confusion}
\begin{tabular}{@{}l|cc|c@{}}
\toprule
& \multicolumn{2}{c|}{\textbf{Predicho}} & \\
\textbf{Real} & Superviviente & Fallecido & \textbf{Total} \\
\midrule
Superviviente & \placeholder{VN = XXX} & \placeholder{FP = XX} & \placeholder{XXX} \\
Fallecido & \placeholder{FN = XX} & \placeholder{VP = XX} & \placeholder{XX} \\
\midrule
\textbf{Total} & \placeholder{XXX} & \placeholder{XX} & \placeholder{N} \\
\bottomrule
\end{tabular}
\end{table}

\subsection{Métricas de Calibración}

\subsubsection{Curva de Calibración}

\begin{figure}[H]
\centering
\begin{placeholderblock}
\textbf{[INSERTAR CURVA DE CALIBRACIÓN]}

Gráfico mostrando:
\begin{itemize}
    \item Probabilidades predichas vs. frecuencias observadas por deciles
    \item Línea de calibración perfecta (diagonal)
    \item Barras de error (IC 95\%)
    \item Histograma de probabilidades en panel inferior
\end{itemize}
\end{placeholderblock}
\caption{Curva de calibración del modelo final en conjunto de test}
\label{fig:calibracion_test}
\end{figure}

\subsubsection{Métricas de Calibración}

\begin{table}[H]
\centering
\caption{Métricas de calibración}
\label{tab:metricas_calibracion}
\begin{tabular}{@{}lc@{}}
\toprule
\textbf{Métrica} & \textbf{Valor} \\
\midrule
Brier Score & \placeholder{0.XXX} \\
Brier Skill Score & \placeholder{0.XXX} \\
Calibration slope & \placeholder{X.XX} \\
Calibration intercept & \placeholder{X.XX} \\
Hosmer-Lemeshow $\chi^2$ (p-valor) & \placeholder{X.XX (p = X.XXX)} \\
Expected Calibration Error (ECE) & \placeholder{0.XXX} \\
\bottomrule
\end{tabular}
\end{table}

\subsection{Comparación con Modelos de Referencia}

\begin{table}[H]
\centering
\caption{Comparación del modelo final con benchmarks}
\label{tab:comparacion_benchmarks}
\begin{tabular}{@{}lcccc@{}}
\toprule
\textbf{Modelo} & \textbf{AUROC} & \textbf{AUPRC} & \textbf{Sensibilidad} & \textbf{Especificidad} \\
\midrule
Score GRACE & \placeholder{0.XXX} & \placeholder{0.XXX} & \placeholder{0.XXX} & \placeholder{0.XXX} \\
Reg. Logística & \placeholder{0.XXX} & \placeholder{0.XXX} & \placeholder{0.XXX} & \placeholder{0.XXX} \\
Random Forest & \placeholder{0.XXX} & \placeholder{0.XXX} & \placeholder{0.XXX} & \placeholder{0.XXX} \\
\rowcolor{accentgold!20}
\textbf{Modelo final} & \placeholder{\textbf{0.XXX}} & \placeholder{\textbf{0.XXX}} & \placeholder{\textbf{0.XXX}} & \placeholder{\textbf{0.XXX}} \\
\bottomrule
\end{tabular}
\end{table}

\subsubsection{Tests de Significancia Estadística}

\begin{table}[H]
\centering
\caption{Comparación estadística de AUROCs (DeLong test)}
\label{tab:delong_test}
\begin{tabular}{@{}lccc@{}}
\toprule
\textbf{Comparación} & \textbf{$\Delta$AUROC} & \textbf{IC 95\%} & \textbf{p-valor} \\
\midrule
Modelo final vs. GRACE & \placeholder{+0.XXX} & \placeholder{[0.XXX -- 0.XXX]} & \placeholder{$<$0.001} \\
Modelo final vs. Reg. Log. & \placeholder{+0.XXX} & \placeholder{[0.XXX -- 0.XXX]} & \placeholder{X.XXX} \\
\bottomrule
\end{tabular}
\end{table}

\subsection{Análisis de Utilidad Clínica}

\subsubsection{Decision Curve Analysis}

\begin{figure}[H]
\centering
\begin{placeholderblock}
\textbf{[INSERTAR DECISION CURVE ANALYSIS]}

Gráfico mostrando:
\begin{itemize}
    \item Eje X: Probabilidad umbral
    \item Eje Y: Beneficio neto estandarizado
    \item Curva del modelo final
    \item Curva de GRACE score
    \item Línea ``Tratar a todos''
    \item Línea ``Tratar a ninguno''
\end{itemize}

Rango de umbrales clínicamente relevantes: \placeholder{5\%--30\%}
\end{placeholderblock}
\caption{Análisis de curva de decisión}
\label{fig:decision_curve}
\end{figure}

\begin{keypoint}
El modelo muestra beneficio neto positivo sobre las estrategias de ``tratar a todos'' o ``tratar a ninguno'' en el rango de umbrales de probabilidad de \placeholder{X\% a XX\%}, correspondiente al rango de utilidad clínica relevante.
\end{keypoint}

\subsubsection{Net Reclassification Index (NRI)}

\begin{table}[H]
\centering
\caption{Índices de reclasificación respecto a modelo baseline}
\label{tab:nri}
\begin{tabular}{@{}lcc@{}}
\toprule
\textbf{Índice} & \textbf{Valor} & \textbf{IC 95\%} \\
\midrule
NRI (eventos) & \placeholder{+0.XXX} & \placeholder{[0.XXX -- 0.XXX]} \\
NRI (no eventos) & \placeholder{+0.XXX} & \placeholder{[0.XXX -- 0.XXX]} \\
NRI total & \placeholder{+0.XXX} & \placeholder{[0.XXX -- 0.XXX]} \\
IDI & \placeholder{+0.XXX} & \placeholder{[0.XXX -- 0.XXX]} \\
\bottomrule
\end{tabular}
\end{table}

\subsection{Análisis de Subgrupos}

\begin{table}[H]
\centering
\caption{Rendimiento del modelo por subgrupos}
\label{tab:subgrupos}
\begin{tabular}{@{}lccc@{}}
\toprule
\textbf{Subgrupo} & \textbf{N} & \textbf{AUROC} & \textbf{IC 95\%} \\
\midrule
\multicolumn{4}{l}{\textit{Por sexo}} \\
\quad Masculino & \placeholder{n} & \placeholder{0.XXX} & \placeholder{[0.XXX -- 0.XXX]} \\
\quad Femenino & \placeholder{n} & \placeholder{0.XXX} & \placeholder{[0.XXX -- 0.XXX]} \\
\midrule
\multicolumn{4}{l}{\textit{Por edad}} \\
\quad $<$65 años & \placeholder{n} & \placeholder{0.XXX} & \placeholder{[0.XXX -- 0.XXX]} \\
\quad 65--79 años & \placeholder{n} & \placeholder{0.XXX} & \placeholder{[0.XXX -- 0.XXX]} \\
\quad $\geq$80 años & \placeholder{n} & \placeholder{0.XXX} & \placeholder{[0.XXX -- 0.XXX]} \\
\midrule
\multicolumn{4}{l}{\textit{Por tipo de IAM}} \\
\quad IAMCEST & \placeholder{n} & \placeholder{0.XXX} & \placeholder{[0.XXX -- 0.XXX]} \\
\quad IAMSEST & \placeholder{n} & \placeholder{0.XXX} & \placeholder{[0.XXX -- 0.XXX]} \\
\midrule
\multicolumn{4}{l}{\textit{Por Killip}} \\
\quad Killip I-II & \placeholder{n} & \placeholder{0.XXX} & \placeholder{[0.XXX -- 0.XXX]} \\
\quad Killip III-IV & \placeholder{n} & \placeholder{0.XXX} & \placeholder{[0.XXX -- 0.XXX]} \\
\bottomrule
\end{tabular}
\end{table}

\begin{figure}[H]
\centering
\begin{placeholderblock}
\textbf{[INSERTAR FOREST PLOT DE SUBGRUPOS]}

Forest plot mostrando AUROC con IC 95\% para cada subgrupo, facilitando la comparación visual.
\end{placeholderblock}
\caption{Rendimiento del modelo por subgrupos (Forest plot)}
\label{fig:forest_subgrupos}
\end{figure}

\subsection{Análisis de Errores}

\subsubsection{Características de Falsos Negativos}

\begin{table}[H]
\centering
\caption{Características de pacientes con falsos negativos}
\label{tab:falsos_negativos}
\begin{tabular}{@{}lcc@{}}
\toprule
\textbf{Variable} & \textbf{FN (n=\placeholder{XX})} & \textbf{VP (n=\placeholder{XX})} \\
\midrule
Edad, años & \placeholder{XX.X ± XX.X} & \placeholder{XX.X ± XX.X} \\
Killip III-IV, \% & \placeholder{XX.X\%} & \placeholder{XX.X\%} \\
Creatinina, mg/dL & \placeholder{X.XX ± X.XX} & \placeholder{X.XX ± X.XX} \\
FEVI, \% & \placeholder{XX.X ± XX.X} & \placeholder{XX.X ± XX.X} \\
\bottomrule
\end{tabular}
\end{table}

\begin{placeholderblock}
\textbf{[ANÁLISIS CUALITATIVO DE ERRORES]}

Describir:
\begin{itemize}
    \item Patrones identificados en falsos negativos
    \item Posibles explicaciones clínicas
    \item Casos extremos o atípicos
\end{itemize}
\end{placeholderblock}

\subsection{Resumen de Resultados}

\begin{keypoint}
\textbf{Resumen de resultados principales:}

\begin{enumerate}
    \item \textbf{Discriminación}: El modelo \placeholder{nombre} alcanzó un AUROC de \placeholder{0.XXX}, superando significativamente al score GRACE (p \placeholder{$<$0.001}).
    
    \item \textbf{Calibración}: El modelo mostró buena calibración con Brier Score de \placeholder{0.XXX} y calibration slope de \placeholder{X.XX}.
    
    \item \textbf{Utilidad clínica}: El Decision Curve Analysis demostró beneficio neto en el rango de umbrales \placeholder{X\%--XX\%}.
    
    \item \textbf{Generalización}: El rendimiento fue consistente entre subgrupos de \placeholder{sexo/edad/tipo IAM}.
    
    \item \textbf{Puntos de mejora}: Se identificaron \placeholder{XX} falsos negativos, principalmente en pacientes con \placeholder{característica}.
\end{enumerate}
\end{keypoint}
