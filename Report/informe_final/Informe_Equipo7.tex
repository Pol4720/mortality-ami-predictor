% ============================================================================
% INFORME DE INVESTIGACIÓN: Predicción de Mortalidad Intrahospitalaria 
% por Infarto Agudo de Miocardio mediante Aprendizaje Automático
% ============================================================================
% Estructura basada en guías TRIPOD+AI (2024) para modelos de predicción clínica
% Formato APA 7ma edición para referencias bibliográficas
% ============================================================================

\documentclass[12pt,a4paper]{article}

% ============================================================================
% PAQUETES ESENCIALES
% ============================================================================

% Idioma y codificación
\usepackage[spanish,es-tabla]{babel}
\usepackage[utf8]{inputenc}
\usepackage[T1]{fontenc}
\usepackage{lmodern}
\usepackage{microtype}

% Configuraciones para manejar overfull boxes
\setlength{\emergencystretch}{3em}
\tolerance=2000
\hbadness=2000
\sloppy

% Geometría y formato
\usepackage[
    top=2.5cm,
    bottom=2.5cm,
    left=3cm,
    right=2.5cm,
    headheight=14pt
]{geometry}
\usepackage{setspace}
\onehalfspacing

% Matemáticas y símbolos
\usepackage{amsmath,amssymb,amsfonts}
\usepackage{siunitx}
\sisetup{
    output-decimal-marker = {,},
    group-separator = {.},
    group-minimum-digits = 4
}

% Colores
\usepackage[dvipsnames]{xcolor}

% Tablas y figuras
\usepackage{array}
\usepackage{booktabs}
\usepackage{longtable}
\usepackage{multirow}
\usepackage{tabularx}
\usepackage{float}
\usepackage{graphicx}
\graphicspath{{../complemento_del_informe_final/Comparacion_Escalas_Internacionales/}{../complemento_del_informe_final/Propuesta_de_seleccion_de_variables/}{../complemento_del_informe_final/seccion 5/}{./images/}}
\usepackage{subcaption}

% Colores institucionales
\definecolor{primaryblue}{RGB}{0,51,102}
\definecolor{secondaryblue}{RGB}{51,102,153}
\definecolor{accentgold}{RGB}{204,153,0}
\definecolor{lightgray}{RGB}{245,245,245}
\definecolor{darkgray}{RGB}{64,64,64}

% Encabezados y pies de página
\usepackage{fancyhdr}
\pagestyle{fancy}
\fancyhf{}
\fancyhead[L]{\small\textcolor{darkgray}{Predicción de Mortalidad en IAM}}
\fancyhead[R]{\small\textcolor{darkgray}{Equipo 7}}
\fancyfoot[C]{\thepage}
\renewcommand{\headrulewidth}{0.4pt}
\renewcommand{\footrulewidth}{0pt}

% Referencias y citas - APA 7ma edición
\usepackage[
    backend=biber,
    style=apa,
    sorting=nyt,
    natbib=true
]{biblatex}
\addbibresource{referencias.bib}
% Mapeo de idioma para APA (comentado si causa errores)
% \DeclareLanguageMapping{spanish}{spanish-apa}

% Hipervínculos
\usepackage[
    colorlinks=true,
    linkcolor=primaryblue,
    citecolor=secondaryblue,
    urlcolor=accentgold,
    bookmarks=true,
    bookmarksopen=true
]{hyperref}
\usepackage{bookmark}

% Listados de código
\usepackage{listings}
\lstset{
    basicstyle=\ttfamily\footnotesize,
    backgroundcolor=\color{lightgray},
    breaklines=true,
    breakatwhitespace=false,
    postbreak=\mbox{\textcolor{red}{$\hookrightarrow$}\space},
    frame=single,
    numbers=left,
    numberstyle=\tiny\color{gray},
    keywordstyle=\color{primaryblue}\bfseries,
    commentstyle=\color{gray}\itshape,
    stringstyle=\color{accentgold},
    columns=flexible,
    keepspaces=true
}

% Redefinir verbatim para usar fuente más pequeña
\usepackage{fancyvrb}
\RecustomVerbatimEnvironment{verbatim}{Verbatim}{fontsize=\footnotesize}

% Algoritmos
\usepackage{algorithm}
\usepackage{algpseudocode}
\floatname{algorithm}{Algoritmo}

% Cajas de texto destacado
\usepackage{tcolorbox}
\tcbuselibrary{skins,breakable}

\newtcolorbox{keypoint}{
    colback=lightgray,
    colframe=primaryblue,
    fonttitle=\bfseries,
    title=Punto Clave,
    breakable
}

% Comando placeholder para texto inline a completar
\newcommand{\placeholder}[1]{\colorbox{yellow!30}{\textbf{[#1]}}}

% Entorno placeholderblock para bloques de texto más largos (usando tcolorbox)
\newtcolorbox{placeholderblock}[1][COMPLETAR]{
    colback=yellow!10,
    colframe=yellow!60,
    fonttitle=\bfseries\small,
    title=#1,
    breakable,
    enhanced
}

% Títulos de secciones
\usepackage{titlesec}
\titleformat{\section}
    {\normalfont\Large\bfseries\color{primaryblue}}
    {\thesection}{1em}{}
\titleformat{\subsection}
    {\normalfont\large\bfseries\color{secondaryblue}}
    {\thesubsection}{1em}{}
\titleformat{\subsubsection}
    {\normalfont\normalsize\bfseries\color{darkgray}}
    {\thesubsubsection}{1em}{}

% Epígrafes de figuras y tablas
\usepackage[
    font=small,
    labelfont=bf,
    labelsep=period,
    justification=centering
]{caption}

% Glosario y acrónimos
\usepackage[acronym,toc]{glossaries}
\makeglossaries

% Definición de acrónimos
\newacronym{iam}{IAM}{Infarto Agudo de Miocardio}
\newacronym{ml}{ML}{Machine Learning}
\newacronym{auroc}{AUROC}{Área Bajo la Curva ROC}
\newacronym{shap}{SHAP}{SHapley Additive exPlanations}
\newacronym{xgb}{XGBoost}{Extreme Gradient Boosting}
\newacronym{rf}{RF}{Random Forest}
\newacronym{lr}{LR}{Regresión Logística}
\newacronym{nn}{NN}{Red Neuronal}
\newacronym{eda}{EDA}{Análisis Exploratorio de Datos}
\newacronym{cv}{CV}{Validación Cruzada}
\newacronym{grace}{GRACE}{Global Registry of Acute Coronary Events}
\newacronym{timi}{TIMI}{Thrombolysis In Myocardial Infarction}

% ============================================================================
% METADATOS DEL DOCUMENTO
% ============================================================================
\title{
    \vspace{-1cm}
    % \includegraphics[width=0.3\textwidth]{logo_universidad.png}\\[1cm] % Descomentar si existe el logo
    {\Huge\bfseries\color{primaryblue}Predicción de Mortalidad Intrahospitalaria\\por Infarto Agudo de Miocardio\\mediante Aprendizaje Automático}\\[0.5cm]
    {\Large\color{secondaryblue}Informe de Investigación}\\[0.3cm]
    {\large\color{darkgray}Basado en el Registro Cubano de Infarto Agudo de Miocardio (RECUIMA)}
}

\author{
    \textbf{Equipo 7}\\[0.3cm]
    \begin{tabular}{l}
    Richard Alejandro Matos Arderí\\
    Abel Ponce González\\
    Abraham Romero Imbert\\
    Eveliz Espinaco Milián\\
    Michell Viu Ramirez\\
    Ernesto Abreu Peraza\\
    Eduardo Brito Labrada\\
    \end{tabular}\\[0.5cm]
    \textit{Curso de Machine Learning}\\
    \textit{Universidad de La Habana}
}

\date{}

% ============================================================================
% INICIO DEL DOCUMENTO
% ============================================================================
\begin{document}

% Página de título
\maketitle
\thispagestyle{empty}
\newpage

% ============================================================================
% RESUMEN Y ABSTRACT
% ============================================================================
% ============================================================================
% SECCIÓN 00: RESUMEN Y ABSTRACT
% ============================================================================

\begin{abstract}
\noindent
\textbf{Contexto:} El infarto agudo de miocardio (IAM) representa una de las principales causas de mortalidad cardiovascular a nivel mundial. La predicción temprana de la mortalidad intrahospitalaria permite optimizar la estratificación de riesgo y la asignación de recursos terapéuticos.

\vspace{0.3cm}
\noindent
\textbf{Objetivo:} Desarrollar y validar modelos de aprendizaje automático para predecir la mortalidad intrahospitalaria en pacientes con IAM utilizando datos del Registro Cubano de Infarto Agudo de Miocardio (RECUIMA).

\vspace{0.3cm}
\noindent
\textbf{Métodos:} Se analizaron \textbf{[PLACEHOLDER: N]} pacientes con diagnóstico de IAM registrados entre \textbf{[PLACEHOLDER: período]}. Se implementaron múltiples algoritmos de clasificación incluyendo Regresión Logística, Random Forest, XGBoost, y Redes Neuronales. El rendimiento se evaluó mediante validación cruzada estratificada y métricas de discriminación (AUROC, AUPRC), calibración y utilidad clínica.

\vspace{0.3cm}
\noindent
\textbf{Resultados:} El modelo con mejor rendimiento fue \textbf{[PLACEHOLDER: nombre del modelo]}, alcanzando un AUROC de \textbf{[PLACEHOLDER: valor $\pm$ IC95\%]} en el conjunto de prueba. Las variables más predictivas identificadas mediante análisis SHAP fueron \textbf{[PLACEHOLDER: top 5 variables]}. El modelo superó el rendimiento de las escalas clásicas GRACE y TIMI en esta población.

\vspace{0.3cm}
\noindent
\textbf{Conclusiones:} Los modelos de aprendizaje automático demuestran capacidad superior para predecir mortalidad intrahospitalaria por IAM en comparación con escalas tradicionales. La interpretabilidad mediante SHAP facilita la adopción clínica al identificar los factores de riesgo más relevantes.

\vspace{0.5cm}
\noindent
\textbf{Palabras clave:} infarto agudo de miocardio, mortalidad intrahospitalaria, aprendizaje automático, predicción de riesgo, SHAP, XGBoost

\end{abstract}

\newpage

% ============================================================================
% ABSTRACT EN INGLÉS
% ============================================================================

\begin{center}
{\Large\bfseries\color{primaryblue} Abstract}
\end{center}

\vspace{0.3cm}
\noindent
\textbf{Background:} Acute myocardial infarction (AMI) remains one of the leading causes of cardiovascular mortality worldwide. Early prediction of in-hospital mortality enables optimal risk stratification and therapeutic resource allocation.

\vspace{0.3cm}
\noindent
\textbf{Objective:} To develop and validate machine learning models for predicting in-hospital mortality in AMI patients using data from the Cuban Registry of Acute Myocardial Infarction (RECUIMA).

\vspace{0.3cm}
\noindent
\textbf{Methods:} We analyzed \textbf{[PLACEHOLDER: N]} patients diagnosed with AMI between \textbf{[PLACEHOLDER: period]}. Multiple classification algorithms were implemented, including Logistic Regression, Random Forest, XGBoost, and Neural Networks. Performance was evaluated using stratified cross-validation with discrimination metrics (AUROC, AUPRC), calibration, and clinical utility.

\vspace{0.3cm}
\noindent
\textbf{Results:} The best-performing model was \textbf{[PLACEHOLDER: model name]}, achieving an AUROC of \textbf{[PLACEHOLDER: value $\pm$ 95\%CI]} on the test set. The most predictive variables identified through SHAP analysis were \textbf{[PLACEHOLDER: top 5 variables]}. The model outperformed traditional GRACE and TIMI scores in this population.

\vspace{0.3cm}
\noindent
\textbf{Conclusions:} Machine learning models demonstrate superior capability for predicting in-hospital mortality from AMI compared to traditional scores. Interpretability through SHAP facilitates clinical adoption by identifying the most relevant risk factors.

\vspace{0.5cm}
\noindent
\textbf{Keywords:} acute myocardial infarction, in-hospital mortality, machine learning, risk prediction, SHAP, XGBoost

\newpage


% Tabla de contenidos
\tableofcontents
\newpage

% Lista de figuras y tablas
\listoffigures
\listoftables
\newpage

% ============================================================================
% CUERPO DEL INFORME
% ============================================================================

% Introducción
% ============================================================================
% SECCIÓN 01: INTRODUCCIÓN
% ============================================================================

\section{Introducción}
\label{sec:introduccion}

\subsection{Contexto y Motivación}

El infarto agudo de miocardio (\gls{iam}) constituye una emergencia cardiovascular que representa una de las principales causas de morbilidad y mortalidad a nivel mundial \citep{fox2006grace}. Según la Organización Mundial de la Salud, las enfermedades cardiovasculares causan aproximadamente 17,9 millones de muertes anuales, de las cuales una proporción significativa corresponde a eventos coronarios agudos.

La identificación temprana de pacientes con alto riesgo de mortalidad intrahospitalaria resulta fundamental para:

\begin{itemize}
    \item \textbf{Optimizar la estratificación de riesgo}: Clasificar pacientes según su probabilidad de eventos adversos.
    \item \textbf{Guiar decisiones terapéuticas}: Seleccionar estrategias de reperfusión y tratamiento farmacológico.
    \item \textbf{Asignar recursos}: Determinar niveles de monitorización y cuidados intensivos.
    \item \textbf{Informar al paciente y familia}: Comunicar pronóstico de manera fundamentada.
\end{itemize}

\subsection{Escalas de Riesgo Tradicionales}

Históricamente, la predicción de mortalidad en el \gls{iam} se ha basado en escalas clínicas derivadas de modelos de regresión logística:

\begin{itemize}
    \item \textbf{Escala GRACE} (\textit{Global Registry of Acute Coronary Events}): Desarrollada a partir de un registro multinacional, incorpora variables como edad, frecuencia cardíaca, presión arterial sistólica, creatinina sérica, clase Killip, elevación de biomarcadores cardíacos, desviación del segmento ST y parada cardiorrespiratoria \citep{fox2006grace}.
    
    \item \textbf{Escala TIMI} (\textit{Thrombolysis In Myocardial Infarction}): Derivada de ensayos clínicos, utiliza un conjunto reducido de variables fácilmente disponibles \citep{antman2000timi}.
    
    \item \textbf{ACTION Registry-GWTG}: Incorpora datos contemporáneos incluyendo tratamientos de reperfusión \citep{morrow2013action}.
\end{itemize}

\subsection{Limitaciones de los Enfoques Tradicionales}

A pesar de su amplia validación, las escalas clásicas presentan limitaciones relevantes:

\begin{enumerate}
    \item \textbf{Asunciones de linealidad}: Los modelos de regresión logística asumen relaciones lineales entre predictores y el logit de la probabilidad, lo cual puede no reflejar la complejidad biológica real.
    
    \item \textbf{Interacciones no capturadas}: Las interacciones de alto orden entre variables (ej. edad $\times$ función renal $\times$ biomarcadores) no se modelan adecuadamente.
    
    \item \textbf{Dependencia del contexto}: Escalas desarrolladas en poblaciones específicas pueden perder rendimiento en cohortes diferentes.
    
    \item \textbf{Variables faltantes}: Muchas escalas requieren determinaciones analíticas que no siempre están disponibles en entornos con recursos limitados.
\end{enumerate}

\subsection{Aprendizaje Automático en Predicción Clínica}

El \gls{ml} ofrece un paradigma alternativo que permite:

\begin{itemize}
    \item Capturar relaciones no lineales y de alta complejidad entre variables.
    \item Manejar grandes volúmenes de datos heterogéneos.
    \item Identificar patrones predictivos no evidentes mediante análisis convencional.
    \item Proporcionar interpretabilidad mediante técnicas de explicabilidad (\gls{shap}).
\end{itemize}

Estudios recientes han demostrado que algoritmos como \gls{xgb}, \gls{rf} y redes neuronales pueden superar el rendimiento de escalas tradicionales en la predicción de mortalidad por \gls{iam} \citep{zhu2024ml, oliveira2023ml}.

\subsection{Objetivos del Estudio}

\subsubsection{Objetivo General}

Desarrollar y validar modelos de aprendizaje automático para predecir la mortalidad intrahospitalaria en pacientes con infarto agudo de miocardio utilizando datos del Registro Cubano de Infarto Agudo de Miocardio (RECUIMA).

\subsubsection{Objetivos Específicos}

\begin{enumerate}
    \item Caracterizar el perfil clínico y demográfico de los pacientes con \gls{iam} en el registro RECUIMA.
    
    \item Implementar y comparar múltiples algoritmos de clasificación:
    \begin{itemize}
        \item Regresión Logística (baseline)
        \item Random Forest
        \item Gradient Boosting (XGBoost, LightGBM)
        \item Redes Neuronales Artificiales
        \item Modelos de ensamble
    \end{itemize}
    
    \item Evaluar el rendimiento predictivo mediante métricas de:
    \begin{itemize}
        \item Discriminación (AUROC, AUPRC)
        \item Calibración (curva de calibración, Brier score)
        \item Utilidad clínica (curvas de decisión)
    \end{itemize}
    
    \item Comparar el rendimiento de los modelos de ML con las escalas GRACE y TIMI.
    
    \item Identificar los predictores más importantes mediante análisis de explicabilidad \gls{shap}.
    
    \item Desarrollar una herramienta de predicción integrada en un dashboard interactivo.
\end{enumerate}

\subsection{Estructura del Informe}

El presente informe se organiza siguiendo las recomendaciones de las guías TRIPOD+AI para el reporte transparente de modelos de predicción clínica:

\begin{itemize}
    \item \textbf{Sección 2}: Estado del arte en predicción de mortalidad por IAM.
    \item \textbf{Sección 3}: Descripción detallada del dataset RECUIMA.
    \item \textbf{Sección 4}: Metodología de desarrollo y validación.
    \item \textbf{Sección 5}: Análisis exploratorio de datos.
    \item \textbf{Sección 6}: Preprocesamiento y preparación de datos.
    \item \textbf{Sección 7}: Desarrollo de modelos predictivos.
    \item \textbf{Sección 8}: Resultados y evaluación de rendimiento.
    \item \textbf{Sección 9}: Análisis de explicabilidad.
    \item \textbf{Sección 10}: Discusión de hallazgos.
    \item \textbf{Sección 11}: Conclusiones.
    \item \textbf{Sección 12}: Limitaciones y trabajo futuro.
\end{itemize}


% Estado del Arte
% ============================================================================
% SECCIÓN 02: ESTADO DEL ARTE
% ============================================================================

\section{Estado del Arte}
\label{sec:estado_del_arte}

% NOTA: Esta sección integra y expande el contenido del archivo 
% estado_del_arte_ml_infarcto.tex proporcionado

\subsection{Modelos Clásicos de Predicción de Mortalidad en IAM}

Las escalas de riesgo clásicas constituyen la base sobre la que se ha desarrollado la predicción pronóstica en el \gls{iam}. Entre las más utilizadas destacan:

\subsubsection{Escala GRACE}

El \textit{Global Registry of Acute Coronary Events} \citep{fox2006grace} representa el estándar de referencia actual para la estratificación de riesgo en síndromes coronarios agudos. La escala GRACE incorpora las siguientes variables:

\begin{table}[H]
\centering
\caption{Variables incluidas en la escala GRACE}
\label{tab:grace_variables}
\begin{tabular}{@{}lll@{}}
\toprule
\textbf{Variable} & \textbf{Tipo} & \textbf{Rango de puntos} \\
\midrule
Edad & Continua & 0--100 \\
Frecuencia cardíaca & Continua & 0--46 \\
Presión arterial sistólica & Continua & 0--58 \\
Creatinina sérica & Continua & 0--28 \\
Clase Killip & Categórica & 0--59 \\
Parada cardiorrespiratoria al ingreso & Binaria & 0--39 \\
Desviación del segmento ST & Binaria & 0--28 \\
Elevación de biomarcadores & Binaria & 0--14 \\
\bottomrule
\end{tabular}
\end{table}

\subsubsection{Escala TIMI}

La escala \textit{Thrombolysis In Myocardial Infarction} \citep{antman2000timi} ofrece una alternativa más sencilla, derivada de ensayos clínicos controlados:

\begin{table}[H]
\centering
\caption{Variables incluidas en la escala TIMI para IAM con elevación del ST}
\label{tab:timi_variables}
\begin{tabular}{@{}ll@{}}
\toprule
\textbf{Variable} & \textbf{Puntos} \\
\midrule
Edad $\geq$ 75 años & 3 \\
Edad 65--74 años & 2 \\
Diabetes, hipertensión o angina & 1 \\
PAS $<$ 100 mmHg & 3 \\
FC $>$ 100 lpm & 2 \\
Killip II--IV & 2 \\
Peso $<$ 67 kg & 1 \\
Elevación ST anterior o BRIHH & 1 \\
Tiempo hasta reperfusión $>$ 4 horas & 1 \\
\bottomrule
\end{tabular}
\end{table}

\subsubsection{Otras Escalas Contemporáneas}

\begin{itemize}
    \item \textbf{ACTION Registry-GWTG}: Incorpora datos contemporáneos incluyendo biomarcadores y tratamientos de reperfusión, mejorando la discriminación en poblaciones modernas \citep{morrow2013action}.
    
    \item \textbf{ProACS}: Desarrollada específicamente para poblaciones europeas, con énfasis en la presentación clínica inicial \citep{pinto2015proacs}.
\end{itemize}

\subsection{Limitaciones de los Modelos Tradicionales}

Los modelos basados en regresión logística presentan limitaciones inherentes:

\begin{enumerate}
    \item \textbf{Asunciones de linealidad}: Relaciones lineales entre variables independientes y el logit del resultado.
    
    \item \textbf{Interacciones predefinidas}: Requieren especificación explícita de términos de interacción.
    
    \item \textbf{Distribuciones asimétricas}: Dificultad para manejar variables con distribuciones no gaussianas.
    
    \item \textbf{Colinealidad}: Sensibilidad a correlaciones entre predictores.
    
    \item \textbf{Falta de actualización}: Modelos estáticos que no se adaptan a cambios en la práctica clínica.
\end{enumerate}

\subsection{Aprendizaje Automático en Predicción Cardiovascular}

\subsubsection{Evolución Histórica}

El uso de técnicas de \gls{ml} en cardiología ha experimentado un crecimiento exponencial en la última década, impulsado por:

\begin{itemize}
    \item Disponibilidad de grandes bases de datos clínicos (EHR, registros).
    \item Avances en capacidad computacional.
    \item Desarrollo de algoritmos más sofisticados.
    \item Técnicas de explicabilidad que facilitan la interpretación.
\end{itemize}

\subsubsection{Estudios Relevantes}

\begin{keypoint}
\textbf{Principales hallazgos en la literatura:}
\begin{itemize}
    \item Los modelos de gradient boosting (XGBoost, LightGBM) consistentemente superan a la regresión logística en predicción de mortalidad por IAM.
    \item El AUROC de modelos de ML típicamente oscila entre 0.85--0.93, comparado con 0.75--0.82 para escalas tradicionales.
    \item Variables como edad, presión arterial, función renal y biomarcadores cardíacos emergen consistentemente como los predictores más importantes.
\end{itemize}
\end{keypoint}

\paragraph{Zhu et al. (2024)}
Estudio multicéntrico con más de 20,000 pacientes con IAM. El modelo XGBoost alcanzó un AUROC de 0.93, significativamente superior a GRACE (0.81) y TIMI (0.78). Los predictores más importantes fueron: edad, presión arterial sistólica, fracción de eyección, NT-proBNP y creatinina \citep{zhu2024ml}.

\paragraph{Oliveira et al. (2023)}
Comparación sistemática de algoritmos (regresión logística penalizada, random forest, XGBoost, redes neuronales) en 5,000 pacientes. Los modelos basados en árboles mostraron mejor rendimiento global (AUROC = 0.89) y mejor calibración \citep{oliveira2023ml}.

\paragraph{Wang et al. (2022)}
Integración de biomarcadores inflamatorios (proteína C reactiva, leucocitos) y función renal (filtrado glomerular estimado) en modelos de ML. La combinación de variables clínicas y analíticas mejoró significativamente la predicción \citep{wang2022ami}.

\subsubsection{Comparación de Algoritmos}

\begin{table}[H]
\centering
\caption{Comparación de algoritmos de ML en predicción de mortalidad por IAM (revisión de literatura)}
\label{tab:comparacion_algoritmos_literatura}
\begin{tabular}{@{}lcccl@{}}
\toprule
\textbf{Algoritmo} & \textbf{AUROC} & \textbf{Sensibilidad} & \textbf{Especificidad} & \textbf{Ventajas} \\
\midrule
Regresión Logística & 0.75--0.82 & 0.65--0.75 & 0.70--0.80 & Interpretable, baseline \\
Random Forest & 0.82--0.88 & 0.70--0.82 & 0.75--0.85 & Robusto, no lineal \\
XGBoost & 0.85--0.93 & 0.75--0.88 & 0.80--0.90 & Alto rendimiento \\
Redes Neuronales & 0.80--0.90 & 0.70--0.85 & 0.75--0.88 & Patrones complejos \\
Ensambles & 0.87--0.94 & 0.78--0.90 & 0.82--0.92 & Combina fortalezas \\
\bottomrule
\end{tabular}
\end{table}

\subsection{Predicción de Arritmias Ventriculares}

Las arritmias ventriculares (taquicardia y fibrilación ventricular) constituyen una causa principal de muerte súbita intrahospitalaria en pacientes con IAM. Los enfoques de ML se han centrado en:

\begin{itemize}
    \item \textbf{Análisis de señales ECG}: Redes neuronales convolucionales (CNN) para detección de patrones arritmogénicos.
    \item \textbf{Variables clínicas tabulares}: Random Forest y XGBoost usando variables como intervalo QTc, bloqueo de rama, número de derivaciones afectadas.
    \item \textbf{Modelos híbridos}: Combinación de datos estructurados y señales electrocardiográficas.
\end{itemize}

\subsection{Desafíos y Vacíos en la Literatura}

\begin{enumerate}
    \item \textbf{Validación externa limitada}: Mayoría de estudios validan en la misma cohorte de desarrollo.
    
    \item \textbf{Desbalance de clases}: Mortalidad intrahospitalaria típicamente $<$10\%, generando sesgos hacia la clase mayoritaria.
    
    \item \textbf{Interpretabilidad}: Modelos de caja negra dificultan la adopción clínica; técnicas SHAP y LIME mitigan este problema.
    
    \item \textbf{Calibración}: Modelos con buena discriminación pero mala calibración pueden inducir decisiones erróneas.
    
    \item \textbf{Heterogeneidad poblacional}: Modelos desarrollados en un contexto pueden no generalizar a otras poblaciones.
    
    \item \textbf{Datos faltantes}: Estrategias de imputación pueden introducir sesgos.
\end{enumerate}

\subsection{Oportunidades de Investigación}

\begin{itemize}
    \item Desarrollo de modelos específicos para poblaciones latinoamericanas y caribeñas.
    \item Integración multimodal de datos clínicos, analíticos, electrocardiográficos e imagenológicos.
    \item Actualización dinámica del riesgo durante la hospitalización.
    \item Evaluación de impacto clínico mediante ensayos de implementación.
    \item Modelos federados que preserven la privacidad de datos.
\end{itemize}

\subsection{Marco Teórico del Presente Estudio}

El presente trabajo se fundamenta en:

\begin{enumerate}
    \item \textbf{Teoría de aprendizaje estadístico}: Principios de generalización, sesgo-varianza, regularización.
    
    \item \textbf{Medicina basada en evidencia}: Evaluación rigurosa de rendimiento predictivo.
    
    \item \textbf{Explicabilidad de IA} (\textit{Explainable AI}): Técnicas SHAP para interpretación de modelos complejos.
    
    \item \textbf{Guías TRIPOD+AI}: Estándares para reporte transparente de modelos de predicción.
\end{enumerate}


% Descripción del Dataset
% ============================================================================
% SECCIÓN 03: DESCRIPCIÓN DEL DATASET
% ============================================================================

\section{Descripción del Conjunto de Datos}
\label{sec:dataset}

% NOTA: Esta sección integra la caracterización del archivo dudas_variables/reporte.tex

\subsection{Fuente de Datos}

\begin{table}[H]
\centering
\caption{Información general del conjunto de datos}
\label{tab:info_dataset}
\begin{tabular}{@{}ll@{}}
\toprule
\textbf{Característica} & \textbf{Descripción} \\
\midrule
Nombre del registro & Registro Cubano de Infarto Agudo de Miocardio (RECUIMA) \\
Institución & Sistema Nacional de Salud de Cuba \\
País/Región & Cuba \\
Período de recolección & 2020 -- 2025 \\
Población objetivo & Pacientes con diagnóstico de IAM \\
Criterios de inclusión & Diagnóstico confirmado de IAM según criterios universales \\
Criterios de exclusión & Datos incompletos en variables críticas \\
\bottomrule
\end{tabular}
\end{table}

\subsection{Tamaño Muestral y Estructura}

\begin{table}[H]
\centering
\caption{Resumen del tamaño muestral}
\label{tab:tamano_muestral}
\begin{tabular}{@{}lr@{}}
\toprule
\textbf{Métrica} & \textbf{Valor} \\
\midrule
Número total de pacientes & 3.112 \\
Variables totales (originales) & 185 \\
Conjunto de entrenamiento & 2.489 \\
Conjunto de test & 623 \\
Eventos de interés (mortalidad) & 55 en test (8,8\%) \\
Supervivientes & 568 en test (91,2\%) \\
Tasa de eventos (test) & 8,8\% \\
\bottomrule
\end{tabular}
\end{table}

\subsection{Categorización de Variables}

El conjunto de datos contiene 185 variables organizadas en las siguientes categorías funcionales:

\subsubsection{Variables Demográficas y Administrativas}

\begin{table}[H]
\centering
\caption{Variables demográficas y administrativas}
\label{tab:vars_demograficas}
\begin{tabular}{@{}llp{6cm}@{}}
\toprule
\textbf{Variable} & \textbf{Tipo} & \textbf{Descripción} \\
\midrule
EDAD & Numérica & Edad del paciente en años \\
SEXO & Categórica & Sexo biológico (Masculino/Femenino) \\
FECHA\_INGRESO & Fecha & Fecha de admisión hospitalaria \\
FECHA\_ALTA & Fecha & Fecha de alta o fallecimiento \\
ESTANCIA & Numérica & Días de estancia hospitalaria \\
\bottomrule
\end{tabular}
\end{table}

\subsubsection{Antecedentes Patológicos}

{\scriptsize
\setlength{\tabcolsep}{2pt}
\begin{longtable}{@{}p{2.8cm}p{1.3cm}p{4cm}p{2.2cm}@{}}
\caption{Variables de antecedentes patológicos personales}
\label{tab:vars_antecedentes} \\
\toprule
\textbf{Variable} & \textbf{Tipo} & \textbf{Descripción} & \textbf{Valores} \\
\midrule
\endfirsthead
\multicolumn{4}{c}{\tablename\ \thetable{} -- Continuación} \\
\toprule
\textbf{Variable} & \textbf{Tipo} & \textbf{Descripción} & \textbf{Valores} \\
\midrule
\endhead
\bottomrule
\endfoot
APP\_HTA & Binaria & Hipertensión arterial & Sí/No \\
APP\_DM & Binaria & Diabetes mellitus & Sí/No \\
APP\_DISLIPIDEMIA & Binaria & Dislipidemia & Sí/No \\
APP\_TABAQUISMO & Categ. & Estado de tabaquismo & Activo/Ex/Nunca \\
APP\_OBESIDAD & Binaria & Obesidad (IMC $\geq$ 30) & Sí/No \\
APP\_IAM\_PREVIO & Binaria & IAM previo & Sí/No \\
APP\_ICP\_PREVIO & Binaria & Angioplastia previa & Sí/No \\
APP\_CABG\_PREVIO & Binaria & Cirugía coronaria previa & Sí/No \\
APP\_ERC & Binaria & Enfermedad renal crónica & Sí/No \\
APP\_FA & Binaria & Fibrilación auricular & Sí/No \\
APP\_ICTUS & Binaria & Ictus/ACV previo & Sí/No \\
APP\_EAP & Binaria & Enf. arterial periférica & Sí/No \\
\end{longtable}
}

\subsubsection{Presentación Clínica}

\begin{table}[H]
\centering
\caption{Variables de presentación clínica al ingreso}
\label{tab:vars_clinicas}
{\scriptsize\setlength{\tabcolsep}{2pt}
\begin{tabular}{@{}p{2.8cm}p{1.6cm}p{3.8cm}p{1.5cm}@{}}
\toprule
\textbf{Variable} & \textbf{Tipo} & \textbf{Descripción} & \textbf{Unidad} \\
\midrule
PAS & Numérica & Presión arterial sistólica & mmHg \\
PAD & Numérica & Presión arterial diastólica & mmHg \\
FC & Numérica & Frecuencia cardíaca & lpm \\
FR & Numérica & Frecuencia respiratoria & rpm \\
SAT\_O2 & Numérica & Saturación de oxígeno & \% \\
KILLIP & Categórica & Clasificación Killip-Kimball & I--IV \\
DOLOR\_TIPICO & Binaria & Dolor torácico típico & Sí/No \\
TIEMPO\_SINTOMAS & Numérica & Tiempo desde síntomas & horas \\
\bottomrule
\end{tabular}
}
\end{table}

\subsubsection{Variables Electrocardiográficas}

\begin{table}[H]
\centering
\caption{Variables electrocardiográficas}
\label{tab:vars_ecg}
{\scriptsize\setlength{\tabcolsep}{2pt}
\begin{tabular}{@{}p{2.5cm}p{1.5cm}p{5cm}@{}}
\toprule
\textbf{Variable} & \textbf{Tipo} & \textbf{Descripción} \\
\midrule
ECG\_RITMO & Categórica & Ritmo cardíaco (sinusal, FA, etc.) \\
ECG\_ELEV\_ST & Binaria & Presencia de elevación del ST \\
ECG\_N\_DERIV\_ST & Numérica & Núm. derivaciones con elevación ST \\
ECG\_LOCAL & Categórica & Localización IAM (anterior, inferior) \\
ECG\_BRIHH & Binaria & Bloqueo de rama izquierda \\
ECG\_QTc & Numérica & Intervalo QT corregido (ms) \\
\bottomrule
\end{tabular}
}
\end{table}

\subsubsection{Biomarcadores de Laboratorio}

{\scriptsize
\setlength{\tabcolsep}{2pt}
\begin{longtable}{@{}p{2.2cm}p{0.9cm}p{3.5cm}p{2cm}p{1.2cm}@{}}
\caption{Biomarcadores y pruebas de laboratorio}
\label{tab:vars_laboratorio} \\
\toprule
\textbf{Variable} & \textbf{Tipo} & \textbf{Descripción} & \textbf{Unidad} & \textbf{Ref.} \\
\midrule
\endfirsthead
\multicolumn{5}{c}{\tablename\ \thetable{} -- Continuación} \\
\toprule
\textbf{Variable} & \textbf{Tipo} & \textbf{Descripción} & \textbf{Unidad} & \textbf{Ref.} \\
\midrule
\endhead
\bottomrule
\endfoot
TROPONINA\_I & Num. & Troponina I cardíaca & ng/mL & $<$0.04 \\
TROPONINA\_T & Num. & Troponina T cardíaca & ng/mL & $<$0.01 \\
CK\_MB & Num. & Creatina quinasa MB & U/L & $<$25 \\
BNP & Num. & Péptido natriurético B & pg/mL & $<$100 \\
NT\_PROBNP & Num. & NT-proBNP & pg/mL & $<$300 \\
CREATININA & Num. & Creatinina sérica & mg/dL & 0.7--1.3 \\
TFG\_E & Num. & TFG estimada & mL/min/1.73m² & $>$60 \\
GLUCOSA & Num. & Glucemia & mg/dL & 70--100 \\
HBA1C & Num. & Hemoglobina glicosilada & \% & $<$5.7 \\
HEMOGLOBINA & Num. & Hemoglobina & g/dL & 12--16 \\
LEUCOCITOS & Num. & Recuento leucocitos & $\times 10^9$/L & 4--11 \\
PLAQUETAS & Num. & Recuento plaquetas & $\times 10^9$/L & 150--400 \\
COLESTEROL & Num. & Colesterol total & mg/dL & $<$200 \\
LDL & Num. & Colesterol LDL & mg/dL & $<$100 \\
HDL & Num. & Colesterol HDL & mg/dL & $>$40 \\
TRIGLICERIDOS & Num. & Triglicéridos & mg/dL & $<$150 \\
PCR & Num. & Proteína C reactiva & mg/L & $<$3 \\
\end{longtable}
}

\subsubsection{Tratamientos y Procedimientos}

\begin{table}[H]
\centering
\caption{Variables de tratamiento y procedimientos}
\label{tab:vars_tratamiento}
{\scriptsize\setlength{\tabcolsep}{2pt}
\begin{tabular}{@{}p{3.2cm}p{1.3cm}p{4.5cm}@{}}
\toprule
\textbf{Variable} & \textbf{Tipo} & \textbf{Descripción} \\
\midrule
TTO\_FIBRINOLISIS & Binaria & Fibrinólisis administrada \\
TTO\_ICP\_PRIMARIA & Binaria & ICP primaria \\
TIEMPO\_PUERTA\_BALON & Numérica & Tiempo puerta-balón (min) \\
TTO\_ASPIRINA & Binaria & Ácido acetilsalicílico \\
TTO\_CLOPIDOGREL & Binaria & Clopidogrel \\
TTO\_TICAGRELOR & Binaria & Ticagrelor \\
TTO\_HEPARINA & Binaria & Heparina \\
TTO\_BETABLOQ & Binaria & Betabloqueantes \\
TTO\_IECA\_ARA2 & Binaria & IECA o ARA2 \\
TTO\_ESTATINA & Binaria & Estatinas \\
TTO\_VENTILACION & Binaria & Ventilación mecánica \\
TTO\_VASOACTIVOS & Binaria & Fármacos vasoactivos \\
\bottomrule
\end{tabular}
}
\end{table}

\subsubsection{Variables de Imagen}

\begin{table}[H]
\centering
\caption{Variables de estudios de imagen}
\label{tab:vars_imagen}
{\scriptsize\setlength{\tabcolsep}{2pt}
\begin{tabular}{@{}p{2.5cm}p{1.3cm}p{4cm}p{1.3cm}@{}}
\toprule
\textbf{Variable} & \textbf{Tipo} & \textbf{Descripción} & \textbf{Unidad} \\
\midrule
ECO\_FEVI & Numérica & Fracción eyección VI & \% \\
ECO\_ALT\_SEGM & Binaria & Alteraciones contractilidad & Sí/No \\
ECO\_VALVULOP & Binaria & Valvulopatía asociada & Sí/No \\
CORO\_N\_VASOS & Numérica & Núm. vasos enfermos & 0--3 \\
CORO\_TCI & Binaria & Enfermedad TCI & Sí/No \\
\bottomrule
\end{tabular}
}
\end{table}

\subsubsection{Complicaciones Intrahospitalarias}

\begin{table}[H]
\centering
\caption{Variables de complicaciones durante la hospitalización}
\label{tab:vars_complicaciones}
{\scriptsize\setlength{\tabcolsep}{2pt}
\begin{tabular}{@{}p{2.8cm}p{1.2cm}p{5cm}@{}}
\toprule
\textbf{Variable} & \textbf{Tipo} & \textbf{Descripción} \\
\midrule
COMP\_ARRITMIA\_V & Binaria & Arritmia ventricular maligna (TV/FV) \\
COMP\_BAV & Binaria & Bloqueo auriculoventricular \\
COMP\_ICC & Binaria & Insuficiencia cardíaca congestiva \\
COMP\_SHOCK & Binaria & Shock cardiogénico \\
COMP\_REINFARTO & Binaria & Reinfarto intrahospitalario \\
COMP\_SANGRADO & Binaria & Sangrado mayor (criterios BARC) \\
COMP\_IRA & Binaria & Injuria renal aguda \\
COMP\_ICTUS & Binaria & Ictus intrahospitalario \\
\bottomrule
\end{tabular}
}
\end{table}

\subsubsection{Variables Outcome}

\begin{table}[H]
\centering
\caption{Variables de desenlace (outcome)}
\label{tab:vars_outcome}
{\scriptsize\setlength{\tabcolsep}{2pt}
\begin{tabular}{@{}p{2.8cm}p{1.2cm}p{5cm}@{}}
\toprule
\textbf{Variable} & \textbf{Tipo} & \textbf{Descripción} \\
\midrule
MORTALIDAD\_HOSP & Binaria & \textbf{Variable objetivo}: Mortalidad intrahospitalaria \\
CAUSA\_MUERTE & Categ. & Causa de muerte (cardiogénica, arrítmica) \\
DESTINO\_ALTA & Categ. & Destino al alta (domicilio, otro hospital) \\
\bottomrule
\end{tabular}
}
\end{table}

\subsection{Consideraciones sobre Calidad de Datos}

\subsubsection{Datos Faltantes}

El análisis de calidad del conjunto de datos original (n = 3.112 pacientes, 195 variables) reveló un patrón heterogéneo de datos faltantes. Se identificaron tres categorías principales según el porcentaje de missingness:

\paragraph{Variables con missingness extremo ($>$90\%):} Estas variables fueron excluidas del análisis por su escasa representatividad.

\begin{table}[H]
\centering
\caption{Variables con $>$90\% de datos faltantes (excluidas)}
\label{tab:missing_extremo}
{\footnotesize\setlength{\tabcolsep}{2pt}
\begin{tabular}{@{}p{3.5cm}p{1.6cm}p{4cm}@{}}
\toprule
\textbf{Variable} & \textbf{\% Falt.} & \textbf{Motivo} \\
\midrule
coronariografias\_centro & 100,0\% & No registrado \\
coronariografias\_medico & 100,0\% & No registrado \\
fecha\_defuncion & 99,6\% & Solo en fallecidos (MNAR) \\
protección\_embólica & 99,4\% & Procedimiento infrecuente \\
rehabilitación & 99,4\% & Dato post-alta \\
volumen\_contraste & 99,4\% & Solo en coronariografías \\
estenosis/arteria/abordaje & 92,8\% & Solo en intervencionismo \\
annos\_sin\_fumar & 92,2\% & Solo exfumadores (MAR) \\
\bottomrule
\end{tabular}
}
\end{table}

\paragraph{Variables con missingness moderado-alto (20--90\%):} Requirieron evaluación individual para decidir estrategia de imputación o exclusión.

\begin{table}[H]
\centering
\caption{Variables con 20--90\% de datos faltantes}
\label{tab:missing_moderado}
{\footnotesize\setlength{\tabcolsep}{2pt}
\begin{tabular}{@{}p{3cm}p{1.5cm}p{1.5cm}p{3.2cm}@{}}
\toprule
\textbf{Variable} & \textbf{\% Falt.} & \textbf{Patrón} & \textbf{Estrategia} \\
\midrule
tiempo\_llegada & 84,9\% & MNAR & Excluida \\
insulina & 83,3\% & MAR & Excluida \\
fs (func. sistólica) & 81,1\% & MAR & Excluida \\
ingresos\_anteriores & 78,1\% & MAR & Excluida \\
tiempo\_isquemia & 64,1\% & MNAR & Excluida \\
tapse & 62,9\% & MAR & Excluida \\
ckmb & 54,8\% & MAR & Imput. múltiple \\
ck & 53,6\% & MAR & Imput. múltiple \\
tipo\_tabaquismo & 47,9\% & MAR & Imput. por moda \\
tiempo\_puerta\_aguja & 46,6\% & MNAR & Imput. condicional \\
betabloqueadores & 22,8\% & MAR & Imput. múltiple \\
leucocitos & 21,3\% & MCAR & Imput. mediana \\
infradesnivel\_ST & 20,0\% & MCAR & Imput. moda \\
\bottomrule
\end{tabular}
}
\end{table}

\paragraph{Variables con missingness bajo ($<$20\%):} Se aplicó imputación estándar según el tipo de variable.

\begin{table}[H]
\centering
\caption{Variables clínicas relevantes con $<$20\% de datos faltantes}
\label{tab:missing_bajo}
{\footnotesize\setlength{\tabcolsep}{2pt}
\begin{tabular}{@{}p{3cm}p{1.5cm}p{1.5cm}p{3cm}@{}}
\toprule
\textbf{Variable} & \textbf{\% Falt.} & \textbf{Patrón} & \textbf{Estrategia} \\
\midrule
triglicéridos & 19,2\% & MCAR & Imput. mediana \\
supradesnivel\_ST & 14,4\% & MCAR & Imput. moda \\
fracción\_eyección & 13,9\% & MAR & Imput. múltiple \\
hemoglobina & 12,5\% & MCAR & Imput. mediana \\
estadia\_ucie & 12,3\% & MCAR & Imput. mediana \\
colesterol & 12,1\% & MCAR & Imput. mediana \\
glicemia & 4,6\% & MCAR & Imput. mediana \\
creatinina & 4,5\% & MCAR & Imput. mediana \\
escala\_grace & 4,5\% & MAR & Recálculo \\
filtrado\_glomerular & 4,5\% & MAR & Derivada \\
\bottomrule
\end{tabular}
}
\end{table}

El análisis del mecanismo de missingness mediante pruebas de Little y correlaciones entre patrones de ausencia sugirió predominio de datos \textit{Missing at Random} (MAR) para la mayoría de variables clínicas, con excepciones notables como \texttt{fecha\_defuncion} y \texttt{tiempo\_isquemia} que presentaron patrón \textit{Missing Not at Random} (MNAR).

\subsubsection{Valores Atípicos Identificados}

Se aplicó detección de outliers mediante el método del rango intercuartílico (IQR) extendido (1.5 $\times$ IQR) para variables numéricas continuas. Los valores atípicos identificados fueron evaluados clínicamente antes de su tratamiento:

\begin{itemize}
    \item \textbf{Creatinina}: Valores $>$10 mg/dL (n=23) verificados como casos de insuficiencia renal severa -- conservados.
    \item \textbf{Glucemia}: Valores $>$500 mg/dL (n=8) correspondientes a crisis hiperglucémicas -- conservados.
    \item \textbf{Troponinas}: Valores extremadamente elevados consistentes con infarto extenso -- conservados.
    \item \textbf{Frecuencia cardíaca}: Valores $<$40 o $>$180 lpm (n=45) revisados individualmente.
    \item \textbf{Presión arterial}: Valores sistólicos $>$220 mmHg (n=12) confirmados como crisis hipertensivas.
\end{itemize}

La estrategia general fue conservar outliers biológicamente plausibles y clínicamente relevantes, aplicando winsorización al percentil 1--99 únicamente en casos de errores evidentes de registro.

\subsection{Resumen Descriptivo Preliminar}

El resumen descriptivo completo de las variables se presenta en la sección de Análisis Exploratorio de Datos (Sección~\ref{sec:eda}), donde se incluyen estadísticos descriptivos estratificados por outcome y análisis bivariados con pruebas de significación estadística.


% Metodología
% ============================================================================
% SECCIÓN 04: METODOLOGÍA
% ============================================================================

\section{Metodología}
\label{sec:metodologia}

Este estudio sigue las recomendaciones de las guías TRIPOD+AI (Transparent Reporting of a multivariable prediction model for Individual Prognosis Or Diagnosis with Artificial Intelligence) para el desarrollo y reporte de modelos de predicción clínica basados en aprendizaje automático \citep{steyerberg2019clinical}.

\subsection{Diseño del Estudio}

\begin{table}[H]
\centering
\caption{Características del diseño del estudio}
\label{tab:diseno_estudio}
\begin{tabular}{@{}ll@{}}
\toprule
\textbf{Aspecto} & \textbf{Descripción} \\
\midrule
Tipo de estudio & Estudio de desarrollo y validación de modelo predictivo \\
Enfoque analítico & Supervisado (clasificación binaria) \\
Variable objetivo & Mortalidad intrahospitalaria \\
Horizonte de predicción & Durante la hospitalización índice \\
Marco temporal & Retrospectivo \\
Validación & Interna (Bootstrap 1000 iter. + Jackknife) \\
\bottomrule
\end{tabular}
\end{table}

\subsection{Preprocesamiento de Datos}

\subsubsection{Tratamiento de Datos Faltantes}

\begin{table}[H]
\centering
\caption{Estrategias de imputación según tipo de variable y patrón de missingness}
\label{tab:estrategias_imputacion}
\begin{tabular}{@{}lll@{}}
\toprule
\textbf{Tipo de variable} & \textbf{Estrategia} & \textbf{Justificación} \\
\midrule
Numéricas (MAR) & Mediana & Robustez ante outliers \\
Numéricas (MNAR) & Indicador + imputación & Preservar información de missingness \\
Categóricas & Moda o categoría ``Desconocido'' & Preservar distribución original \\
\bottomrule
\end{tabular}
\end{table}

Variables excluidas por exceso de datos faltantes ($>$50\%) y sin relevancia crítica para la predicción:

{\footnotesize
\begin{itemize}
    \item \texttt{coronariografias\_centro/medico} (100\%) -- Datos administrativos
    \item \texttt{proteccion\_embolica, rehabilitacion, prescripcion\_optima} (99,4\%) -- Variables post-alta
    \item \texttt{volumen\_contraste, funcion\_renal} (99,4\%) -- Solo en intervencionismo
    \item \texttt{estenosis, arteria, abordaje, localizacion} (92,8\%) -- Datos de coronariografía
    \item \texttt{annos\_sin\_fumar} (92,2\%) -- Solo aplica a exfumadores
    \item \texttt{tiempo\_llegada, tiempo\_respuesta} (84,9\%) -- Tiempos prehospitalarios
    \item \texttt{insulina} (83,3\%) -- Solo en diabéticos con insulinoterapia
    \item \texttt{fs} (función sistólica) (81,1\%) -- Medición ecocardiográfica avanzada
    \item \texttt{ingresos\_anteriores} (78,1\%) -- Dato histórico incompleto
    \item \texttt{ud, pat} (77\%) -- Variables administrativas
    \item \texttt{ee} (espesor endócrano) (65,3\%) -- Parámetro ecocardiográfico
    \item \texttt{tiempo\_isquemia} (64,1\%) -- Difícil determinación retrospectiva
    \item \texttt{tapse, ea, pp, dsvi, tiv, ddvi} (54--63\%) -- Parámetros ecocardiográficos avanzados
    \item \texttt{ckmb, ck} (53--55\%) -- Biomarcadores en desuso (reemplazados por troponinas)
\end{itemize}
}

\subsubsection{Tratamiento de Valores Atípicos}

\begin{itemize}
    \item \textbf{Detección}: Método IQR para variables continuas y análisis de distribución.
    \item \textbf{Manejo}: Winsorización al percentil 1-99 para variables numéricas extremas.
    \item \textbf{Variables afectadas}: Troponinas, tiempos de atención, CK-MB, glucemia.
\end{itemize}

\subsubsection{Codificación de Variables Categóricas}

\begin{table}[H]
\centering
\small
\caption{Estrategias de codificación de variables categóricas}
\label{tab:codificacion_categoricas}
\begin{tabular}{@{}p{2cm}p{2.2cm}p{8cm}@{}}
\toprule
\textbf{Tipo} & \textbf{Codificación} & \textbf{Variables} \\
\midrule
Binarias & 0/1 & Sexo, DM, HTA, tabaquismo, IAM previo, ICP previa, CABG previo, ERC, fibrilación auricular \\
Nominales baja card. & One-Hot & Localización del IAM, tipo de SCA, ritmo cardíaco \\
Nominales alta card. & Target Encoding & Provincia, municipio, área de salud \\
Ordinales & Ordinal Encoding & Killip I$\rightarrow$IV, categoría GRACE \\
\bottomrule
\end{tabular}
\end{table}

\subsubsection{Transformación y Normalización}

\begin{itemize}
    \item \textbf{Variables sesgadas}: Transformación log para biomarcadores (troponinas, CK-MB, NT-proBNP).
    \item \textbf{Normalización}: RobustScaler para variables con outliers residuales.
    \item \textbf{Variables aplicadas}: Creatinina, glucemia, biomarcadores cardíacos, tiempos de atención.
\end{itemize}

\subsection{Ingeniería de Características}

\subsubsection{Creación de Variables Derivadas}

\begin{table}[H]
\centering
\caption{Variables derivadas creadas}
\label{tab:variables_derivadas}
\begin{tabular}{@{}p{3cm}p{4.5cm}p{5cm}@{}}
\toprule
\textbf{Variable nueva} & \textbf{Fórmula/Definición} & \textbf{Justificación clínica} \\
\midrule
IMC & peso/(talla/100)² & Indicador de obesidad, factor de riesgo cardiovascular \\
TFG estimada & CKD-EPI (creatinina, edad, sexo) & Función renal, predictor de mortalidad en IAM \\
Índice de shock & FC/PAS & Indicador de inestabilidad hemodinámica \\
SCORE\_GRACE\_CALC & Cálculo según fórmula original & Comparación con escala de referencia internacional \\
\bottomrule
\end{tabular}
\end{table}

\subsubsection{Selección de Características}

Métodos aplicados:

\begin{enumerate}
    \item \textbf{Filtrado}: Correlación con variable objetivo, test $\chi^2$/ANOVA.
    
    \item \textbf{Embedded}: Importancia de características en \gls{rf} y XGBoost.
    
    \item \textbf{Wrapper}: RFE (Recursive Feature Elimination) con XGBoost.
\end{enumerate}

\begin{keypoint}
\textbf{Resultado de selección de variables:}

Se desarrollaron dos modelos con diferente número de variables:
\begin{itemize}
    \item \textbf{Modelo reducido (10 variables)}: Para comparación directa con la escala GRACE. Variables: filtrado glomerular, fracción de eyección, edad, glicemia, presión arterial diastólica, creatinina, presión arterial sistólica, diabetes mellitus, frecuencia cardíaca, betabloqueadores.
    \item \textbf{Modelo extendido (57 variables)}: Propuesta principal de investigación, incluyendo variables demográficas, antecedentes, biomarcadores, variables electrocardiográficas, tratamientos y complicaciones.
\end{itemize}
\end{keypoint}

\subsection{División de Datos}

\begin{table}[H]
\centering
\caption{Esquema de partición de datos}
\label{tab:particion_datos}
\begin{tabular}{@{}lccc@{}}
\toprule
\textbf{Conjunto} & \textbf{Porcentaje} & \textbf{N pacientes} & \textbf{Uso} \\
\midrule
Entrenamiento & 80\% & 2.489 & Ajuste de modelos \\
Test & 20\% & 623 & Evaluación final \\
\bottomrule
\end{tabular}
\end{table}

\begin{itemize}
    \item \textbf{Estratificación}: Por variable objetivo (mortalidad) para mantener proporción de eventos.
    \item \textbf{Semilla aleatoria}: 42 para reproducibilidad.
\end{itemize}

\subsection{Manejo del Desbalance de Clases}

Dado que la mortalidad intrahospitalaria por IAM típicamente oscila entre 5--10\%, se implementaron las siguientes estrategias:

\begin{table}[H]
\centering
\caption{Técnicas para manejo de desbalance de clases evaluadas}
\label{tab:manejo_desbalance}
\begin{tabular}{@{}lp{6cm}l@{}}
\toprule
\textbf{Técnica} & \textbf{Descripción} & \textbf{Aplicada} \\
\midrule
Class weights & Ponderación inversa a frecuencia de clase & Sí \\
SMOTE & Synthetic Minority Over-sampling & Evaluado \\
ADASYN & Adaptive Synthetic Sampling & Evaluado \\
Random Undersampling & Submuestreo de clase mayoritaria & No \\
Threshold adjustment & Ajuste del umbral de decisión & Sí (Youden) \\
\bottomrule
\end{tabular}
\end{table}

\subsection{Algoritmos de Aprendizaje Automático}

\subsubsection{Modelos Evaluados}

\begin{enumerate}
    \item \textbf{Regresión Logística Penalizada} (\gls{lr}): Modelo baseline lineal con regularización L2.
    
    \item \textbf{K-Nearest Neighbors} (KNN): Clasificación basada en vecinos más cercanos.
    
    \item \textbf{Árbol de Decisión}: Modelo interpretable basado en particiones recursivas.
    
    \item \textbf{Random Forest} (\gls{rf}): Ensamble de árboles de decisión con bagging y selección aleatoria de características.
    
    \item \textbf{XGBoost}: Gradient boosting optimizado con regularización y manejo de missings.
    
    \item \textbf{XGBoost Balanced}: XGBoost con ajuste de scale\_pos\_weight para desbalance.
    
    \item \textbf{LightGBM}: Gradient boosting basado en histogramas para eficiencia computacional.
\end{enumerate}

\subsubsection{Optimización de Hiperparámetros}

\begin{table}[H]
\centering
\caption{Estrategia de optimización de hiperparámetros}
\label{tab:optimizacion_hp}
\begin{tabular}{@{}ll@{}}
\toprule
\textbf{Aspecto} & \textbf{Configuración} \\
\midrule
Método de búsqueda & Validación cruzada estratificada \\
Validación cruzada & 5-fold estratificada \\
Métrica de optimización & AUROC \\
Validación robustez & Bootstrap (1000 iter.) + Jackknife \\
\bottomrule
\end{tabular}
\end{table}

El espacio de búsqueda de hiperparámetros se detalla en el Apéndice \ref{app:hiperparametros}.

\subsection{Métricas de Evaluación}

\subsubsection{Discriminación}

\begin{table}[H]
\centering
\caption{Métricas de discriminación}
\label{tab:metricas_discriminacion}
\begin{tabular}{@{}lp{8cm}@{}}
\toprule
\textbf{Métrica} & \textbf{Descripción} \\
\midrule
\gls{auroc} & Área bajo la curva ROC. Probabilidad de que el modelo asigne mayor riesgo a un caso positivo que a uno negativo. \\
AUPRC & Área bajo la curva Precision-Recall. Más informativa con clases desbalanceadas. \\
Sensibilidad & Tasa de verdaderos positivos (recall). \\
Especificidad & Tasa de verdaderos negativos. \\
Precisión & Valor predictivo positivo. \\
F1-Score & Media armónica de precisión y recall. \\
\bottomrule
\end{tabular}
\end{table}

\subsubsection{Calibración}

\begin{itemize}
    \item \textbf{Curva de calibración}: Probabilidades predichas vs. frecuencias observadas por deciles.
    \item \textbf{Brier Score}: $\text{BS} = \frac{1}{N}\sum_{i=1}^{N}(p_i - y_i)^2$, donde menor es mejor.
    \item \textbf{Test de Hosmer-Lemeshow}: Bondad de ajuste de probabilidades calibradas.
    \item \textbf{Calibration slope e intercept}: Regresión logística de outcomes sobre probabilidades predichas.
\end{itemize}

\subsubsection{Utilidad Clínica}

\begin{itemize}
    \item \textbf{Decision Curve Analysis}: Beneficio neto a diferentes umbrales de probabilidad.
    \item \textbf{Net Reclassification Index (NRI)}: Mejora en reclasificación respecto a modelo baseline.
    \item \textbf{Integrated Discrimination Index (IDI)}: Mejora en separación de probabilidades.
\end{itemize}

\subsection{Análisis de Explicabilidad}

Para garantizar la interpretabilidad clínica del modelo, se aplicaron las siguientes técnicas:

\subsubsection{Explicabilidad Global}

\begin{itemize}
    \item \textbf{Feature Importance}: Importancia permutacional y basada en ganancia (para modelos de árboles).
    \item \textbf{\gls{shap} Summary Plot}: Distribución del impacto de cada variable en las predicciones.
    \item \textbf{Partial Dependence Plots}: Efecto marginal de variables individuales.
\end{itemize}

\subsubsection{Explicabilidad Local}

\begin{itemize}
    \item \textbf{SHAP Force Plots}: Explicación de predicciones individuales.
    \item \textbf{SHAP Waterfall Plots}: Contribución acumulativa de variables por caso.
\end{itemize}

\subsection{Validación del Modelo}

\begin{table}[H]
\centering
\caption{Estrategia de validación}
\label{tab:estrategia_validacion}
{\footnotesize
\begin{tabular}{@{}p{3.5cm}p{7cm}@{}}
\toprule
\textbf{Tipo} & \textbf{Descripción} \\
\midrule
Validación rápida & K-fold 3$\times$3 estratificada para selección inicial de modelos \\
Validación completa & K-fold 100$\times$100 estratificada para evaluación final robusta \\
Análisis sensibilidad & Estabilidad ante diferentes imputaciones y particiones \\
\bottomrule
\end{tabular}
}
\end{table}

La validación interna se realizó exclusivamente mediante validación cruzada estratificada repetida, sin validación temporal ni externa debido a las características del estudio.

\subsection{Herramientas Tecnológicas}

\begin{table}[H]
\centering
\caption{Stack tecnológico utilizado}
\label{tab:stack_tecnologico}
\begin{tabular}{@{}ll@{}}
\toprule
\textbf{Componente} & \textbf{Herramienta/Versión} \\
\midrule
Lenguaje de programación & Python 3.x \\
Manipulación de datos & pandas, numpy \\
Visualización & matplotlib, seaborn, plotly \\
Aprendizaje automático & scikit-learn, XGBoost, LightGBM \\
Redes neuronales & \placeholder{TensorFlow/PyTorch/AutoKeras} \\
Explicabilidad & SHAP, eli5 \\
Tracking de experimentos & MLflow \\
Interfaz de usuario & Streamlit \\
Control de versiones & Git, GitHub \\
Documentación & MkDocs \\
\bottomrule
\end{tabular}
\end{table}

\subsection{Reproducibilidad}

Para garantizar la reproducibilidad del estudio:

\begin{itemize}
    \item Todo el código fuente está disponible en \url{https://github.com/Pol4720/mortality-ami-predictor}.
    \item Se fijaron semillas aleatorias en todos los procesos estocásticos.
    \item Las dependencias están especificadas en archivos \texttt{requirements.txt} y \texttt{environment.yml}.
    \item Los modelos entrenados se guardaron en formato serializado (\texttt{.joblib}).
    \item Se utilizó MLflow para el tracking de experimentos y versiones de modelos.
\end{itemize}


% Análisis Exploratorio de Datos
% ============================================================================
% SECCIÓN 05: ANÁLISIS EXPLORATORIO DE DATOS (EDA)
% ============================================================================

\section{Análisis Exploratorio de Datos}
\label{sec:eda}

El \gls{eda} constituye una fase fundamental para comprender la estructura, calidad y distribución de los datos antes del modelado predictivo.

\subsection{Distribución de la Variable Objetivo}

\begin{figure}[H]
\centering
\includegraphics[width=0.75\textwidth]{distribucion_moratality_inhospital.png}
\caption{Distribución de la variable objetivo (mortalidad intrahospitalaria). Se observa el marcado desbalance entre las clases: la mayoría de pacientes sobrevivieron (clase 0), mientras que una minoría falleció (clase 1), característico de estudios de mortalidad cardiovascular.}
\label{fig:distribucion_outcome}
\end{figure}

\begin{keypoint}
\textbf{Hallazgo clave:} La tasa de mortalidad intrahospitalaria fue del 8,80\% (n=274 de 3.112 pacientes), confirmando el desbalance de clases característico de este tipo de estudios.
\end{keypoint}

\subsection{Análisis de Variables Demográficas}

\subsubsection{Distribución de Edad}

\begin{figure}[H]
\centering
\includegraphics[width=0.9\textwidth]{distribucion_edad.png}
\caption{Distribución de edad en la población de estudio. El histograma muestra una distribución aproximadamente normal con ligera asimetría hacia edades mayores, con mayor concentración de pacientes entre 55 y 75 años.}
\label{fig:distribucion_edad}
\end{figure}

\begin{figure}[H]
\centering
\begin{subfigure}[b]{0.48\textwidth}
    \centering
    \includegraphics[width=\textwidth]{plot_bigote_edad.png}
    \caption{Boxplot de edad}
\end{subfigure}
\hfill
\begin{subfigure}[b]{0.48\textwidth}
    \centering
    \includegraphics[width=\textwidth]{plot_violin_edad.png}
    \caption{Violin plot de edad}
\end{subfigure}
\caption{Análisis comparativo de edad. Los diagramas de caja y violín revelan la presencia de valores atípicos en edades extremas y confirman la distribución unimodal de la variable.}
\label{fig:edad_boxplot_violin}
\end{figure}

\begin{table}[H]
\centering
\caption{Estadísticos descriptivos de edad}
\label{tab:estadisticos_edad}
\begin{tabular}{@{}lcccc@{}}
\toprule
\textbf{Grupo} & \textbf{N} & \textbf{Media ± DE} & \textbf{Mediana [IQR]} & \textbf{Rango} \\
\midrule
Total & 3.112 & 65,2 ± 12,4 & 65 [56--75] & 25--98 \\
Supervivientes & 2.838 & 64,5 ± 12,1 & 64 [55--74] & 25--95 \\
Fallecidos & 274 & 71,8 ± 11,9 & 73 [63--81] & 35--98 \\
\midrule
\multicolumn{5}{l}{\textit{p-valor (test Mann-Whitney): p $<$ 0,001}} \\
\bottomrule
\end{tabular}
\end{table}

\subsubsection{Distribución por Sexo}

\begin{figure}[H]
\centering
\begin{subfigure}[b]{0.48\textwidth}
    \centering
    \includegraphics[width=\textwidth]{distribucion_de_sexo.png}
    \caption{Distribución por sexo}
\end{subfigure}
\hfill
\begin{subfigure}[b]{0.48\textwidth}
    \centering
    \includegraphics[width=\textwidth]{frecuencias_de_sexo.png}
    \caption{Frecuencias por sexo}
\end{subfigure}
\caption{Distribución por sexo en la cohorte de estudio. Se observa predominio del sexo masculino, consistente con la epidemiología del IAM donde los hombres presentan mayor incidencia, especialmente en edades más tempranas.}
\label{fig:distribucion_sexo}
\end{figure}

\subsection{Análisis de Antecedentes Patológicos}

\begin{figure}[H]
\centering
\begin{subfigure}[b]{0.48\textwidth}
    \centering
    \includegraphics[width=\textwidth]{distribucion_de_hipertension_arterial.png}
    \caption{Hipertensión arterial}
\end{subfigure}
\hfill
\begin{subfigure}[b]{0.48\textwidth}
    \centering
    \includegraphics[width=\textwidth]{distribucion_de_diabetes_mellitus.png}
    \caption{Diabetes mellitus}
\end{subfigure}
\caption{Distribución de los principales factores de riesgo cardiovascular. La hipertensión arterial presenta alta prevalencia en la cohorte, mientras que la diabetes mellitus afecta aproximadamente a un tercio de los pacientes.}
\label{fig:prevalencia_app}
\end{figure}

\begin{figure}[H]
\centering
\begin{subfigure}[b]{0.48\textwidth}
    \centering
    \includegraphics[width=\textwidth]{distribucion_De_tabaquismo.png}
    \caption{Tabaquismo}
\end{subfigure}
\hfill
\begin{subfigure}[b]{0.48\textwidth}
    \centering
    \includegraphics[width=\textwidth]{distribucion_insuficiencia_renal_cronica.png}
    \caption{Insuficiencia renal crónica}
\end{subfigure}
\caption{Distribución de antecedentes patológicos adicionales. El tabaquismo constituye un factor de riesgo modificable frecuente, mientras que la insuficiencia renal crónica, aunque menos prevalente, se asocia con peor pronóstico.}
\label{fig:antecedentes_adicionales}
\end{figure}

\begin{figure}[H]
\centering
\begin{subfigure}[b]{0.48\textwidth}
    \centering
    \includegraphics[width=\textwidth]{violin_plot_diabetes_mellitus.png}
    \caption{Diabetes mellitus vs mortalidad}
\end{subfigure}
\hfill
\begin{subfigure}[b]{0.48\textwidth}
    \centering
    \includegraphics[width=\textwidth]{violin_plot_tabaquismo.png}
    \caption{Tabaquismo vs mortalidad}
\end{subfigure}
\caption{Análisis de antecedentes patológicos según desenlace. Los violin plots permiten visualizar la distribución de estos factores de riesgo estratificados por mortalidad intrahospitalaria.}
\label{fig:antecedentes_violin}
\end{figure}

\begin{table}[H]
\centering
\caption{Comparación de antecedentes patológicos entre grupos}
\label{tab:comparacion_app}
\begin{tabular}{@{}lccc@{}}
\toprule
\textbf{Antecedente} & \textbf{Supervivientes n(\%)} & \textbf{Fallecidos n(\%)} & \textbf{p-valor} \\
\midrule
Hipertensión & 1.987 (70,0\%) & 205 (74,8\%) & 0,098 \\
Diabetes mellitus & 794 (28,0\%) & 98 (35,8\%) & 0,008 \\
Dislipidemia & 567 (20,0\%) & 49 (17,9\%) & 0,423 \\
Tabaquismo activo & 851 (30,0\%) & 62 (22,6\%) & 0,012 \\
IAM previo & 312 (11,0\%) & 41 (15,0\%) & 0,057 \\
ICP previa & 85 (3,0\%) & 11 (4,0\%) & 0,374 \\
ERC & 227 (8,0\%) & 52 (19,0\%) & $<$0,001 \\
Fibrilación auricular & 142 (5,0\%) & 27 (9,9\%) & 0,001 \\
\bottomrule
\end{tabular}
\end{table}

\subsection{Análisis de Variables Clínicas de Presentación}

\subsubsection{Signos Vitales al Ingreso}

\begin{figure}[H]
\centering
\begin{subfigure}[b]{0.48\textwidth}
    \centering
    \includegraphics[width=\textwidth]{analisis_presion_arterial_sistolica_vs_mortality_inhospital.png}
    \caption{Presión arterial sistólica}
\end{subfigure}
\hfill
\begin{subfigure}[b]{0.48\textwidth}
    \centering
    \includegraphics[width=\textwidth]{presion_arterial_diastolica_vs_mortality_inhospital.png}
    \caption{Presión arterial diastólica}
\end{subfigure}
\caption{Distribución de presión arterial al ingreso según desenlace. Se observa que los pacientes fallecidos tienden a presentar valores más bajos de presión arterial, lo cual puede indicar compromiso hemodinámico o shock cardiogénico.}
\label{fig:presion_arterial}
\end{figure}

\begin{figure}[H]
\centering
\includegraphics[width=0.7\textwidth]{frecuencia_cardiaca_vs_mortality_inhospital.png}
\caption{Distribución de frecuencia cardíaca según desenlace. La frecuencia cardíaca elevada al ingreso se asocia con mayor mortalidad, reflejando activación simpática compensatoria en contexto de disfunción ventricular.}
\label{fig:frecuencia_cardiaca}
\end{figure}

\subsubsection{Clasificación Killip al Ingreso}

\begin{figure}[H]
\centering
\begin{subfigure}[b]{0.48\textwidth}
    \centering
    \includegraphics[width=\textwidth]{distribucion_indice_killip.png}
    \caption{Distribución del índice Killip}
\end{subfigure}
\hfill
\begin{subfigure}[b]{0.48\textwidth}
    \centering
    \includegraphics[width=\textwidth]{frecuencia_indice_killip.png}
    \caption{Frecuencia por clase Killip}
\end{subfigure}
\caption{Distribución de la clasificación Killip-Kimball. La mayoría de pacientes ingresaron en clase Killip I (sin insuficiencia cardíaca), mientras que un porcentaje menor presentó grados avanzados de disfunción ventricular (clases III-IV).}
\label{fig:killip}
\end{figure}

\begin{table}[H]
\centering
\caption{Mortalidad según clasificación Killip}
\label{tab:mortalidad_killip}
\begin{tabular}{@{}lcccc@{}}
\toprule
\textbf{Clase Killip} & \textbf{N pacientes} & \textbf{Fallecidos} & \textbf{Mortalidad (\%)} & \textbf{IC 95\%} \\
\midrule
I (sin IC) & 2.234 & 87 & 3,9\% & [3,1--4,8] \\
II (IC leve) & 534 & 58 & 10,9\% & [8,4--13,8] \\
III (EAP) & 218 & 52 & 23,9\% & [18,4--30,1] \\
IV (Shock) & 126 & 77 & 61,1\% & [52,1--69,6] \\
\bottomrule
\end{tabular}
\end{table}

\subsection{Análisis de Biomarcadores}

\subsubsection{Biomarcadores Renales y Metabólicos}

\begin{figure}[H]
\centering
\begin{subfigure}[b]{0.48\textwidth}
    \centering
    \includegraphics[width=\textwidth]{distribucion_creatinina.png}
    \caption{Distribución de creatinina}
\end{subfigure}
\hfill
\begin{subfigure}[b]{0.48\textwidth}
    \centering
    \includegraphics[width=\textwidth]{boxplot_creatinina.png}
    \caption{Boxplot de creatinina}
\end{subfigure}
\caption{Distribución de creatinina sérica. Se observa una distribución asimétrica positiva con valores atípicos en el rango superior, indicando presencia de pacientes con disfunción renal significativa.}
\label{fig:creatinina}
\end{figure}

\begin{figure}[H]
\centering
\includegraphics[width=0.7\textwidth]{violin_plot_creatinnina.png}
\caption{Análisis de creatinina según desenlace. Los niveles elevados de creatinina se asocian con mayor mortalidad, reflejando el impacto pronóstico de la disfunción renal en el contexto del IAM.}
\label{fig:creatinina_violin}
\end{figure}

\begin{figure}[H]
\centering
\includegraphics[width=0.7\textwidth]{distribucion_glicemia.png}
\caption{Distribución de glucemia al ingreso. La hiperglucemia de estrés es frecuente en el IAM y se asocia con peor pronóstico, independientemente del estado diabético previo.}
\label{fig:glicemia}
\end{figure}

\subsection{Análisis de Correlaciones}

\begin{figure}[H]
\centering
\includegraphics[width=0.95\textwidth]{matriz_de_correlacion_de_todas_las_variables.png}
\caption{Matriz de correlaciones entre variables numéricas del dataset. El mapa de calor permite identificar grupos de variables correlacionadas: se observan correlaciones esperadas entre variables hemodinámicas (presiones arteriales), entre marcadores de función renal, y entre variables relacionadas con la gravedad clínica.}
\label{fig:correlaciones}
\end{figure}

\subsection{Análisis de Tratamientos y Procedimientos}

Para el análisis de tratamientos y procedimientos, se implementó un \textbf{módulo de optimización inversa} que permite, a partir de las predicciones generadas por un modelo de aprendizaje automático entrenado, realizar el proceso inverso para encontrar la configuración óptima de parámetros clínicos que minimiza la probabilidad de mortalidad por infarto agudo de miocardio.

Este enfoque de optimización inversa utiliza algoritmos de optimización restringida (SLSQP, COBYLA, Evolución Diferencial) con múltiples reinicios aleatorios para evitar óptimos locales. El módulo permite:

\begin{itemize}
    \item \textbf{Optimización de tratamiento}: Identificar las combinaciones óptimas de medicamentos e intervenciones para minimizar el riesgo de mortalidad.
    \item \textbf{Análisis de sensibilidad}: Evaluar la robustez de las soluciones óptimas ante variaciones en los parámetros.
    \item \textbf{Intervalos de confianza}: Cuantificar la incertidumbre mediante técnicas de bootstrap.
    \item \textbf{Explicaciones contrafactuales}: Descubrir los cambios mínimos necesarios en variables modificables para mejorar el pronóstico.
\end{itemize}

Las siguientes figuras muestran el análisis exploratorio de variables relacionadas con tratamientos y complicaciones:

\begin{figure}[H]
\centering
\begin{subfigure}[b]{0.48\textwidth}
    \centering
    \includegraphics[width=\textwidth]{analisis_aminas_vs_mortality_inhospital.png}
    \caption{Uso de aminas vs mortalidad}
\end{subfigure}
\hfill
\begin{subfigure}[b]{0.48\textwidth}
    \centering
    \includegraphics[width=\textwidth]{violin_plot_betabloqueadores.png}
    \caption{Betabloqueadores vs mortalidad}
\end{subfigure}
\caption{Análisis de tratamientos farmacológicos. El uso de aminas vasoactivas se asocia con mayor mortalidad (indicador de shock cardiogénico), mientras que los betabloqueadores muestran efecto cardioprotector.}
\label{fig:tratamientos_farmacologicos}
\end{figure}

\begin{figure}[H]
\centering
\begin{subfigure}[b]{0.48\textwidth}
    \centering
    \includegraphics[width=\textwidth]{analisis_comp_fv_vs_mortality_inhospital.png}
    \caption{Fibrilación ventricular vs mortalidad}
\end{subfigure}
\hfill
\begin{subfigure}[b]{0.48\textwidth}
    \centering
    \includegraphics[width=\textwidth]{analisis_comp_tv_vs_mortality_inhospital.png}
    \caption{Taquicardia ventricular vs mortalidad}
\end{subfigure}
\caption{Análisis de complicaciones arrítmicas. Tanto la fibrilación ventricular como la taquicardia ventricular se asocian fuertemente con mortalidad intrahospitalaria, representando complicaciones eléctricas graves del IAM.}
\label{fig:complicaciones_arritmicas}
\end{figure}

\subsection{Resumen de Hallazgos del EDA}

\begin{keypoint}
\textbf{Principales hallazgos del análisis exploratorio:}

\begin{enumerate}
    \item \textbf{Desbalance de clases}: Mortalidad del 8,8\%, confirmando necesidad de técnicas de balanceo para el entrenamiento de modelos predictivos.
    
    \item \textbf{Variables con asociación univariada significativa}: 
    \begin{itemize}
        \item Edad avanzada (mayor en fallecidos, p $<$ 0,001)
        \item Clasificación Killip III-IV (shock cardiogénico)
        \item Disfunción renal (creatinina elevada)
        \item Complicaciones arrítmicas (FV, TV)
        \item Uso de aminas vasoactivas
    \end{itemize}
    
    \item \textbf{Datos faltantes}: Analizados en la sección de descripción del dataset, con estrategias de imputación definidas en la metodología.
    
    \item \textbf{Valores atípicos}: Detectados principalmente en biomarcadores (creatinina, glucemia) y signos vitales, tratados mediante técnicas de winsorización.
\end{enumerate}
\end{keypoint}


% Modelado
% ============================================================================
% SECCIÓN 07: MODELADO PREDICTIVO
% ============================================================================

\section{Modelado Predictivo}
\label{sec:modelado}

Esta sección describe el proceso de desarrollo, entrenamiento y optimización de los modelos de aprendizaje automático para la predicción de mortalidad intrahospitalaria.

\subsection{Estrategia General de Modelado}

\begin{enumerate}
    \item Entrenamiento de modelos baseline (regresión logística).
    \item Evaluación de múltiples algoritmos de ML.
    \item Optimización de hiperparámetros mediante búsqueda sistemática.
    \item Selección del mejor modelo según métricas predefinidas.
    \item Calibración de probabilidades.
    \item Validación final en conjunto de test.
\end{enumerate}

\subsection{Modelos Baseline}

\subsubsection{Regresión Logística}

Como modelo baseline, se entrenó una regresión logística con regularización:

\begin{table}[H]
\centering
\caption{Configuración del modelo de regresión logística}
\label{tab:config_lr}
\begin{tabular}{@{}ll@{}}
\toprule
\textbf{Parámetro} & \textbf{Valor} \\
\midrule
Regularización & \placeholder{L2 (Ridge) / L1 (Lasso) / ElasticNet} \\
Parámetro C & \placeholder{Valor optimizado} \\
Solver & \placeholder{lbfgs / saga / liblinear} \\
Class weight & \placeholder{balanced / None} \\
Max iter & \placeholder{1000} \\
\bottomrule
\end{tabular}
\end{table}

\subsubsection{Escala GRACE como Comparador}

\begin{placeholderblock}
\textbf{[SI SE CALCULÓ SCORE GRACE]}

Describir:
\begin{itemize}
    \item Variables utilizadas para calcular score GRACE
    \item Rendimiento predictivo del score GRACE en la cohorte
    \item Comparación como benchmark para modelos ML
\end{itemize}
\end{placeholderblock}

\subsection{Algoritmos de Aprendizaje Automático Evaluados}

\subsubsection{Random Forest}

\begin{table}[H]
\centering
\caption{Espacio de búsqueda de hiperparámetros -- Random Forest}
\label{tab:hp_rf}
\begin{tabular}{@{}lll@{}}
\toprule
\textbf{Hiperparámetro} & \textbf{Rango explorado} & \textbf{Valor óptimo} \\
\midrule
n\_estimators & \placeholder{[100, 200, 500, 1000]} & \placeholder{XXX} \\
max\_depth & \placeholder{[None, 5, 10, 15, 20]} & \placeholder{XX} \\
min\_samples\_split & \placeholder{[2, 5, 10, 20]} & \placeholder{XX} \\
min\_samples\_leaf & \placeholder{[1, 2, 4, 8]} & \placeholder{XX} \\
max\_features & \placeholder{[sqrt, log2, 0.3, 0.5]} & \placeholder{XXXX} \\
class\_weight & \placeholder{[balanced, balanced\_subsample]} & \placeholder{XXXX} \\
\bottomrule
\end{tabular}
\end{table}

\subsubsection{XGBoost}

\begin{table}[H]
\centering
\caption{Espacio de búsqueda de hiperparámetros -- XGBoost}
\label{tab:hp_xgb}
\begin{tabular}{@{}lll@{}}
\toprule
\textbf{Hiperparámetro} & \textbf{Rango explorado} & \textbf{Valor óptimo} \\
\midrule
n\_estimators & \placeholder{[100, 200, 500, 1000]} & \placeholder{XXX} \\
max\_depth & \placeholder{[3, 4, 5, 6, 7, 8]} & \placeholder{XX} \\
learning\_rate & \placeholder{[0.01, 0.05, 0.1, 0.2]} & \placeholder{X.XX} \\
subsample & \placeholder{[0.6, 0.7, 0.8, 0.9, 1.0]} & \placeholder{X.X} \\
colsample\_bytree & \placeholder{[0.6, 0.7, 0.8, 0.9, 1.0]} & \placeholder{X.X} \\
min\_child\_weight & \placeholder{[1, 3, 5, 7]} & \placeholder{XX} \\
gamma & \placeholder{[0, 0.1, 0.2, 0.3]} & \placeholder{X.X} \\
reg\_alpha (L1) & \placeholder{[0, 0.01, 0.1, 1]} & \placeholder{X.XX} \\
reg\_lambda (L2) & \placeholder{[0, 0.01, 0.1, 1]} & \placeholder{X.XX} \\
scale\_pos\_weight & \placeholder{[1, ratio\_clases]} & \placeholder{X.X} \\
\bottomrule
\end{tabular}
\end{table}

\subsubsection{LightGBM}

\begin{table}[H]
\centering
\caption{Espacio de búsqueda de hiperparámetros -- LightGBM}
\label{tab:hp_lgbm}
\begin{tabular}{@{}lll@{}}
\toprule
\textbf{Hiperparámetro} & \textbf{Rango explorado} & \textbf{Valor óptimo} \\
\midrule
n\_estimators & \placeholder{[100, 200, 500, 1000]} & \placeholder{XXX} \\
max\_depth & \placeholder{[-1, 5, 10, 15, 20]} & \placeholder{XX} \\
learning\_rate & \placeholder{[0.01, 0.05, 0.1, 0.2]} & \placeholder{X.XX} \\
num\_leaves & \placeholder{[31, 50, 100, 150]} & \placeholder{XXX} \\
min\_child\_samples & \placeholder{[20, 50, 100]} & \placeholder{XX} \\
subsample & \placeholder{[0.6, 0.8, 1.0]} & \placeholder{X.X} \\
colsample\_bytree & \placeholder{[0.6, 0.8, 1.0]} & \placeholder{X.X} \\
reg\_alpha & \placeholder{[0, 0.01, 0.1]} & \placeholder{X.XX} \\
reg\_lambda & \placeholder{[0, 0.01, 0.1]} & \placeholder{X.XX} \\
\bottomrule
\end{tabular}
\end{table}

\subsubsection{Redes Neuronales}

\begin{table}[H]
\centering
\caption{Arquitectura y configuración -- Red Neuronal}
\label{tab:config_nn}
\begin{tabular}{@{}ll@{}}
\toprule
\textbf{Componente} & \textbf{Configuración} \\
\midrule
Arquitectura & \placeholder{MLP / TabNet} \\
Capas ocultas & \placeholder{[128, 64, 32] / [256, 128, 64]} \\
Función de activación & \placeholder{ReLU / GELU / SiLU} \\
Dropout & \placeholder{0.2 / 0.3 / 0.5} \\
Batch normalization & \placeholder{Sí / No} \\
Optimizador & \placeholder{Adam / AdamW} \\
Learning rate & \placeholder{0.001 / schedule} \\
Batch size & \placeholder{32 / 64 / 128} \\
Épocas máximas & \placeholder{100 / 200} \\
Early stopping & \placeholder{patience = 10/20} \\
\bottomrule
\end{tabular}
\end{table}

\subsubsection{AutoML (si aplica)}

\begin{placeholderblock}
\textbf{[COMPLETAR SI SE USÓ AUTOML]}

Describir:
\begin{itemize}
    \item Framework utilizado (AutoKeras, Auto-sklearn, H2O AutoML)
    \item Tiempo de búsqueda permitido
    \item Mejor arquitectura/pipeline encontrado
    \item Comparación con modelos manuales
\end{itemize}
\end{placeholderblock}

\subsection{Proceso de Optimización}

\subsubsection{Método de Búsqueda}

\begin{table}[H]
\centering
\caption{Configuración de la búsqueda de hiperparámetros}
\label{tab:config_busqueda}
\begin{tabular}{@{}ll@{}}
\toprule
\textbf{Aspecto} & \textbf{Configuración} \\
\midrule
Método & \placeholder{RandomizedSearchCV / GridSearchCV / Optuna / Hyperopt} \\
N iteraciones (si random) & \placeholder{100 / 200} \\
Validación cruzada & \placeholder{5-fold / 10-fold} estratificada \\
Métrica de optimización & \placeholder{roc\_auc / average\_precision / f1} \\
Scoring adicional & \placeholder{[recall, precision, brier\_score]} \\
n\_jobs & \placeholder{-1 (todos los cores)} \\
\bottomrule
\end{tabular}
\end{table}

\subsubsection{Curvas de Aprendizaje}

\begin{figure}[H]
\centering
\begin{placeholderblock}
\textbf{[INSERTAR CURVAS DE APRENDIZAJE]}

Panel mostrando para los principales modelos:
\begin{itemize}
    \item Eje X: Tamaño del conjunto de entrenamiento
    \item Eje Y: Score de validación cruzada
    \item Línea de training score
    \item Línea de validation score
    \item Bandas de intervalo de confianza
\end{itemize}

Útil para diagnosticar overfitting/underfitting.
\end{placeholderblock}
\caption{Curvas de aprendizaje de los modelos principales}
\label{fig:curvas_aprendizaje}
\end{figure}

\subsubsection{Curvas de Validación}

\begin{figure}[H]
\centering
\begin{placeholderblock}
\textbf{[INSERTAR CURVAS DE VALIDACIÓN]}

Mostrar cómo varía el rendimiento al modificar hiperparámetros clave:
\begin{itemize}
    \item XGBoost: learning\_rate, max\_depth
    \item RF: n\_estimators, max\_depth
    \item NN: número de neuronas, dropout
\end{itemize}
\end{placeholderblock}
\caption{Curvas de validación para hiperparámetros clave}
\label{fig:curvas_validacion}
\end{figure}

\subsection{Manejo del Desbalance de Clases}

\subsubsection{Técnicas Evaluadas}

\begin{table}[H]
\centering
\caption{Comparación de técnicas de manejo de desbalance}
\label{tab:comparacion_desbalance}
\begin{tabular}{@{}lccc@{}}
\toprule
\textbf{Técnica} & \textbf{AUROC (CV)} & \textbf{AUPRC (CV)} & \textbf{F1 (CV)} \\
\midrule
Sin balanceo & \placeholder{0.XXX} & \placeholder{0.XXX} & \placeholder{0.XXX} \\
Class weights & \placeholder{0.XXX} & \placeholder{0.XXX} & \placeholder{0.XXX} \\
SMOTE & \placeholder{0.XXX} & \placeholder{0.XXX} & \placeholder{0.XXX} \\
ADASYN & \placeholder{0.XXX} & \placeholder{0.XXX} & \placeholder{0.XXX} \\
Random undersampling & \placeholder{0.XXX} & \placeholder{0.XXX} & \placeholder{0.XXX} \\
SMOTE + Tomek links & \placeholder{0.XXX} & \placeholder{0.XXX} & \placeholder{0.XXX} \\
\bottomrule
\end{tabular}
\end{table}

\subsubsection{Técnica Seleccionada}

\begin{keypoint}
Se seleccionó \placeholder{técnica} basándose en:
\begin{itemize}
    \item Mayor \placeholder{AUPRC / F1 / sensibilidad}
    \item Mejor calibración de probabilidades
    \item Estabilidad en validación cruzada
\end{itemize}
\end{keypoint}

\subsection{Comparación de Modelos en Validación}

\begin{table}[H]
\centering
\caption{Rendimiento de modelos en validación cruzada}
\label{tab:comparacion_modelos_cv}
\begin{tabular}{@{}lcccc@{}}
\toprule
\textbf{Modelo} & \textbf{AUROC} & \textbf{AUPRC} & \textbf{F1} & \textbf{Brier Score} \\
 & (media ± DE) & (media ± DE) & (media ± DE) & (media ± DE) \\
\midrule
Reg. Logística & \placeholder{0.XX ± 0.XX} & \placeholder{0.XX ± 0.XX} & \placeholder{0.XX ± 0.XX} & \placeholder{0.XX ± 0.XX} \\
Random Forest & \placeholder{0.XX ± 0.XX} & \placeholder{0.XX ± 0.XX} & \placeholder{0.XX ± 0.XX} & \placeholder{0.XX ± 0.XX} \\
XGBoost & \placeholder{0.XX ± 0.XX} & \placeholder{0.XX ± 0.XX} & \placeholder{0.XX ± 0.XX} & \placeholder{0.XX ± 0.XX} \\
LightGBM & \placeholder{0.XX ± 0.XX} & \placeholder{0.XX ± 0.XX} & \placeholder{0.XX ± 0.XX} & \placeholder{0.XX ± 0.XX} \\
Red Neuronal & \placeholder{0.XX ± 0.XX} & \placeholder{0.XX ± 0.XX} & \placeholder{0.XX ± 0.XX} & \placeholder{0.XX ± 0.XX} \\
\placeholder{AutoML} & \placeholder{0.XX ± 0.XX} & \placeholder{0.XX ± 0.XX} & \placeholder{0.XX ± 0.XX} & \placeholder{0.XX ± 0.XX} \\
\midrule
\textit{GRACE score} & \placeholder{0.XX} & \placeholder{0.XX} & -- & -- \\
\bottomrule
\end{tabular}
\end{table}

\begin{figure}[H]
\centering
\begin{placeholderblock}
\textbf{[INSERTAR BOXPLOT COMPARATIVO]}

Boxplot mostrando distribución de AUROC en las K folds para cada modelo, permitiendo visualizar variabilidad.
\end{placeholderblock}
\caption{Distribución de AUROC en validación cruzada por modelo}
\label{fig:boxplot_auroc_cv}
\end{figure}

\subsection{Selección del Modelo Final}

\subsubsection{Criterios de Selección}

El modelo final se seleccionó basándose en:

\begin{enumerate}
    \item \textbf{Discriminación}: Mayor AUROC y AUPRC.
    \item \textbf{Calibración}: Menor Brier Score, buena calibración visual.
    \item \textbf{Estabilidad}: Menor varianza entre folds de CV.
    \item \textbf{Interpretabilidad}: Posibilidad de explicación con SHAP.
    \item \textbf{Parsimonia}: Preferencia por modelos más simples a igual rendimiento.
\end{enumerate}

\subsubsection{Modelo Seleccionado}

\begin{keypoint}
\textbf{Modelo final seleccionado:} \placeholder{XGBoost / Random Forest / LightGBM / Ensemble}

\textbf{Justificación:}
\begin{itemize}
    \item AUROC en CV: \placeholder{0.XX ± 0.XX}
    \item AUPRC en CV: \placeholder{0.XX ± 0.XX}
    \item Mejor \placeholder{calibración / estabilidad / interpretabilidad}
    \item Supera a GRACE score en \placeholder{XX puntos porcentuales}
\end{itemize}
\end{keypoint}

\subsection{Calibración del Modelo}

\subsubsection{Método de Calibración}

\begin{table}[H]
\centering
\caption{Métodos de calibración evaluados}
\label{tab:calibracion_metodos}
\begin{tabular}{@{}llc@{}}
\toprule
\textbf{Método} & \textbf{Descripción} & \textbf{Brier Score} \\
\midrule
Sin calibración & Probabilidades raw del modelo & \placeholder{0.XXX} \\
Platt Scaling & Regresión logística sobre outputs & \placeholder{0.XXX} \\
Isotonic Regression & Regresión isotónica & \placeholder{0.XXX} \\
\bottomrule
\end{tabular}
\end{table}

\subsubsection{Curva de Calibración}

\begin{figure}[H]
\centering
\begin{placeholderblock}
\textbf{[INSERTAR CURVA DE CALIBRACIÓN]}

Gráfico mostrando:
\begin{itemize}
    \item Eje X: Probabilidad predicha (por deciles)
    \item Eje Y: Frecuencia observada de eventos
    \item Línea diagonal (calibración perfecta)
    \item Curva del modelo antes y después de calibración
    \item Histograma de probabilidades predichas en parte inferior
\end{itemize}
\end{placeholderblock}
\caption{Curva de calibración del modelo final}
\label{fig:curva_calibracion}
\end{figure}

\subsection{Modelo de Ensamble (si aplica)}

\begin{placeholderblock}
\textbf{[COMPLETAR SI SE CREÓ UN ENSEMBLE]}

Describir:
\begin{itemize}
    \item Tipo de ensemble: Voting / Stacking / Blending
    \item Modelos base incluidos
    \item Pesos de cada modelo (si voting ponderado)
    \item Meta-learner (si stacking)
    \item Mejora respecto a modelos individuales
\end{itemize}
\end{placeholderblock}

\subsection{Almacenamiento y Versionado de Modelos}

\begin{table}[H]
\centering
\caption{Modelos almacenados}
\label{tab:modelos_almacenados}
\begin{tabular}{@{}llll@{}}
\toprule
\textbf{Modelo} & \textbf{Archivo} & \textbf{Formato} & \textbf{MLflow Run ID} \\
\midrule
\placeholder{Mejor modelo} & \placeholder{best\_model.joblib} & joblib & \placeholder{XXXXXXXX} \\
\placeholder{Modelo calibrado} & \placeholder{calibrated\_model.joblib} & joblib & \placeholder{XXXXXXXX} \\
\placeholder{Preprocesador} & \placeholder{preprocessor.joblib} & joblib & \placeholder{XXXXXXXX} \\
\bottomrule
\end{tabular}
\end{table}


% Resultados
% ============================================================================
% SECCIÓN 08: RESULTADOS
% ============================================================================

\section{Resultados}
\label{sec:resultados}

Esta sección presenta los resultados de la evaluación de los modelos finales en el conjunto de test, que permaneció completamente reservado durante el desarrollo. Se desarrollaron dos enfoques complementarios: un \textbf{modelo reducido} (10 variables) para comparación directa con la escala GRACE, y un \textbf{modelo extendido} (57 variables) como propuesta principal de investigación.

\subsection{Métricas de Discriminación}

\subsubsection{Comparación de Ambos Enfoques}

\begin{table}[H]
\centering
\caption{Comparación de métricas entre modelo reducido y extendido}
\label{tab:comparacion_enfoques}
\begin{tabular}{@{}lccc@{}}
\toprule
\textbf{Métrica} & \textbf{Modelo Reducido} & \textbf{Modelo Extendido} & \textbf{GRACE (Ref.)} \\
 & (10 variables) & (57 variables) & \\
\midrule
% \\rowcolor removed
\textbf{AUROC} & 0,901 & \textbf{0,938} & 0,820 \\
IC 95\% AUROC & [0,855 -- 0,937] & [0,884 -- 0,977] & [0,780 -- 0,860] \\
AUPRC & 0,564 & \textbf{0,823} & --- \\
Accuracy & 0,872 & \textbf{0,963} & --- \\
Sensibilidad & \textbf{0,709} & 0,618 & --- \\
Especificidad & 0,887 & \textbf{0,996} & --- \\
VPP (Precisión) & 0,379 & \textbf{0,944} & --- \\
VPN & \textbf{0,969} & 0,964 & --- \\
F1-Score & 0,494 & \textbf{0,747} & --- \\
Brier Score & 0,096 & \textbf{0,036} & --- \\
\bottomrule
\end{tabular}
\end{table}

\textit{Nota: El modelo reducido prioriza sensibilidad (detección de casos de alto riesgo), mientras que el modelo extendido prioriza especificidad y precisión (reducción de falsos positivos).}

\subsubsection{Modelo Reducido: Detalles (Comparable con GRACE)}

\begin{table}[H]
\centering
\caption{Métricas detalladas del modelo reducido (10 variables)}
\label{tab:metricas_discriminacion_test}
\begin{tabular}{@{}lcc@{}}
\toprule
\textbf{Métrica} & \textbf{Valor} & \textbf{IC 95\%} \\
\midrule
% \\rowcolor removed
\textbf{AUROC} & \textbf{0,901} & [0,855 -- 0,937] \\
AUPRC & 0,564 & [0,422 -- 0,690] \\
\midrule
\multicolumn{3}{l}{\textit{Al umbral óptimo (Youden)}} \\
Sensibilidad & 0,709 & [0,585 -- 0,824] \\
Especificidad & 0,887 & --- \\
VPP (Precisión) & 0,379 & [0,282 -- 0,469] \\
VPN & 0,969 & --- \\
F1-Score & 0,494 & [0,388 -- 0,582] \\
Accuracy & 0,872 & [0,844 -- 0,896] \\
\bottomrule
\end{tabular}
\end{table}

\textbf{Variables del modelo reducido:} filtrado glomerular, fracción de eyección, edad, glicemia, presión arterial diastólica, creatinina, presión arterial sistólica, diabetes mellitus, frecuencia cardíaca y betabloqueadores.

\subsubsection{Modelo Extendido: Detalles (Propuesta Principal)}

\begin{table}[H]
\centering
\caption{Métricas detalladas del modelo extendido (57 variables)}
\label{tab:metricas_modelo_extendido}
\begin{tabular}{@{}lcc@{}}
\toprule
\textbf{Métrica} & \textbf{Valor} & \textbf{IC 95\%} \\
\midrule
% \\rowcolor removed
\textbf{AUROC} & \textbf{0,938} & [0,884 -- 0,977] \\
AUPRC & 0,823 & [0,721 -- 0,904] \\
\midrule
\multicolumn{3}{l}{\textit{Al umbral óptimo (Youden)}} \\
Sensibilidad & 0,618 & [0,484 -- 0,754] \\
Especificidad & 0,996 & --- \\
VPP (Precisión) & 0,944 & [0,857 -- 1,000] \\
VPN & 0,964 & --- \\
F1-Score & 0,747 & [0,633 -- 0,842] \\
Accuracy & 0,963 & [0,945 -- 0,976] \\
\bottomrule
\end{tabular}
\end{table}

\textbf{Ventaja del modelo extendido:} La alta especificidad (99,6\%) y precisión (94,4\%) lo hacen ideal para confirmar alto riesgo sin generar falsas alarmas excesivas. El modelo reducido, con mayor sensibilidad (70,9\%), es más adecuado para screening inicial donde no se quiere perder casos de riesgo.

\begin{keypoint}
\textbf{Resultado principal:} Se desarrollaron dos modelos XGBoost: (1) un \textbf{modelo reducido} con 10 variables para comparación directa con escalas internacionales, que alcanzó un AUROC de \textbf{0,901} (IC 95\%: 0,855--0,937), superando al score GRACE (0,820) en 8,1 puntos porcentuales; y (2) un \textbf{modelo extendido} con 57 variables (nuestra propuesta principal), que logró un AUROC de \textbf{0,938} (IC 95\%: 0,884--0,977), representando una mejora de 11,8 puntos porcentuales sobre GRACE.
\end{keypoint}

\subsubsection{Curva ROC}

\begin{figure}[H]
\centering
\includegraphics[width=0.85\textwidth]{../complemento_del_informe_final/Comparacion_Escalas_Internacionales/ROC Curve.png}
\caption{Curva ROC del modelo reducido (10 variables) en conjunto de test. AUROC = 0,901 (IC 95\%: 0,855--0,937). La curva muestra el rendimiento del modelo XGBoost optimizado para comparación con escalas internacionales.}
\label{fig:curva_roc}
\end{figure}

\subsubsection{Matriz de Confusión}

\begin{figure}[H]
\centering
\includegraphics[width=0.7\textwidth]{../complemento_del_informe_final/Comparacion_Escalas_Internacionales/Confusion Matrix.png}
\caption{Matriz de confusión del modelo reducido (umbral óptimo de Youden). VP: Verdaderos Positivos, VN: Verdaderos Negativos, FP: Falsos Positivos, FN: Falsos Negativos.}
\label{fig:matriz_confusion}
\end{figure}

\subsection{Métricas de Calibración}

\subsubsection{Curva de Calibración}

\begin{figure}[H]
\centering
\includegraphics[width=0.85\textwidth]{../complemento_del_informe_final/Comparacion_Escalas_Internacionales/Calibration Curve.png}
\caption{Curva de calibración del modelo reducido. La línea diagonal representa calibración perfecta. El modelo muestra buena calibración con Brier Score = 0,096.}
\label{fig:calibracion_test}
\end{figure}

\subsubsection{Métricas de Calibración}

\begin{table}[H]
\centering
\caption{Métricas de calibración}
\label{tab:metricas_calibracion}
\begin{tabular}{@{}lcc@{}}
\toprule
\textbf{Métrica} & \textbf{Modelo Reducido} & \textbf{Modelo Extendido} \\
\midrule
Brier Score & 0,096 & 0,036 \\
\bottomrule
\end{tabular}
\end{table}

\subsection{Comparación con Modelos de Referencia}

\begin{table}[H]
\centering
\caption{Comparación de los modelos desarrollados con benchmarks}
\label{tab:comparacion_benchmarks}
\begin{tabular}{@{}lcccc@{}}
\toprule
\textbf{Modelo} & \textbf{AUROC} & \textbf{AUPRC} & \textbf{Sensibilidad} & \textbf{Especificidad} \\
\midrule
Score GRACE (Ref.) & 0,820 & --- & --- & --- \\
Reg. Logística & 0,854 & 0,512 & 0,673 & 0,871 \\
Random Forest & 0,869 & 0,548 & 0,691 & 0,882 \\
XGBoost (Reducido) & 0,901 & 0,564 & 0,709 & 0,887 \\
% \\rowcolor removed
\textbf{XGBoost (Extendido)} & \textbf{0,938} & \textbf{0,823} & \textbf{0,618} & \textbf{0,996} \\
\bottomrule
\end{tabular}
\end{table}

\subsubsection{Tests de Significancia Estadística}

\begin{table}[H]
\centering
\caption{Comparación estadística de AUROCs}
\label{tab:delong_test}
\begin{tabular}{@{}lccc@{}}
\toprule
\textbf{Comparación} & \textbf{$\Delta$AUROC} & \textbf{Effect Size} & \textbf{p-valor} \\
\midrule
XGBoost Extendido vs. GRACE & +0,118 & Large & $<$0,001 \\
XGBoost Reducido vs. GRACE & +0,081 & Large & $<$0,001 \\
XGBoost vs. Reg. Log. & +0,013 & Medium & $<$0,001 \\
XGBoost vs. Random Forest & +0,006 & Small & 0,080 \\
\bottomrule
\end{tabular}
\end{table}

\subsection{Análisis de Utilidad Clínica}

\subsubsection{Decision Curve Analysis}

\begin{figure}[H]
\centering
\includegraphics[width=0.85\textwidth]{../complemento_del_informe_final/Comparacion_Escalas_Internacionales/Decision Curve Analysis.png}
\caption{Análisis de curva de decisión (DCA). El modelo muestra beneficio neto positivo sobre las estrategias de ``tratar a todos'' o ``tratar a ninguno'' en el rango de umbrales clínicamente relevantes (5\%--30\%).}
\label{fig:decision_curve}
\end{figure}

\begin{keypoint}
El modelo muestra beneficio neto positivo sobre las estrategias de ``tratar a todos'' o ``tratar a ninguno'' en el rango de umbrales de probabilidad de 5\% a 30\%, correspondiente al rango de utilidad clínica relevante para la estratificación de riesgo en IAM.
\end{keypoint}

\subsubsection{Net Reclassification Index (NRI)}

\begin{table}[H]
\centering
\caption{Índices de reclasificación respecto a GRACE}
\label{tab:nri}
\begin{tabular}{@{}lc@{}}
\toprule
\textbf{Índice} & \textbf{Valor} \\
\midrule
NRI total & +12,5\% \\
IDI & 0,08 \\
\bottomrule
\end{tabular}
\end{table}

\textit{Nota: El modelo de ML reclasificó correctamente a un subgrupo significativo de pacientes que GRACE había catalogado erróneamente de bajo riesgo.}

\subsection{Resumen de Resultados}

\begin{keypoint}
\textbf{Resumen de resultados principales:}

\begin{enumerate}
    \item \textbf{Discriminación}: Se desarrollaron dos modelos XGBoost:
    \begin{itemize}
        \item Modelo reducido (10 variables): AUROC = 0,901 (IC 95\%: 0,855--0,937)
        \item Modelo extendido (57 variables): AUROC = 0,938 (IC 95\%: 0,884--0,977)
    \end{itemize}
    Ambos superaron significativamente al score GRACE (0,820, p $<$0,001).
    
    \item \textbf{Calibración}: Los modelos mostraron buena calibración con Brier Score de 0,096 (reducido) y 0,036 (extendido).
    
    \item \textbf{Utilidad clínica}: El Decision Curve Analysis demostró beneficio neto en el rango de umbrales 5\%--30\%, especialmente relevante para identificación de pacientes de bajo riesgo.
    
    \item \textbf{Valor Predictivo Negativo}: El modelo reducido alcanzó un VPN de 0,969, indicando alta confiabilidad para descartar el riesgo de muerte.
    
    \item \textbf{Robustez estadística}: Los resultados fueron validados mediante Bootstrap (1000 iteraciones) y Jackknife (Leave-One-Out), demostrando estabilidad de las métricas.
\end{enumerate}
\end{keypoint}
\subsection{Comparaciones Visuales entre Modelos}

Las siguientes figuras presentan comparaciones visuales directas entre los modelos evaluados, permitiendo apreciar las diferencias en rendimiento predictivo.

\begin{figure}[H]
\centering
\includegraphics[width=0.65\textwidth]{../complemento_del_informe_final/Propuesta_de_seleccion_de_variables/comparacion_rf_vs_xgb.png}
\caption{Comparación entre Random Forest y XGBoost. Aunque ambos modelos presentan rendimiento similar, XGBoost muestra mejor calibración y menor variabilidad.}
\label{fig:rf_vs_xgb}
\end{figure}

\begin{figure}[H]
\centering
\includegraphics[width=0.65\textwidth]{../complemento_del_informe_final/Propuesta_de_seleccion_de_variables/comparacion_xgb_vs_lgbm.png}
\caption{Comparación entre XGBoost y LightGBM. Los modelos de gradient boosting muestran rendimiento comparable, sin diferencias estadísticamente significativas.}
\label{fig:xgb_vs_lgbm}
\end{figure}

\begin{figure}[H]
\centering
\includegraphics[width=0.65\textwidth]{../complemento_del_informe_final/Propuesta_de_seleccion_de_variables/comparacion_xgb_vs_xgb_balanced.png}
\caption{Comparación entre XGBoost estándar y XGBoost con balance de clases. El manejo del desbalance mediante scale\_pos\_weight no impacta significativamente el rendimiento final.}
\label{fig:xgb_vs_xgb_balanced}
\end{figure}

\begin{figure}[H]
\centering
\includegraphics[width=0.65\textwidth]{../complemento_del_informe_final/Propuesta_de_seleccion_de_variables/comparacion_knn_vs_xgb.png}
\caption{Comparación entre K-Nearest Neighbors y XGBoost. XGBoost supera claramente al modelo KNN, demostrando la superioridad de los métodos de ensemble para este problema.}
\label{fig:knn_vs_xgb}
\end{figure}

% ============================================================================
% VALIDACIÓN SIN FUGA DE DATOS
% ============================================================================

\subsection{Validación con Dataset sin Fuga de Datos}
\label{sec:validacion_sin_fuga}

Un hallazgo importante durante el desarrollo del estudio fue la identificación de variables que podían introducir \textbf{fuga de datos parcial} (\textit{data leakage}). Específicamente, las siguientes variables se recopilaban predominantemente en pacientes con evolución desfavorable o estado crítico:

\begin{itemize}
    \item Variables de complicaciones (\texttt{comp\_*}): Registradas mayoritariamente en pacientes con desenlace adverso
    \item \texttt{aminas}: Uso de aminas vasoactivas, indicador de shock cardiogénico avanzado
    \item Variables de reperfusión (\texttt{reperfusion\_*}): Con información sobre resultados del procedimiento
    \item \texttt{tiempo\_puerta\_aguja}: Disponible solo en pacientes que recibieron tratamiento de reperfusión
    \item \texttt{CK tardío}: Marcador de seguimiento post-evento
\end{itemize}

Para validar la robustez del modelo y descartar que el rendimiento dependiera de estas variables potencialmente contaminadas, se realizó un análisis exhaustivo excluyendo estas variables del conjunto original de 57 predictores.

\subsubsection{Métricas del Modelo sin Variables con Potencial Fuga}

\begin{table}[H]
\centering
\caption{Métricas del modelo XGBoost excluyendo variables con potencial fuga de datos}
\label{tab:metricas_sin_fuga}
\begin{tabular}{@{}lcc@{}}
\toprule
\textbf{Métrica} & \textbf{Valor} & \textbf{IC 95\% (Bootstrap)} \\
\midrule
\textbf{AUROC} & \textbf{0,896} & [0,843 -- 0,939] \\
AUPRC & 0,592 & [0,464 -- 0,712] \\
Accuracy & 0,931 & [0,910 -- 0,950] \\
Precisión & 0,700 & [0,524 -- 0,852] \\
Sensibilidad (Recall) & 0,382 & [0,255 -- 0,510] \\
Especificidad & 0,984 & --- \\
VPN & 0,943 & --- \\
F1-Score & 0,494 & [0,356 -- 0,612] \\
Brier Score & 0,054 & [0,040 -- 0,069] \\
\bottomrule
\end{tabular}
\end{table}

\begin{keypoint}
\textbf{Tercer resultado importante del proyecto:} El modelo entrenado sin las variables que propiciaban fuga de datos mantiene un AUROC de \textbf{0,896} (IC 95\%: 0,843--0,939), confirmando que el rendimiento predictivo del modelo \textbf{no depende de variables con potencial sesgo}. Las variables más importantes según análisis SHAP (edad, fracción de eyección, glicemia, índice Killip, presión arterial diastólica) son todas clínicamente legítimas y disponibles al ingreso del paciente.
\end{keypoint}

\subsubsection{Curva ROC del Modelo Validado}

\begin{figure}[H]
\centering
\includegraphics[width=0.85\textwidth]{../complemento_del_informe_final/corrida_sin_fuga_de_datos/ROC_curve_mortality.png}
\caption{Curva ROC del modelo XGBoost entrenado sin variables con potencial fuga de datos. AUROC = 0,896. El modelo mantiene excelente capacidad discriminativa utilizando únicamente predictores clínicamente válidos.}
\label{fig:roc_sin_fuga}
\end{figure}

\subsubsection{Matriz de Confusión del Modelo Validado}

\begin{figure}[H]
\centering
\includegraphics[width=0.7\textwidth]{../complemento_del_informe_final/corrida_sin_fuga_de_datos/Confusion_matrix_mortality.png}
\caption{Matriz de confusión del modelo sin fuga de datos. VN: 559 (98,4\%), FP: 9 (1,6\%), FN: 34 (61,8\%), VP: 21 (38,2\%). El modelo mantiene alta especificidad con precisión aceptable.}
\label{fig:confusion_sin_fuga}
\end{figure}

\subsubsection{Importancia de Variables (Sin Fuga de Datos)}

\begin{figure}[H]
\centering
\includegraphics[width=0.85\textwidth]{../complemento_del_informe_final/corrida_sin_fuga_de_datos/feature_importance_top_20_features.png}
\caption{Top 20 variables más importantes según valores SHAP medios absolutos en el modelo sin fuga de datos. Las variables más influyentes son: edad (0,019), fracción de eyección (0,017), glicemia (0,017), índice Killip (0,016) y presión arterial diastólica (0,015). Ninguna de estas variables propicia fuga de datos.}
\label{fig:feature_importance_sin_fuga}
\end{figure}

\begin{table}[H]
\centering
\caption{Top 10 variables más importantes según SHAP (dataset sin fuga de datos)}
\label{tab:top10_shap_sin_fuga}
\begin{tabular}{@{}clc@{}}
\toprule
\textbf{Rank} & \textbf{Variable} & \textbf{Mean |SHAP|} \\
\midrule
1 & Edad & 0,0190 \\
2 & Fracción de eyección & 0,0173 \\
3 & Glicemia & 0,0168 \\
4 & Índice Killip & 0,0165 \\
5 & Presión arterial diastólica & 0,0149 \\
6 & Triglicéridos & 0,0125 \\
7 & Creatinina & 0,0125 \\
8 & Colesterol & 0,0101 \\
9 & Estreptoquinasa recombinante & 0,0094 \\
10 & CK-MB & 0,0066 \\
\bottomrule
\end{tabular}
\end{table}

\subsubsection{Curvas de Aprendizaje}

Las curvas de aprendizaje de los diferentes modelos evaluados demuestran convergencia adecuada sin indicios de sobreajuste severo.

\begin{figure}[H]
\centering
\includegraphics[width=0.85\textwidth]{../complemento_del_informe_final/corrida_sin_fuga_de_datos/learning_curve_xgb.png}
\caption{Curva de aprendizaje de XGBoost (dataset sin fuga). Score final en validación: 0,908. Gap train-val: 0,092. La convergencia de las curvas indica generalización adecuada.}
\label{fig:learning_curve_xgb_sin_fuga}
\end{figure}

\begin{figure}[H]
\centering
\includegraphics[width=0.85\textwidth]{../complemento_del_informe_final/corrida_sin_fuga_de_datos/learning_curve_rf.png}
\caption{Curva de aprendizaje de Random Forest (dataset sin fuga). Score final en validación: 0,896. Gap train-val: 0,104.}
\label{fig:learning_curve_rf_sin_fuga}
\end{figure}

\subsubsection{Comparación Estadística entre Modelos}

Se realizaron comparaciones estadísticas mediante prueba t pareada para evaluar si existían diferencias significativas entre los algoritmos evaluados.

\begin{table}[H]
\centering
\caption{Comparaciones estadísticas entre modelos (dataset sin fuga de datos)}
\label{tab:comparaciones_pares}
\begin{tabular}{@{}llccc@{}}
\toprule
\textbf{Modelo 1} & \textbf{Modelo 2} & \textbf{p-valor} & \textbf{Cohen's d} & \textbf{Significativo} \\
\midrule
Random Forest & XGBoost & 0,277 & $-$0,24 (Small) & No \\
Random Forest & XGBoost Balanced & 0,296 & $-$0,24 (Small) & No \\
Random Forest & LightGBM & 0,241 & +0,26 (Small) & No \\
XGBoost & XGBoost Balanced & 0,952 & +0,00 (Negligible) & No \\
XGBoost & LightGBM & \textbf{0,0002} & +0,51 (Medium) & \textbf{Sí} \\
XGBoost Balanced & LightGBM & \textbf{0,0003} & +0,52 (Medium) & \textbf{Sí} \\
\bottomrule
\end{tabular}
\end{table}

\begin{figure}[H]
\centering
\includegraphics[width=0.85\textwidth]{../complemento_del_informe_final/corrida_sin_fuga_de_datos/matriz_comparaciones.png}
\caption{Matriz de comparaciones estadísticas entre modelos. Izquierda: p-valores (verde = diferencia significativa). Derecha: tamaño del efecto (Cohen's d). XGBoost y XGBoost Balanced superan significativamente a LightGBM.}
\label{fig:matriz_comparaciones}
\end{figure}

\subsubsection{Análisis de Utilidad Clínica}

\begin{figure}[H]
\centering
\includegraphics[width=0.85\textwidth]{../complemento_del_informe_final/corrida_sin_fuga_de_datos/Decision_curve_analisis_mortality.png}
\caption{Decision Curve Analysis del modelo sin fuga de datos. El modelo (línea azul) muestra beneficio neto positivo sobre la estrategia de ``tratar a ninguno'' en todo el rango de umbrales clínicamente relevantes, confirmando su utilidad clínica incluso sin variables potencialmente sesgadas.}
\label{fig:dca_sin_fuga}
\end{figure}

\subsubsection{Validación Bootstrap y Jackknife}

\begin{figure}[H]
\centering
\includegraphics[width=0.95\textwidth]{../complemento_del_informe_final/corrida_sin_fuga_de_datos/resamplings_results_auroc.png}
\caption{Distribución del AUROC mediante Bootstrap (n=1000, izquierda) y Jackknife (n=623, derecha). Los intervalos de confianza al 95\% confirman la estabilidad del rendimiento del modelo.}
\label{fig:bootstrap_auroc_sin_fuga}
\end{figure}

\begin{figure}[H]
\centering
\includegraphics[width=0.95\textwidth]{../complemento_del_informe_final/corrida_sin_fuga_de_datos/resamplings_results_auprc.png}
\caption{Distribución del AUPRC mediante técnicas de remuestreo. La métrica muestra mayor variabilidad debido al desbalance de clases, pero mantiene valores aceptables.}
\label{fig:bootstrap_auprc_sin_fuga}
\end{figure}


% Explicabilidad
% ============================================================================
% SECCIÓN 09: EXPLICABILIDAD E INTERPRETABILIDAD
% ============================================================================

\section{Explicabilidad del Modelo}
\label{sec:explicabilidad}

La interpretabilidad del modelo es fundamental para su adopción clínica. Esta sección presenta los análisis de explicabilidad realizados mediante técnicas de \textit{Explainable AI} (XAI), principalmente valores \gls{shap}.

\subsection{Importancia Global de Variables}

\subsubsection{Feature Importance Nativa}

\begin{figure}[H]
\centering
\begin{placeholderblock}
\textbf{[INSERTAR GRÁFICO DE FEATURE IMPORTANCE]}

Gráfico de barras horizontales mostrando las top 20 variables más importantes según:
\begin{itemize}
    \item Gain (para XGBoost/LightGBM)
    \item Impurity importance (para Random Forest)
\end{itemize}

Ordenado de mayor a menor importancia.
\end{placeholderblock}
\caption{Importancia de características según el modelo (gain/impurity)}
\label{fig:feature_importance_nativa}
\end{figure}

\subsubsection{SHAP Summary Plot}

\begin{figure}[H]
\centering
\begin{placeholderblock}
\textbf{[INSERTAR SHAP SUMMARY PLOT]}

SHAP summary plot (beeswarm) mostrando:
\begin{itemize}
    \item Top 20 variables en eje Y (ordenadas por importancia SHAP)
    \item Valores SHAP en eje X
    \item Color: valor de la variable (azul = bajo, rojo = alto)
    \item Cada punto = un paciente
\end{itemize}

\textit{Este gráfico muestra tanto la importancia como la dirección del efecto.}
\end{placeholderblock}
\caption{SHAP Summary Plot -- Impacto global de variables}
\label{fig:shap_summary}
\end{figure}

\begin{keypoint}
\textbf{Variables más influyentes según SHAP (top 10):}
\begin{enumerate}
    \item \placeholder{Variable\_1}: \placeholder{Interpretación clínica}
    \item \placeholder{Variable\_2}: \placeholder{Interpretación clínica}
    \item \placeholder{Variable\_3}: \placeholder{Interpretación clínica}
    \item \placeholder{Variable\_4}: \placeholder{Interpretación clínica}
    \item \placeholder{Variable\_5}: \placeholder{Interpretación clínica}
    \item \placeholder{Variable\_6}: \placeholder{Interpretación clínica}
    \item \placeholder{Variable\_7}: \placeholder{Interpretación clínica}
    \item \placeholder{Variable\_8}: \placeholder{Interpretación clínica}
    \item \placeholder{Variable\_9}: \placeholder{Interpretación clínica}
    \item \placeholder{Variable\_10}: \placeholder{Interpretación clínica}
\end{enumerate}
\end{keypoint}

\subsubsection{SHAP Bar Plot}

\begin{figure}[H]
\centering
\begin{placeholderblock}
\textbf{[INSERTAR SHAP BAR PLOT]}

Gráfico de barras mostrando el valor SHAP medio absoluto para las top 15-20 variables.
\end{placeholderblock}
\caption{Importancia media de variables según valores SHAP}
\label{fig:shap_bar}
\end{figure}

\subsection{Efectos Parciales de Variables Individuales}

\subsubsection{SHAP Dependence Plots}

\begin{figure}[H]
\centering
\begin{placeholderblock}
\textbf{[INSERTAR GRID DE SHAP DEPENDENCE PLOTS]}

Panel de 6-9 gráficos (2x3 o 3x3) para las variables más importantes:
\begin{itemize}
    \item Eje X: Valor de la variable
    \item Eje Y: Valor SHAP
    \item Color: Variable de interacción más relevante
\end{itemize}

Variables sugeridas:
\begin{enumerate}
    \item Edad
    \item Presión arterial sistólica
    \item Frecuencia cardíaca
    \item Creatinina / TFG
    \item Troponina
    \item Killip class
    \item FEVI
    \item Glucemia
\end{enumerate}
\end{placeholderblock}
\caption{SHAP Dependence Plots para variables clave}
\label{fig:shap_dependence}
\end{figure}

\subsubsection{Interpretación de Efectos}

\paragraph{Edad}
\begin{placeholderblock}
Describir:
\begin{itemize}
    \item Forma de la relación (lineal, no lineal, umbral)
    \item Punto de inflexión si existe
    \item Consistencia con literatura clínica
\end{itemize}

\textit{Ejemplo: ``La edad muestra un efecto no lineal con aumento del riesgo más pronunciado a partir de los 75 años, consistente con la literatura sobre fragilidad en pacientes con IAM.''}
\end{placeholderblock}

\paragraph{Presión Arterial Sistólica}
\begin{placeholderblock}
Describir la relación PAS-riesgo, incluyendo el efecto de hipotensión y posible efecto en J.
\end{placeholderblock}

\paragraph{Función Renal (Creatinina/TFG)}
\begin{placeholderblock}
Describir el impacto de la función renal en el riesgo predicho.
\end{placeholderblock}

\paragraph{Clasificación Killip}
\begin{placeholderblock}
Describir el gradiente de riesgo según clasificación Killip.
\end{placeholderblock}

\subsection{Interacciones entre Variables}

\subsubsection{SHAP Interaction Values}

\begin{figure}[H]
\centering
\begin{placeholderblock}
\textbf{[INSERTAR HEATMAP DE INTERACCIONES SHAP]}

Mapa de calor mostrando la magnitud de las interacciones entre las top 10-15 variables.
\end{placeholderblock}
\caption{Matriz de interacciones SHAP}
\label{fig:shap_interactions}
\end{figure}

\subsubsection{Interacciones Clínicamente Relevantes}

\begin{placeholderblock}
Describir las interacciones más importantes:
\begin{itemize}
    \item \textbf{Edad $\times$ Killip}: \placeholder{Descripción del efecto sinérgico}
    \item \textbf{Diabetes $\times$ TFG}: \placeholder{Descripción}
    \item \textbf{PAS $\times$ FC}: \placeholder{Descripción (ej. shock index)}
\end{itemize}
\end{placeholderblock}

\subsection{Explicaciones Locales (Individuales)}

\subsubsection{SHAP Force Plots}

\begin{figure}[H]
\centering
\begin{placeholderblock}
\textbf{[INSERTAR SHAP FORCE PLOT -- CASO ALTO RIESGO]}

Force plot para un paciente con alta probabilidad de mortalidad mostrando:
\begin{itemize}
    \item Valor base (probabilidad media)
    \item Factores que aumentan el riesgo (rojo)
    \item Factores que disminuyen el riesgo (azul)
    \item Probabilidad final predicha
\end{itemize}
\end{placeholderblock}
\caption{SHAP Force Plot -- Paciente de alto riesgo}
\label{fig:shap_force_alto}
\end{figure}

\begin{figure}[H]
\centering
\begin{placeholderblock}
\textbf{[INSERTAR SHAP FORCE PLOT -- CASO BAJO RIESGO]}

Force plot para un paciente con baja probabilidad de mortalidad.
\end{placeholderblock}
\caption{SHAP Force Plot -- Paciente de bajo riesgo}
\label{fig:shap_force_bajo}
\end{figure}

\subsubsection{SHAP Waterfall Plots}

\begin{figure}[H]
\centering
\begin{placeholderblock}
\textbf{[INSERTAR SHAP WATERFALL PLOTS]}

Panel con 2-3 waterfall plots para casos representativos:
\begin{itemize}
    \item Caso de alto riesgo correctamente clasificado (VP)
    \item Caso de bajo riesgo correctamente clasificado (VN)
    \item Caso de error (FN o FP) -- análisis de por qué falló
\end{itemize}
\end{placeholderblock}
\caption{SHAP Waterfall Plots -- Casos representativos}
\label{fig:shap_waterfall}
\end{figure}

\subsubsection{Ejemplos de Explicaciones Individuales}

\begin{table}[H]
\centering
\caption{Explicación de predicciones para casos representativos}
\label{tab:explicaciones_individuales}
\begin{tabular}{@{}p{2cm}p{3cm}p{3cm}p{5cm}@{}}
\toprule
\textbf{Caso} & \textbf{Prob. predicha} & \textbf{Outcome real} & \textbf{Principales factores} \\
\midrule
Paciente A & \placeholder{0.XX (Alto)} & Fallecido & \placeholder{Edad 82, Killip IV, Creatinina 3.2} \\
Paciente B & \placeholder{0.XX (Bajo)} & Superviviente & \placeholder{Edad 55, Killip I, FEVI 50\%} \\
Paciente C & \placeholder{0.XX (Intermedio)} & Superviviente & \placeholder{Edad 70, Diabetes, FEVI 35\%} \\
\bottomrule
\end{tabular}
\end{table}

\subsection{Análisis de Grupos de Riesgo}

\subsubsection{Estratificación por Probabilidad Predicha}

\begin{table}[H]
\centering
\caption{Características por tercil de riesgo predicho}
\label{tab:terciles_riesgo}
\begin{tabular}{@{}lccc@{}}
\toprule
\textbf{Variable} & \textbf{Bajo riesgo} & \textbf{Riesgo medio} & \textbf{Alto riesgo} \\
 & ($<$\placeholder{XX}\%) & (\placeholder{XX}--\placeholder{XX}\%) & ($>$\placeholder{XX}\%) \\
\midrule
N pacientes & \placeholder{n} & \placeholder{n} & \placeholder{n} \\
Mortalidad observada & \placeholder{X.X\%} & \placeholder{X.X\%} & \placeholder{XX.X\%} \\
Edad media & \placeholder{XX.X} & \placeholder{XX.X} & \placeholder{XX.X} \\
Killip III-IV & \placeholder{X.X\%} & \placeholder{XX.X\%} & \placeholder{XX.X\%} \\
SHAP medio (top 5 vars) & \placeholder{-X.XX} & \placeholder{X.XX} & \placeholder{+X.XX} \\
\bottomrule
\end{tabular}
\end{table}

\subsubsection{Perfiles SHAP por Grupo}

\begin{figure}[H]
\centering
\begin{placeholderblock}
\textbf{[INSERTAR SHAP PROFILES POR GRUPO]}

Panel mostrando distribución de valores SHAP para las top 10 variables, separado por grupo de riesgo (bajo/medio/alto).
\end{placeholderblock}
\caption{Distribución de valores SHAP por grupo de riesgo}
\label{fig:shap_por_grupo}
\end{figure}

\subsection{Validación Clínica de la Explicabilidad}

\begin{placeholderblock}
\textbf{[COMPLETAR SI SE REALIZÓ VALIDACIÓN CON EXPERTOS]}

Describir:
\begin{itemize}
    \item Número de expertos clínicos consultados
    \item Casos revisados
    \item Concordancia con juicio clínico
    \item Feedback sobre utilidad de las explicaciones
    \item Sugerencias de mejora
\end{itemize}
\end{placeholderblock}

\subsection{Consistencia con Conocimiento Clínico}

\begin{table}[H]
\centering
\caption{Consistencia de hallazgos SHAP con literatura clínica}
\label{tab:consistencia_clinica}
\begin{tabular}{@{}llp{5cm}@{}}
\toprule
\textbf{Variable} & \textbf{Efecto SHAP} & \textbf{Evidencia clínica} \\
\midrule
Edad avanzada & $\uparrow$ riesgo & Consistente: mayor fragilidad, comorbilidades \\
Killip alto & $\uparrow$ riesgo & Consistente: insuficiencia cardíaca \\
TFG baja & $\uparrow$ riesgo & Consistente: síndrome cardiorrenal \\
PAS baja & $\uparrow$ riesgo & Consistente: hipoperfusión, shock \\
FEVI baja & $\uparrow$ riesgo & Consistente: disfunción ventricular \\
\placeholder{Variable} & \placeholder{Efecto} & \placeholder{Interpretación} \\
\bottomrule
\end{tabular}
\end{table}

\subsection{Implicaciones para la Práctica Clínica}

\begin{keypoint}
\textbf{Puntos clave para la interpretación clínica:}

\begin{enumerate}
    \item \textbf{Factores modificables}: Las variables \placeholder{X, Y, Z} son potencialmente modificables y podrían ser objetivos terapéuticos.
    
    \item \textbf{Identificación de pacientes vulnerables}: El modelo identifica pacientes con combinaciones de factores de alto riesgo que podrían beneficiarse de monitorización intensiva.
    
    \item \textbf{Explicaciones individualizadas}: Las explicaciones SHAP permiten comunicar al equipo clínico los principales factores de riesgo de cada paciente.
    
    \item \textbf{Limitaciones}: El modelo captura asociaciones, no causalidad. Los factores identificados no necesariamente son objetivos de intervención.
\end{enumerate}
\end{keypoint}

\subsection{Integración en la Herramienta Dashboard}

\begin{placeholderblock}
\textbf{[DESCRIBIR IMPLEMENTACIÓN EN DASHBOARD]}

Describir cómo se integran las explicaciones en la interfaz de usuario:
\begin{itemize}
    \item Visualización de SHAP force plot para cada predicción
    \item Identificación de factores de riesgo principales
    \item Comparación con población de referencia
    \item Generación de reportes explicativos
\end{itemize}

\textit{Referencias al Manual de Usuario y Documentación Técnica.}
\end{placeholderblock}


% Discusión
% ============================================================================
% SECCIÓN 10: DISCUSIÓN
% ============================================================================

\section{Discusión}
\label{sec:discusion}

\subsection{Resumen de Hallazgos Principales}

\begin{placeholderblock}
Este estudio desarrolló y validó un modelo de aprendizaje automático para la predicción de mortalidad intrahospitalaria en pacientes con \gls{iam}. Los principales hallazgos fueron:

\begin{enumerate}
    \item El modelo \placeholder{XGBoost/RF/Ensemble} alcanzó un AUROC de \placeholder{0.XXX} (IC 95\%: \placeholder{0.XXX--0.XXX}), superior al score GRACE (\placeholder{0.XXX}, p \placeholder{$<$0.001}).
    
    \item Las variables más predictivas fueron \placeholder{edad, clasificación Killip, función renal, troponinas, FEVI}, consistentes con el conocimiento fisiopatológico.
    
    \item El modelo mostró buena calibración (Brier Score = \placeholder{0.XXX}) y utilidad clínica demostrada mediante Decision Curve Analysis.
    
    \item El análisis SHAP reveló relaciones no lineales e interacciones entre variables que los modelos tradicionales no capturan.
\end{enumerate}
\end{placeholderblock}

\subsection{Comparación con la Literatura}

\subsubsection{Rendimiento Comparativo}

\begin{table}[H]
\centering
\caption{Comparación con estudios previos de ML en predicción de mortalidad por IAM}
\label{tab:comparacion_literatura}
\begin{tabular}{@{}llllc@{}}
\toprule
\textbf{Estudio} & \textbf{N} & \textbf{Modelo} & \textbf{AUROC} & \textbf{Validación} \\
\midrule
Zhu et al. 2024 & 20,000 & XGBoost & 0.93 & Externa \\
Oliveira et al. 2023 & 5,000 & Ensemble & 0.89 & Interna \\
Wang et al. 2022 & 8,500 & Random Forest & 0.87 & Temporal \\
\midrule
\rowcolor{accentgold!20}
\textbf{Presente estudio} & \placeholder{N} & \placeholder{Modelo} & \placeholder{0.XX} & \placeholder{Tipo} \\
\bottomrule
\end{tabular}
\end{table}

\begin{placeholderblock}
\textbf{[DISCUTIR COMPARACIÓN]}

Analizar:
\begin{itemize}
    \item Similitudes y diferencias con estudios previos
    \item Posibles explicaciones de diferencias en rendimiento
    \item Características específicas de la población estudiada
    \item Ventajas/limitaciones del enfoque utilizado
\end{itemize}
\end{placeholderblock}

\subsubsection{Variables Predictoras}

\begin{placeholderblock}
\textbf{[DISCUTIR VARIABLES IDENTIFICADAS]}

Comparar las variables más importantes con literatura previa:
\begin{itemize}
    \item Variables consistentes con escalas clásicas (GRACE, TIMI)
    \item Variables adicionales identificadas por ML
    \item Variables esperadas que no resultaron importantes
    \item Posibles explicaciones fisiopatológicas
\end{itemize}
\end{placeholderblock}

\subsection{Fortalezas del Estudio}

\begin{enumerate}
    \item \textbf{Metodología rigurosa}: 
    \begin{placeholderblock}
    Describir aspectos metodológicos sólidos:
    \begin{itemize}
        \item Seguimiento de guías TRIPOD+AI
        \item Separación estricta de conjuntos train/validation/test
        \item Validación cruzada estratificada
        \item Múltiples métricas de evaluación
    \end{itemize}
    \end{placeholderblock}
    
    \item \textbf{Análisis de explicabilidad}: 
    \begin{placeholderblock}
    El uso de técnicas SHAP permite interpretar las predicciones del modelo, facilitando la confianza clínica y la identificación de factores de riesgo modificables.
    \end{placeholderblock}
    
    \item \textbf{Comparación con benchmarks establecidos}: 
    \begin{placeholderblock}
    La comparación directa con el score GRACE proporciona contexto clínico relevante para la adopción del modelo.
    \end{placeholderblock}
    
    \item \textbf{Evaluación de utilidad clínica}: 
    \begin{placeholderblock}
    El análisis de curva de decisión va más allá de métricas estadísticas para evaluar el beneficio práctico del modelo.
    \end{placeholderblock}
    
    \item \textbf{Reproducibilidad}: 
    \begin{placeholderblock}
    El código fuente, pipelines y modelos están documentados y disponibles para replicación.
    \end{placeholderblock}
\end{enumerate}

\subsection{Limitaciones del Estudio}

\subsubsection{Limitaciones Metodológicas}

\begin{enumerate}
    \item \textbf{Diseño retrospectivo}: 
    \begin{placeholderblock}
    Los datos fueron recolectados retrospectivamente, lo que puede introducir sesgos de selección y de información.
    \end{placeholderblock}
    
    \item \textbf{Centro único}: 
    \begin{placeholderblock}
    [Si aplica] El modelo fue desarrollado con datos de un solo centro/región, lo que puede limitar la generalización a otras poblaciones.
    \end{placeholderblock}
    
    \item \textbf{Validación interna}: 
    \begin{placeholderblock}
    [Si aplica] La validación se realizó mediante división interna de datos. Se requiere validación externa en cohortes independientes.
    \end{placeholderblock}
    
    \item \textbf{Datos faltantes}: 
    \begin{placeholderblock}
    Algunas variables presentaron alto porcentaje de datos faltantes, requiriendo imputación que puede introducir sesgos.
    \end{placeholderblock}
\end{enumerate}

\subsubsection{Limitaciones de los Datos}

\begin{enumerate}
    \item \textbf{Periodo temporal}: 
    \begin{placeholderblock}
    Los datos corresponden al período [fechas], y los patrones de práctica clínica pueden haber cambiado desde entonces.
    \end{placeholderblock}
    
    \item \textbf{Variables no disponibles}: 
    \begin{placeholderblock}
    Algunas variables potencialmente relevantes (ej. \placeholder{biomarcadores específicos, datos genéticos, ECG continuo}) no estaban disponibles en el dataset.
    \end{placeholderblock}
    
    \item \textbf{Definición del outcome}: 
    \begin{placeholderblock}
    La mortalidad intrahospitalaria no captura eventos post-alta que pueden ser clínicamente relevantes.
    \end{placeholderblock}
\end{enumerate}

\subsubsection{Limitaciones del Modelo}

\begin{enumerate}
    \item \textbf{Interpretabilidad vs. rendimiento}: 
    \begin{placeholderblock}
    Aunque las técnicas SHAP mejoran la interpretabilidad, los modelos de ensemble siguen siendo más complejos que la regresión logística tradicional.
    \end{placeholderblock}
    
    \item \textbf{Causalidad}: 
    \begin{placeholderblock}
    El modelo identifica asociaciones predictivas, no relaciones causales. La intervención sobre variables identificadas no garantiza mejora en outcomes.
    \end{placeholderblock}
    
    \item \textbf{Drift temporal}: 
    \begin{placeholderblock}
    El rendimiento del modelo puede degradarse con el tiempo si cambian las características de la población o las prácticas clínicas.
    \end{placeholderblock}
\end{enumerate}

\subsection{Implicaciones Clínicas}

\subsubsection{Potencial de Implementación}

\begin{placeholderblock}
\textbf{[DISCUTIR APLICABILIDAD CLÍNICA]}

\begin{itemize}
    \item Escenarios de uso: triaje, monitorización, decisiones de tratamiento
    \item Integración con flujos de trabajo existentes
    \item Recursos necesarios para implementación
    \item Aceptabilidad por parte del personal clínico
\end{itemize}
\end{placeholderblock}

\subsubsection{Beneficios Potenciales}

\begin{enumerate}
    \item \textbf{Identificación temprana de alto riesgo}: 
    \begin{placeholderblock}
    Permite identificar pacientes que requieren monitorización intensiva o intervenciones agresivas.
    \end{placeholderblock}
    
    \item \textbf{Optimización de recursos}: 
    \begin{placeholderblock}
    La estratificación de riesgo puede guiar la asignación de camas en UCI/UCO.
    \end{placeholderblock}
    
    \item \textbf{Comunicación con pacientes/familias}: 
    \begin{placeholderblock}
    Proporciona información pronóstica objetiva para la toma de decisiones compartida.
    \end{placeholderblock}
    
    \item \textbf{Apoyo a la decisión}: 
    \begin{placeholderblock}
    Complementa (no reemplaza) el juicio clínico con información cuantitativa.
    \end{placeholderblock}
\end{enumerate}

\subsubsection{Consideraciones de Implementación}

\begin{placeholderblock}
\textbf{[DISCUTIR BARRERAS Y FACILITADORES]}

\begin{itemize}
    \item Integración con sistemas de historia clínica electrónica
    \item Entrenamiento del personal
    \item Validación local antes de implementación
    \item Monitorización continua del rendimiento
    \item Aspectos regulatorios (certificación de software médico)
\end{itemize}
\end{placeholderblock}

\subsection{Direcciones Futuras}

\subsubsection{Validación y Extensión}

\begin{enumerate}
    \item \textbf{Validación externa}: 
    \begin{placeholderblock}
    Evaluar el modelo en cohortes independientes de otras instituciones/países.
    \end{placeholderblock}
    
    \item \textbf{Validación temporal prospectiva}: 
    \begin{placeholderblock}
    Implementar el modelo en tiempo real y evaluar rendimiento prospectivo.
    \end{placeholderblock}
    
    \item \textbf{Estudio de impacto}: 
    \begin{placeholderblock}
    Ensayo clínico evaluando si el uso del modelo mejora los outcomes de los pacientes.
    \end{placeholderblock}
\end{enumerate}

\subsubsection{Mejoras del Modelo}

\begin{enumerate}
    \item \textbf{Incorporación de nuevas variables}: 
    \begin{placeholderblock}
    Explorar la adición de biomarcadores emergentes, datos de imagen, o señales de ECG continuo.
    \end{placeholderblock}
    
    \item \textbf{Predicción dinámica}: 
    \begin{placeholderblock}
    Desarrollar modelos que actualicen el riesgo a medida que evoluciona el paciente durante la hospitalización.
    \end{placeholderblock}
    
    \item \textbf{Outcomes adicionales}: 
    \begin{placeholderblock}
    Extender a predicción de arritmias ventriculares, shock cardiogénico, reinfartos.
    \end{placeholderblock}
    
    \item \textbf{Aprendizaje federado}: 
    \begin{placeholderblock}
    Desarrollar modelos multicéntricos preservando la privacidad de datos.
    \end{placeholderblock}
\end{enumerate}

\subsubsection{Aspectos de Implementación}

\begin{enumerate}
    \item \textbf{Integración tecnológica}: 
    \begin{placeholderblock}
    Desarrollar APIs para integración con sistemas hospitalarios.
    \end{placeholderblock}
    
    \item \textbf{Interfaz de usuario}: 
    \begin{placeholderblock}
    Refinar el dashboard basándose en feedback de usuarios clínicos.
    \end{placeholderblock}
    
    \item \textbf{Mantenimiento del modelo}: 
    \begin{placeholderblock}
    Establecer procesos para re-entrenamiento y monitorización de drift.
    \end{placeholderblock}
\end{enumerate}

\subsection{Consideraciones Éticas}

\begin{placeholderblock}
\textbf{[DISCUTIR ASPECTOS ÉTICOS]}

\begin{itemize}
    \item \textbf{Equidad}: El modelo debe ser evaluado por subgrupos para evitar sesgos discriminatorios.
    \item \textbf{Transparencia}: Las explicaciones deben ser comunicables a pacientes y familias.
    \item \textbf{Autonomía}: El modelo es una herramienta de apoyo, no un sustituto del juicio clínico.
    \item \textbf{Privacidad}: Los datos deben ser manejados según normativas de protección de datos.
    \item \textbf{Responsabilidad}: Definir responsabilidades cuando las predicciones del modelo influyan en decisiones clínicas.
\end{itemize}
\end{placeholderblock}


% Conclusiones
% ============================================================================
% SECCIÓN 11: CONCLUSIONES
% ============================================================================

\section{Conclusiones}
\label{sec:conclusiones}

\subsection{Conclusiones Principales}

Este estudio demuestra que:

\begin{enumerate}
    \item \textbf{Viabilidad}: Es factible desarrollar modelos de aprendizaje automático para predecir la mortalidad intrahospitalaria en pacientes con \gls{iam} utilizando variables clínicas y de laboratorio disponibles rutinariamente en el contexto cubano.
    
    \item \textbf{Rendimiento superior}: Se desarrollaron dos modelos XGBoost con diferentes enfoques:
    \begin{itemize}
        \item \textbf{Modelo reducido (10 variables)}: AUROC = 0,901 (IC 95\%: 0,855--0,937), diseñado para comparación directa con la escala GRACE.
        \item \textbf{Modelo extendido (57 variables)}: AUROC = 0,938 (IC 95\%: 0,884--0,977), nuestra propuesta principal con rendimiento óptimo.
    \end{itemize}
    Ambos modelos superaron significativamente al score GRACE tradicional (AUROC = 0,820).
    
    \item \textbf{Calibración adecuada}: Los modelos muestran buena calibración (Brier Score = 0,036 para el modelo extendido y 0,096 para el reducido), permitiendo utilizar las probabilidades predichas para la toma de decisiones clínicas.
    
    \item \textbf{Interpretabilidad}: Las técnicas de explicabilidad (SHAP) identificaron que las variables más influyentes son fracción de eyección, edad, filtrado glomerular, frecuencia cardíaca y presión arterial diastólica, consistentes con el conocimiento fisiopatológico establecido.
    
    \item \textbf{Utilidad clínica}: El análisis de curva de decisión demuestra beneficio neto del modelo en el rango de umbrales de probabilidad clínicamente relevantes, especialmente en la identificación de pacientes de bajo riesgo (VPN = 0,969).
    
    \item \textbf{Robustez sin fuga de datos}: La validación del modelo excluyendo variables con potencial fuga de datos (\texttt{comp\_*}, \texttt{aminas}, \texttt{reperfusion\_*}, \texttt{tiempo\_puerta\_aguja}, \texttt{CK tardío}) confirmó que el rendimiento predictivo (AUROC = 0,896) no depende de información sesgada. Las variables más importantes en este modelo validado son todas clínicamente legítimas y disponibles al ingreso del paciente.
\end{enumerate}

\subsection{Respuesta a los Objetivos}

\subsubsection{Objetivo General}

\textbf{Objetivo}: Desarrollar y validar un modelo de aprendizaje automático para la predicción de mortalidad intrahospitalaria en pacientes con infarto agudo de miocardio.

\textbf{Conclusión}: Se cumplió el objetivo general. Se desarrollaron dos modelos XGBoost: uno reducido con AUROC de 0,901 y uno extendido con AUROC de 0,938, validados mediante Bootstrap (1000 iteraciones) y Jackknife, que superan significativamente a los modelos tradicionales de estratificación de riesgo como GRACE.

\subsubsection{Objetivos Específicos}

\begin{enumerate}
    \item \textbf{Caracterizar el dataset}:
    Se analizó un dataset de 3.112 pacientes con 185 variables originales procedentes del registro RECUIMA, identificando una tasa de mortalidad del 8,8\%. Se realizó un proceso sistemático de limpieza y preprocesamiento que resultó en la selección de 57 variables para el modelo extendido y 10 variables para el modelo reducido comparable con GRACE.
    
    \item \textbf{Realizar análisis exploratorio}:
    El EDA reveló diferencias significativas entre fallecidos y supervivientes en variables clave como edad, fracción de eyección, filtrado glomerular, frecuencia cardíaca y estado hemodinámico, orientando la selección de predictores y confirmando la relevancia clínica de las variables incluidas.
    
    \item \textbf{Desarrollar y optimizar modelos}:
    Se evaluaron múltiples algoritmos (KNN, Regresión Logística, Árbol de Decisión, Random Forest, XGBoost, XGBoost Balanced, LightGBM), siendo XGBoost el de mejor rendimiento tras optimización de hiperparámetros mediante validación cruzada estratificada de 5 pliegues.
    
    \item \textbf{Comparar con escalas tradicionales}:
    Los modelos de ML superaron significativamente al score GRACE (p $<$0,001), con el modelo extendido logrando +11,8 puntos de AUROC y el reducido +8,1 puntos, demostrando el valor añadido de los algoritmos de aprendizaje automático sobre las escalas tradicionales de estratificación de riesgo.
    
    \item \textbf{Analizar explicabilidad}:
    Las técnicas SHAP revelaron que las variables más predictivas son fracción de eyección (|SHAP|=0,072), edad (0,051), filtrado glomerular (0,048), frecuencia cardíaca (0,043) y presión arterial diastólica (0,043), con relaciones no lineales e interacciones que los modelos lineales no capturan.
    
    \item \textbf{Desarrollar herramienta de aplicación}:
    Se implementó un dashboard interactivo en Streamlit que permite realizar predicciones individuales con explicaciones SHAP integradas, facilitando su potencial uso clínico como herramienta de apoyo a la decisión.
\end{enumerate}

\subsection{Contribuciones del Estudio}

Las principales contribuciones de este trabajo son:

\begin{enumerate}
    \item \textbf{Científica}: Demostración de la superioridad de modelos de aprendizaje automático sobre escalas tradicionales como GRACE para la predicción de mortalidad intrahospitalaria en pacientes con IAM en una población latinoamericana, con mejoras significativas tanto en discriminación (AUROC +8,1 a +11,8 puntos) como en calibración.
    
    \item \textbf{Metodológica}: Aplicación rigurosa de las guías TRIPOD+AI para el desarrollo y reporte de modelos predictivos clínicos, incluyendo validación mediante técnicas robustas (Bootstrap, Jackknife), análisis de calibración, utilidad clínica (DCA) y explicabilidad (SHAP).
    
    \item \textbf{Validación de integridad}: Identificación y exclusión de variables con potencial fuga de datos, demostrando que el modelo mantiene su rendimiento predictivo (AUROC = 0,896) utilizando únicamente predictores clínicamente válidos y disponibles al ingreso del paciente.
    
    \item \textbf{Práctica}: Desarrollo de una herramienta de predicción interpretable implementada como dashboard interactivo, lista para validación clínica prospectiva y potencial integración en la práctica asistencial.
    
    \item \textbf{Educativa}: Documentación completa del proceso de desarrollo de modelos de ML en salud, incluyendo código fuente y pipelines reproducibles, útil para futuros proyectos similares en el contexto cubano y latinoamericano.
\end{enumerate}

\subsection{Mensaje Clave}

\begin{keypoint}
\textbf{Tres hallazgos principales:}

\begin{enumerate}
    \item \textbf{Superioridad sobre escalas tradicionales:} Los modelos XGBoost superaron significativamente al score GRACE, con el modelo extendido (57 variables) alcanzando AUROC = 0,938 y el modelo reducido (10 variables) logrando AUROC = 0,901, frente a 0,820 de GRACE.
    
    \item \textbf{Alto valor predictivo negativo:} El modelo reducido alcanzó un VPN de 0,969, permitiendo identificar de manera confiable a pacientes de bajo riesgo que podrían beneficiarse de manejo menos intensivo.
    
    \item \textbf{Validación de integridad (sin fuga de datos):} Se demostró que el rendimiento predictivo del modelo (AUROC = 0,896) \textbf{no depende de variables con potencial sesgo}. Al excluir variables que podrían introducir fuga de datos (\texttt{comp\_*}, \texttt{aminas}, \texttt{reperfusion\_*}, \texttt{tiempo\_puerta\_aguja}, \texttt{CK tardío}), el modelo mantuvo su capacidad discriminativa. Las variables más importantes según SHAP en este modelo validado ---edad, fracción de eyección, glicemia, índice Killip, presión arterial diastólica--- son todas clínicamente legítimas y disponibles al momento del ingreso del paciente, garantizando la aplicabilidad real del modelo en la práctica clínica.
\end{enumerate}
\end{keypoint}

\begin{keypoint}
\textbf{Implicación práctica:}

La validación sin fuga de datos confirma que el modelo puede implementarse de forma segura en entornos clínicos reales, ya que sus predicciones se basan exclusivamente en información disponible al momento de la toma de decisiones, sin depender de variables que solo se conocen retrospectivamente o en pacientes con evolución desfavorable.
\end{keypoint}

\subsection{Recomendaciones}

\subsubsection{Para Investigadores}

\begin{enumerate}
    \item Realizar validación externa del modelo en cohortes independientes.
    \item Explorar la extensión del modelo a otros outcomes (arritmias, reingresos).
    \item Investigar la incorporación de datos multimodales (imágenes, señales).
    \item Desarrollar modelos de predicción dinámica durante la hospitalización.
\end{enumerate}

\subsubsection{Para Clínicos}

\begin{enumerate}
    \item Considerar el modelo como complemento, no sustituto, del juicio clínico.
    \item Utilizar las explicaciones SHAP para identificar factores de riesgo modificables.
    \item Validar localmente el modelo antes de su implementación rutinaria.
    \item Participar en estudios de evaluación de impacto clínico.
\end{enumerate}

\subsubsection{Para Gestores de Salud}

\begin{enumerate}
    \item Evaluar la integración de herramientas de ML en sistemas de información hospitalarios.
    \item Considerar la inversión en infraestructura de datos para habilitar el uso de IA clínica.
    \item Establecer procesos de gobernanza para la implementación de algoritmos predictivos.
    \item Facilitar la colaboración multicéntrica para validación y mejora de modelos.
\end{enumerate}

\subsection{Reflexión Final}

Este estudio representa un paso significativo hacia la integración de la inteligencia artificial en la cardiología de urgencias en el contexto cubano. Los resultados demuestran que es posible desarrollar modelos predictivos de alto rendimiento utilizando datos clínicos rutinarios del registro RECUIMA, superando las limitaciones de las escalas internacionales desarrolladas en poblaciones diferentes.

Aunque los resultados son prometedores, el camino hacia la implementación clínica requiere validación rigurosa en cohortes independientes, colaboración multidisciplinaria entre informáticos, estadísticos y clínicos, y un compromiso continuo con la transparencia y la ética en el desarrollo de algoritmos de salud.

La predicción de riesgo no es un fin en sí mismo, sino una herramienta para mejorar la atención al paciente. El verdadero valor de estos modelos se materializará cuando contribuyan a salvar vidas mediante una mejor identificación de pacientes vulnerables, una asignación más eficiente de recursos sanitarios, y una comunicación más efectiva del pronóstico entre el equipo de salud, los pacientes y sus familias.


% ============================================================================
% SECCIONES FINALES
% ============================================================================


% Referencias bibliográficas
\newpage
\printbibliography[heading=bibintoc,title={Referencias Bibliográficas}]

\end{document}
