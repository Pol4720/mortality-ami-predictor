%===================================================================================
% ARTÍCULO: Documentación Científica de la App "mortality-ami-predictor"
% Basado en la plantilla `Informe JCE.tex` adaptada para describir la aplicación,
% su arquitectura, metodología y resultados resumidos a partir de la carpeta `Tools/docs`.
%===================================================================================
\documentclass[a4paper,10pt,twoside]{article}
\usepackage{latexjcefisica}
\MakeOuterQuote{"}
\hypersetup{colorlinks, citecolor=blue, filecolor=blue, linkcolor=blue, urlcolor=blue}

\title{Predictor de Mortalidad en Infarto Agudo de Miocardio: \\ Documento Científico de la Aplicación}
\author{\name Equipo de Desarrollo\inst{1}\\
\institute \addr Proyecto `mortality-ami-predictor`.\\ Universidad / Organización\\
\email \href{mailto:equipo@ejemplo.org}{equipo@ejemplo.org}}
\tutors{Repositorio: \texttt{Pol4720/mortality-ami-predictor}}
\jcefisicaheading{2025}{1-\pageref{end}}{Equipo de Desarrollo}
\ShortHeadings{Predictor mortalidad IAM}{Equipo de Desarrollo}{2025}

\begin{document}
\twocolumn[
\begin{@twocolumnfalse}
\maketitle
\selectlanguage{spanish}
\vspace*{-.5cm}
\begin{center}\rule{.95\textwidth}{.5mm} \end{center}
\begin{abstract}
Presentamos la adaptación de la plantilla de la Jornada Científica Estudiantil para
documentar la aplicación `mortality-ami-predictor`. El artículo describe el objetivo
de la app, la estructura de datos, los pasos de preprocesamiento, la selección de
variables, el entrenamiento de modelos, la evaluación, la exportación de modelos y
las herramientas de despliegue (dashboard). El contenido está basado en la
documentación interna disponible en `Tools/docs` y en el código del repositorio.
\end{abstract}
\begin{keywords}
Infarto agudo de miocardio, predicción de mortalidad, machine learning, reproducibilidad, explainability
\end{keywords}
]\vspace*{-.5cm}

\section{Introducción}
La aplicación `mortality-ami-predictor` provee un flujo reproducible para la
predicción de riesgo de mortalidad en pacientes con infarto agudo de miocardio (IAM).
El proyecto integra:
- conjuntos de datos clínicos (carpeta `DATA/`),
- módulos para carga y limpieza (`Tools/docs/data_load`, `src/data_load`),
- componentes de preprocesamiento y extracción de características (`Tools/docs/preprocessing`, `src/preprocessing`),
- implementaciones de modelos y pipelines (`Tools/docs/models`, `src/models`),
y una interfaz de despliegue tipo dashboard (`Tools/dashboard`).

El objetivo del artículo es describir la arquitectura y metodología,
documentar decisiones clave y reportar resultados de evaluación para facilitar
la reproducibilidad y la transferencia del sistema a entornos clínicos de prueba.

\section{Datos}
Los datos utilizados se encuentran en `DATA/` (por ejemplo `recuima-020425.csv`).
La estructura contiene variables demográficas, signos vitales, resultados de laboratorio
y variables clínicas relevantes para IAM. El repositorio incluye además
`variable_metadata.json` con metadatos describiendo cada variable.

De acuerdo con la documentación de carga y utilitarios (`Tools/docs/data_load`),
los pasos iniciales son:
\begin{itemize}
  \item validación y lectura eficiente de CSVs,
  \item preservación de metadatos y mapeo de nombres de columnas,
  \item particionado reproducible (train/validation/test) con splitters configurables.
\end{itemize}

\section{Preprocesamiento y Features}
La etapa de limpieza y preprocesamiento sigue las guías de `Tools/docs/api/cleaning` y
`Tools/docs/preprocessing/pipelines` e incluye:
\begin{itemize}
  \item imputación de valores faltantes (métodos configurables),
  \item discretización/encoding de variables categóricas,
  \item detección y manejo de outliers,
  \item ingeniería y selección de características según `features/`.
\end{itemize}

Los transformadores y selectores están encapsulados en pipelines y son reutilizables
en entrenamiento y en la predicción en producción, garantizando congruencia entre
entrenamiento y servicio.

\section{Modelos y Entrenamiento}
La carpeta `Tools/docs/models` y `src/training` documentan el conjunto de modelos
soportados (clasificadores clásicos y redes) y la estrategia de entrenamiento:
\begin{itemize}
  \item búsqueda y ajuste de hiperparámetros (`hyperparameter_tuning`),
  \item validación cruzada y particionado temporal/estratificado según disponibilidad de datos,
  \item persistencia de modelos y artefactos con `joblib` y registro en `mlruns`.
\end{itemize}

En el repositorio hay modelos serializados (`models/best_classifier_test.joblib`) y
experimentos registrados en `mlruns/` que documentan métricas y parámetros.

\section{Evaluación}
La evaluación se adhiere a las recomendaciones en `Tools/docs/evaluation` e incluye:
\begin{itemize}
  \item métricas de discriminación: AUC-ROC, AUC-PR,
  \item métricas de calibración y curvas de calibrado,
  \item análisis de sensibilidad y curvas de decisión para soporte clínico,
  \item generación de reportes PDF con resúmenes y visualizaciones (`eda/pdf_reports`, `evaluation/pdf_reports`).
\end{itemize}

Donde es posible, los informes automatizados y las gráficas se guardan en `processed/plots`.

\section{Explainability y Análisis}
La documentación sobre explainability (`Tools/docs/explainability`) detalla
implementaciones de SHAP, parcial dependence y análisis por permutación, usadas para:
\begin{itemize}
  \item interpretar contribuciones de variables a predicciones individuales y de población,
  \item identificar variables clínicamente relevantes y validar coherencia con la literatura,
  \item producir visualizaciones para usuarios clínicos en el dashboard.
\end{itemize}

\section{Despliegue y Dashboard}
El componente de despliegue incluye un dashboard (carpeta `Tools/dashboard`) que
proporciona páginas para:
\begin{itemize}
  \item carga de casos y predicción en lote o por paciente,
  \item visualización de explicaciones y reportes PDF,
  \item generación de reportes para descarga.
\end{itemize}

La infraestructura está documentada en `Tools/docker` y archivos de entorno
(`Tools/environment.yml`, `requirements.txt`) para facilitar despliegue reproducible.

\section{Resultados (Resumen)}
Presentamos un resumen de resultados reproducible desde los artefactos del repo:
\begin{itemize}
  \item modelos serializados (ver `models/`) muestran rendimiento competitivo en métricas internas;
  \item experimentos registrados en `mlruns/` permiten replicar los pasos y métricas reportadas.
\end{itemize}

Una evaluación exhaustiva con tablas y figuras detalladas debería extraerse de los
reportes en `Tools/docs/evaluation/pdf_reports` y de los `mlruns` para incluir en
la versión final del artículo.

\section{Reproducibilidad y Buenas Prácticas}
El proyecto contiene recursos para reproducibilidad:
\begin{itemize}
  \item `Tools/environment.yml` y `requirements.txt` para reconstruir entornos,
  \item scripts en `Tools/scripts` y `docker/` para ejecución reproducible,
  \item pruebas unitarias en `tests/` que validan componentes claves.
\end{itemize}

Recomendamos ejecutar pipelines en un entorno controlado (conda / Docker) e
usar los experimentos de `mlruns` para conservar trazabilidad.

\section{Conclusiones}
La aplicación `mortality-ami-predictor` consolida un flujo completo desde la
ingesta de datos hasta la entrega de predicciones y explicaciones clínicas.
Este artículo adapta la plantilla de la Jornada Científica para dejar constancia
de la arquitectura, metodología y recursos disponibles para validación
posterior en entornos clínicos.

\section*{Recomendaciones}
Se recomienda:
\begin{itemize}
  \item añadir un apartado de evaluación clínica y validación externa con cohortes independientes,
  \item documentar cuidadosamente los criterios de inclusión/exclusión en los datos,
  \item preparar una sección de implicaciones éticas y manejo de datos sensibles.
\end{itemize}

\section*{Agradecimientos}
Agradecemos al equipo del repositorio y a los autores de la documentación interna
(`Tools/docs`) por facilitar la trazabilidad del proyecto.

\begin{thebibliography}{9}
  \bibitem{repo} Pol4720. ``mortality-ami-predictor'' repository. URL: \texttt{https://github.com/Pol4720/mortality-ami-predictor}.
  \bibitem{docsindex} Documentación del proyecto, carpeta `Tools/docs` en el repositorio.
  \bibitem{latexjce} Plantilla `Informe JCE.tex` adaptada para este artículo.
\end{thebibliography}

\label{end}
\end{document}
