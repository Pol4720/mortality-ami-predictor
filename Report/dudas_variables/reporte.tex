\documentclass[11pt,a4paper]{article}

% Configuración básica
\usepackage[margin=1.8cm,bottom=2.5cm]{geometry}
\usepackage[spanish,es-nodecimaldot]{babel}
\usepackage[utf8]{inputenc}
\usepackage[T1]{fontenc}
\usepackage{lmodern}
\usepackage{csquotes}
\usepackage{microtype} % Mejora el espaciado y reduce overfull
\usepackage{hyperref}
\hypersetup{
    colorlinks=true,
    linkcolor=black,
    urlcolor=blue
}

% Tablas
\usepackage{array}
\usepackage{booktabs}
\usepackage{longtable}
\usepackage{multirow}
\usepackage{xcolor}
\usepackage{tabularx}
\usepackage{ragged2e}
\renewcommand{\arraystretch}{1.2}

% Ajustes para mejorar el espaciado y evitar overfull
\setlength{\emergencystretch}{4em}
\setlength{\hfuzz}{2pt}
\tolerance=5000
\hbadness=5000
\sloppy


% Utilidad para celdas de párrafo alineadas a la izquierda
\newcolumntype{L}[1]{>{\RaggedRight\arraybackslash}p{#1}}
% Columna que ajusta automáticamente el ancho
\newcolumntype{Y}{>{\RaggedRight\arraybackslash}X}

% Metadatos personalizables
\newcommand{\DatasetNombre}{RECUIMA (Registro Cubano de Infarto Agudo de Miocardio)}
\newcommand{\DatasetVersion}{v0.2 (post-limpieza inicial)}
\newcommand{\Proyecto}{Predicción de Mortalidad en Infarto Agudo de Miocardio}
\newcommand{\Responsable}{Equipo ML}

\title{Planilla de dudas sobre variables del dataset}
\author{\Responsable}
\date{\today}

\begin{document}
\maketitle

\section*{Resumen del dataset}
\small
\setlength{\tabcolsep}{3pt}
\begin{tabular}{@{}L{0.28\textwidth} L{0.68\textwidth}@{}}
\toprule
Nombre del dataset & \DatasetNombre \\
Versión & \DatasetVersion \\
Fuente / Sistema origen & Registro hospitalario de pacientes con infarto agudo de miocardio \\
Periodo de cobertura & 2016--2025 \\
Población y unidad de análisis & Pacientes ingresados con diagnóstico de IAM;unidad: episodio de internación \\
Número de registros / variables & 3,112 registros / 185 variables (después de limpieza inicial) \\
Fecha de extracción & 02/04/2025 \\
Objetivo analítico & Predicción de mortalidad intrahospitalaria en pacientes con IAM mediante modelos de machine learning \\
Restricciones legales / privacidad & Se eliminaron variables de identificación personal (nombres, números de identidad, números de contacto) para cumplir con protección de datos \\
Notas generales de calidad (duplicados, faltantes, codificaciones especiales) & Se observa presencia significativa de valores faltantes en múltiples variables. Existen duplicaciones de variables clave que requieren aclaración. El dataset parece ser resultado de la fusión de múltiples registros o fuentes. Se eliminaron variables redundantes identificadas en la limpieza inicial: \texttt{anno}, \texttt{numero} (identificador), \texttt{unidad}. \\
\bottomrule
\end{tabular}

\vspace{1em}
\noindent\textbf{Nota importante sobre duplicaciones:} Se han identificado las siguientes variables que aparecen duplicadas en el dataset: \texttt{presion\_arterial\_sistolica}, \texttt{presion\_arterial\_diastolica}, \texttt{asa}, \texttt{betabloqueadores}, \texttt{ieca}, \texttt{estatinas}, \texttt{clopidogrel}, \texttt{furosemida}, \texttt{nitratos}, \texttt{anticoagulantes}, \texttt{otros\_diureticos}, \texttt{fecha\_egreso} y \texttt{fecha\_ingreso}. Se desconoce el motivo exacto de estas duplicaciones, aunque se presume que puede deberse a: (1) registro en diferentes momentos temporales (ingreso vs. egreso), (2) fusión de múltiples fuentes de datos, o (3) diferencias entre prescripción y administración real. Se requiere aclaración urgente sobre la interpretación correcta de estas columnas duplicadas.

\section*{Guía rápida}
- Tipo: numérico, categórico, booleano, fecha/hora, texto libre, identificador. 
- Códigos especiales: por ejemplo, -1, 9, 99, 999 = “desconocido/no aplica”. 
- Estados: Pendiente, Enviado, Resuelto, Rechazado, En progreso.

\section{Tabla maestra de variables}
% Descripción breve: complete una fila por variable con la mejor información disponible.
\scriptsize
\setlength{\LTleft}{-1cm}
\setlength{\LTright}{-1cm}
\setlength{\tabcolsep}{1.5pt}
\renewcommand{\arraystretch}{1.15}
\begin{longtable}{@{}L{0.095\textwidth} L{0.17\textwidth} L{0.07\textwidth} L{0.16\textwidth} L{0.14\textwidth} L{0.13\textwidth} L{0.15\textwidth}@{}}
\toprule
\textbf{Variable} & \textbf{Descripción} & \textbf{Tipo} & \textbf{Unidad / Rango o Dominio} & \textbf{Faltantes y códigos} & \textbf{Reglas / Validación} & \textbf{Dudas principales} \\
\midrule
\endfirsthead
\toprule
\textbf{Variable} & \textbf{Descripción} & \textbf{Tipo} & \textbf{Unidad / Rango o Dominio} & \textbf{Faltantes y códigos} & \textbf{Reglas / Validación} & \textbf{Dudas principales} \\
\midrule
\endhead
% ---- Variables de identificación y temporales ----
numero & Identificador único del paciente & numérico & enteros positivos & ninguno & único por paciente & Eliminada varias veces: variable redundante \\
anno & Año de registro & numérico & 2016--2025 & ninguno & -- & Eliminada varias veces: variable redundante \\
unidad & Código de unidad hospitalaria & numérico & códigos específicos & presentes & -- & Eliminada varias: variable redundante \\
fecha\_ingreso & Fecha de ingreso hospitalario & fecha & formato dd/mm/yyyy & presentes & fecha válida & \textbf{DUPLICADA}: requiere aclaración \\
fecha\_egreso & Fecha de egreso hospitalario & fecha & formato dd/mm/yyyy & presentes & >= fecha\_ingreso & \textbf{DUPLICADA}: asociada a reingresos \\
% ---- Variables demográficas ----
numero\_identidad & Número de documento de identidad & texto & -- & -- & -- & Eliminada: protección de datos \\
numero\_contacto & Número telefónico de contacto & texto & -- & -- & -- & Eliminada: protección de datos \\
nombre & Nombre del paciente & texto & -- & -- & -- & Eliminada: protección de datos \\
primer\_apellido & Primer apellido del paciente & texto & -- & -- & -- & Eliminada: protección de datos \\
segundo\_apellido & Segundo apellido del paciente & texto & -- & -- & -- & Eliminada: protección de datos \\
edad & Edad del paciente en años & numérico & 0--120 años & escasos & >0, <120 & Ninguna \\
sexo & Sexo del paciente & categórico & masculino, femenino & presentes & dominio cerrado & Requiere codificación binaria \\
color\_piel & Etnia o color de piel registrado & categórico & blanca, mestiza, negra & presentes & dominio cerrado & Requiere codificación numérica \\
peso & Peso corporal del paciente & numérico & kg, 20--200 & presentes & >0 & Ninguna \\
talla & Estatura del paciente & numérico & cm, 100--220 & presentes & >0 & Ninguna \\
imc & Índice de masa corporal calculado & numérico & kg/m², 10--60 & presentes & peso/(talla/100)² & Ninguna \\
% ---- Variables de ubicación geográfica ----
provincia & Nombre de la provincia & categórico & nombres normalizados & presentes & -- & Valores con caracteres especiales \\
municipio & Nombre del municipio & categórico & nombres normalizados & presentes & -- & Valores con caracteres especiales \\
area\_salud & Código de área de salud & categórico & códigos específicos & presentes & -- & Valores con caracteres especiales \\
idprovincia & Identificador numérico de provincia & numérico & enteros positivos & ninguno & -- & \textbf{DUPLICADA}: columna repetida \\
idmunicipio & Identificador numérico de municipio & numérico & enteros positivos & ninguno & -- & Ninguna \\
idareasalud & Identificador numérico de área de salud & numérico & enteros positivos & presentes & -- & Ninguna \\
% ---- Variables de presentación y atención inicial ----
atencion\_inicial & Tipo de atención inicial recibida & categórico & servicio, cuerpo, sala & presentes & -- & Requiere documentación del significado \\
horario\_llegada & Horario de llegada al hospital & categórico & 7am7pm, 7pm7am & presentes & -- & Requiere codificación binaria \\
ecg\_previo & Electrocardiograma previo realizado & booleano & si, no & presentes & -- & Requiere codificación binaria \\
ecg & Código de hallazgo electrocardiográfico & numérico & enteros (5--35) & presentes & -- & Requiere tabla de correspondencia \\
llamada\_emergencias & Llamada al servicio de emergencias & categórico & si, no & presentes & -- & Requiere codificación binaria \\
tiempo\_respuesta & Tiempo de respuesta de emergencias & numérico & probablemente minutos & presentes (solo si llamada=si) & >=0 & Requiere confirmación de unidad \\
tiempo\_llegada & Tiempo de llegada al hospital & numérico & probablemente minutos & presentes (solo si llamada=si) & >=0 & Requiere confirmación de unidad \\
primera\_asistencia\_medica & Tiempo hasta primera asistencia & numérico & probablemente minutos & presentes & >=0 & Requiere confirmación de unidad \\
% ---- Variables de diagnóstico ----
scacest & Síndrome coronario agudo con elevación ST & booleano & 0, 1 & presentes & -- & Requiere explicación clínica detallada \\
scacest\_secundario & SCACEST secundario & booleano & 0, 1 & presentes & -- & Requiere explicación clínica detallada \\
angina & Tipo de angina presentada & categórico & inestable, (otros) & presentes & -- & Requiere dominio completo de valores \\
angina24h & Angina en las últimas 24 horas & booleano & 0, 1 & presentes & -- & Requiere aclaración del período exacto \\
angina\_inestable & Presencia de angina inestable & booleano & 0, 1 & muchos & -- & Posible variable de seguimiento \\
presentacion & Forma de presentación clínica & categórico & dolor\_tipico, dolor\_atipico, sincope, asintomatico, otros & presentes & -- & Ninguna \\
depresion\_st & Depresión del segmento ST & booleano & si, no & presentes & -- & Requiere explicación clínica y codificación \\
depresion\_ondat & Depresión de onda T & categórico & (valores diversos) & presentes & -- & Requiere explicación clínica completa \\
supradesnivel & Supradesnivel del segmento ST & numérico & enteros & muchos & -- & Requiere explicación clínica y unidad \\
infradesnivel & Infradesnivel del segmento ST & numérico & enteros & muchos & -- & Requiere explicación clínica y unidad \\
% ---- Derivaciones electrocardiográficas (V1-V9, D1-D3, AVL, AVF, AVR, V3R, V4R) ----
v1 & Derivación precordial V1 & booleano & 0, 1 & muchos & -- & Requiere explicación: ¿alteración presente? \\
v2 & Derivación precordial V2 & booleano & 0, 1 & muchos & -- & Requiere explicación: ¿alteración presente? \\
v3 & Derivación precordial V3 & booleano & 0, 1 & muchos & -- & Requiere explicación: ¿alteración presente? \\
v4 & Derivación precordial V4 & booleano & 0, 1 & muchos & -- & Requiere explicación: ¿alteración presente? \\
v5 & Derivación precordial V5 & booleano & 0, 1 & muchos & -- & Requiere explicación: ¿alteración presente? \\
v6 & Derivación precordial V6 & booleano & 0, 1 & muchos & -- & Requiere explicación: ¿alteración presente? \\
v7 & Derivación precordial V7 & booleano & 0, 1 & muchos & -- & Requiere explicación: ¿alteración presente? \\
v8 & Derivación precordial V8 & booleano & 0, 1 & muchos & -- & Requiere explicación: ¿alteración presente? \\
v9 & Derivación precordial V9 & booleano & 0, 1 & muchos & -- & Requiere explicación: ¿alteración presente? \\
d1 & Derivación de miembro D1 & booleano & 0, 1 & muchos & -- & Requiere explicación: ¿alteración presente? \\
d2 & Derivación de miembro D2 & booleano & 0, 1 & muchos & -- & Requiere explicación: ¿alteración presente? \\
d3 & Derivación de miembro D3 & booleano & 0, 1 & muchos & -- & Requiere explicación: ¿alteración presente? \\
avl & Derivación aumentada AVL & booleano & 0, 1 & muchos & -- & Requiere explicación: ¿alteración presente? \\
avf & Derivación aumentada AVF & booleano & 0, 1 & muchos & -- & Requiere explicación: ¿alteración presente? \\
avr & Derivación aumentada AVR & booleano & 0, 1 & muchos & -- & Requiere explicación: ¿alteración presente? \\
v3r & Derivación precordial derecha V3R & booleano & 0, 1 & muchos & -- & Requiere explicación: ¿alteración presente? \\
v4r & Derivación precordial derecha V4R & booleano & 0, 1 & muchos & -- & Requiere explicación: ¿alteración presente? \\
% ---- Variables de signos vitales y estado hemodinámico ----
presion\_arterial\_sistolica & Presión arterial sistólica & numérico & mmHg, 60--250 & escasos & >0 & \textbf{DUPLICADA}: requiere aclaración \\
presion\_arterial\_diastolica & Presión arterial diastólica & numérico & mmHg, 40--150 & escasos & >0 & \textbf{DUPLICADA}: requiere aclaración \\
frecuencia\_cardiaca & Frecuencia cardíaca & numérico & latidos/min, 30--200 & escasos & >0 & Ninguna \\
shock & Presencia de shock & booleano & 0, 1 & presentes & -- & Requiere tipo de shock (cardiogénico, etc.) \\
indice\_mkillip & Clasificación Killip modificada & categórico & I, II, III, IV (romanos) & presentes & -- & Requiere explicación de diferencia con Killip \\
indice\_killip & Clasificación Killip & categórico & I, II, III, IV (romanos) & presentes & -- & Requiere explicación de diferencia con Killip-M \\
ingresos\_anteriores & Ingresos previos por IAM & booleano & 0, 1 & presentes & -- & Requiere codificación binaria \\
% ---- Antecedentes patológicos ----
diabetes\_mellitus & Diabetes mellitus previa & booleano & 0, 1 & ninguno & -- & Ninguna \\
insulina & Tratamiento con insulina & numérico & (valores diversos) & muchos & -- & Requiere aclaración: ¿dosis, tipo, booleano? \\
insuficiencia\_cardiaca\_congestiva & Insuficiencia cardíaca congestiva previa & booleano & 0, 1 & ninguno & -- & Ninguna \\
insuficiencia\_cardiaca & Insuficiencia cardíaca (evento) & booleano & 0, 1 & muchos & -- & Diferencia con ICC como antecedente \\
hipertension\_arterial & Hipertensión arterial previa & booleano & 0, 1 & ninguno & -- & Ninguna \\
hiperlipoproteinemia & Dislipidemia previa & booleano & 0, 1 & ninguno & -- & Ninguna \\
enfermedad\_arterias\_coronarias & Enfermedad arterial coronaria previa & booleano & 0, 1 & ninguno & -- & Ninguna \\
infarto\_miocardio\_agudo & Infarto de miocardio previo & booleano & 0, 1 & ninguno & -- & Ninguna \\
fibrilacion\_auricular & Fibrilación auricular previa & booleano & 0, 1 & ninguno & -- & Ninguna \\
intervencion\_coronaria\_percutanea & ICP previa & booleano & 0, 1 & ninguno & -- & Ninguna \\
cabg & Cirugía de revascularización coronaria previa (CABG) & booleano & 0, 1 & ninguno & -- & Requiere aclaración del acrónimo CABG \\
enfermedad\_venosa\_periferica & Enfermedad venosa periférica & booleano & 0, 1 & ninguno & -- & Ninguna \\
insuficiencia\_renal\_cronica & Insuficiencia renal crónica & booleano & 0, 1 & ninguno & -- & Ninguna \\
dialisis & Tratamiento con diálisis & booleano & 0, 1 & ninguno & -- & Ninguna \\
enfermedad\_cerebro\_vascular & Enfermedad cerebrovascular previa & booleano & 0, 1 & ninguno & -- & Ninguna \\
anemia & Anemia previa o actual & booleano & 0, 1 & ninguno & -- & Ninguna \\
epoc & Enfermedad pulmonar obstructiva crónica & booleano & 0, 1 & ninguno & -- & Ninguna \\
otros & Otros antecedentes & booleano & 0, 1 & presentes & -- & Requiere especificación de contenido \\
otras & Otras complicaciones o condiciones & booleano & 0, 1 & muchos & -- & Requiere especificación de contenido \\
% ---- Hábito tabáquico ----
tabaquismo & Antecedente de tabaquismo & booleano & 0, 1 & ninguno & -- & Ninguna \\
tipo\_tabaquismo & Tipo de tabaquismo & categórico & activo, noactivo & presentes & -- & Requiere codificación binaria \\
annos\_fumando & Años de consumo de tabaco & numérico & años, 0--80 & presentes & >=0 & Ninguna \\
annos\_sin\_fumar & Años sin fumar (ex-fumadores) & numérico & años, 0--80 & muchos & >=0 & Ninguna \\
% ---- Medicamentos y tratamientos ----
asa & Ácido acetilsalicílico (aspirina) & numérico & (valores diversos) & presentes & -- & \textbf{DUPLICADA}: ¿dosis o booleano? \\
betabloqueadores & Betabloqueadores & numérico & (valores diversos) & presentes & -- & \textbf{DUPLICADA}: requiere escala o codificación \\
clopidogrel & Clopidogrel & categórico & si, no & presentes & -- & \textbf{DUPLICADA}: requiere codificación binaria \\
heparina & Heparina & categórico & si, no & presentes & -- & Requiere codificación binaria \\
estatinas & Estatinas (hipolipemiantes) & categórico & si, no & presentes & -- & \textbf{DUPLICADA}: requiere codificación binaria \\
furosemida & Furosemida (diurético de asa) & categórico & si, no & presentes & -- & \textbf{DUPLICADA}: requiere codificación binaria \\
nitratos & Nitratos & categórico & si, no & presentes & -- & \textbf{DUPLICADA}: requiere codificación binaria \\
anticoagulantes & Anticoagulantes & categórico & si, no & presentes & -- & \textbf{DUPLICADA}: requiere codificación binaria \\
anticalcicos & Antagonistas del calcio & categórico & si, no & presentes & -- & Requiere codificación binaria \\
anticalcico & Antagonista del calcio (¿duplicado?) & booleano & 0, 1 & presentes & -- & Posible duplicación de anticalcicos \\
ieca & Inhibidores de ECA & categórico & si, no & presentes & -- & \textbf{DUPLICADA}: requiere codificación binaria \\
otros\_diureticos & Otros diuréticos & categórico & si, no & presentes & -- & \textbf{DUPLICADA}: requiere codificación binaria \\
% ---- Variables de reperfusión y procedimientos ----
estreptoquinasa\_recombinante & Administración de estreptoquinasa & categórico & si, no & presentes & -- & Requiere codificación binaria \\
tiempo\_puerta\_aguja & Tiempo puerta-aguja & numérico & probablemente minutos & presentes & >=0 & Requiere confirmación de unidad y definición \\
tiempo\_isquemia & Tiempo de isquemia & numérico & probablemente minutos & presentes & >=0 & Requiere confirmación de unidad \\
escala\_grace & Puntaje de riesgo GRACE & numérico & 0--372 & presentes & -- & Requiere validación de rango y significado \\
reperfusion & Tipo de reperfusión realizada & categórico & no, parcial, total, otro & presentes & -- & Requiere codificación y explicación detallada \\
coronariografia & Coronariografía realizada & categórico & no, si, otro, centro & presentes & -- & Requiere codificación y explicación \\
lugar\_trombolisis & Lugar de trombolisis & categórico & ucie, sala, servicio, otro & presentes & -- & Candidata a eliminación según análisis \\
ergometria & Prueba de esfuerzo (ergometría) & categórico & positiva, negativa, no & muchos & -- & Requiere dominio completo y codificación \\
% ---- Variables de intervención hemodinámica y soporte ----
avc & Asistencia ventricular o complicación & booleano & 0, 1 & ninguno & -- & Requiere aclaración del acrónimo \\
mpt & Marcapaso temporal & booleano & 0, 1 & ninguno & -- & Requiere confirmación del acrónimo \\
vam & Ventilación asistida mecánica & booleano & 0, 1 & ninguno & -- & Requiere confirmación del acrónimo \\
mpp & Marcapaso permanente & booleano & 0, 1 & ninguno & -- & Requiere confirmación del acrónimo \\
aminas & Uso de aminas vasoactivas & booleano & 0, 1 & ninguno & -- & Ninguna \\
balon & Balón de contrapulsación intraaórtico & booleano & 0, 1 & ninguno & -- & Ninguna \\
% ---- Variables de seguimiento y desenlaces ----
proxima & Próxima consulta o evento & numérico & enteros & muchos & -- & Requiere explicación del significado \\
fecha\_consulta & Fecha de consulta de seguimiento & fecha & formato dd/mm/yyyy & muchos & fecha >= fecha\_egreso & Ninguna \\
resultado & Resultado del seguimiento & categórico & vivo\_sin, vivo\_con, noevaluado, alta, fallecido & muchos & -- & Requiere explicación de categorías \\
fecha\_defuncion & Fecha de defunción & fecha & formato dd/mm/yyyy & muchos & >= fecha\_ingreso & Ninguna \\
motivo & Motivo de reingreso & categórico & scacest (principalmente) & muchos & -- & Asociada a fecha\_ingreso duplicada \\
% ---- Variables de coronariografía y calidad de atención ----
fecha\_realizacion & Fecha de realización de procedimiento & fecha & formato dd/mm/yyyy & muchos & -- & Requiere especificación de procedimiento \\
razones\_documentadas & Razones documentadas & booleano & 0, 1 & muchos & -- & Requiere contexto: ¿razones de qué? \\
riesgo\_beneficio & Evaluación riesgo-beneficio & booleano & 0, 1 & muchos & -- & Requiere contexto específico \\
anti\_agregacion\_plaquetaria & Antiagregación plaquetaria & booleano & 0, 1 & muchos & -- & Requiere aclaración del contexto \\
proteccion\_embolica & Protección embólica & booleano & 0, 1 & muchos & -- & Requiere aclaración del contexto \\
funcion\_renal & Evaluación de función renal & booleano & 0, 1 & muchos & -- & Requiere aclaración del contexto \\
volumen\_contraste & Volumen de contraste utilizado & booleano & 0, 1 & muchos & -- & Requiere aclaración: ¿adecuado/inadecuado? \\
prescripcion\_optima & Prescripción óptima al egreso & booleano & 0, 1 & muchos & -- & Requiere definición de "óptima" \\
rehabilitacion & Rehabilitación cardíaca & booleano & 0, 1 & muchos & -- & \textbf{DUPLICADA}: aparece dos veces \\
participacion\_registro & Participación en registro & booleano & 0, 1 & muchos & -- & Requiere aclaración del propósito \\
coronariografias\_medico & Coronariografías por médico & booleano & 0, 1 & muchos & -- & Requiere aclaración del significado \\
coronariografias\_centro & Coronariografías por centro & booleano & 0, 1 & muchos & -- & Requiere aclaración del significado \\
% ---- Resultados de coronariografía ----
idresultado & Identificador de resultado & numérico & enteros & muchos & -- & Requiere explicación del sistema de ID \\
arteria & Arteria afectada & categórico & cd, cx, ada, otros & muchos & -- & Requiere tabla completa de códigos \\
localizacion & Localización de lesión & categórico & proximal, media, distal, sin\_lesiones & muchos & -- & Ninguna \\
estenosis & Grado de estenosis & numérico & porcentaje, 0--100 & muchos & 0--100 & Ninguna \\
abordaje & Tipo de abordaje & categórico & stent\_farmaco, stent\_metalico, ninguno & muchos & -- & Requiere tabla completa de valores \\
% ---- Variables de egreso y estadía ----
fecha\_egreso & Fecha de egreso hospitalario & fecha & formato dd/mm/yyyy & presentes & >= fecha\_ingreso & \textbf{DUPLICADA}: requiere aclaración \\
estado\_vital & Estado vital al egreso & categórico & vivo, fallecido & presentes & -- & Requiere codificación binaria \\
otra\_institucion & Traslado a otra institución & categórico & si, no & muchos & -- & Requiere codificación binaria \\
estadia\_ucie & Estadía en UCIE & numérico & días & presentes & >=0 & Ninguna (unidad confirmada: días) \\
estadia\_uci & Estadía en UCI & numérico & días & presentes & >=0 & Ninguna (unidad confirmada: días) \\
estadia\_intrahospitalaria & Estadía intrahospitalaria total & numérico & días & presentes & >=0 & Ninguna (unidad confirmada: días) \\
% ---- Consejerías y educación al paciente ----
consejeria\_antitabaquica & Consejería antitabáquica brindada & booleano & 0, 1 & presentes & -- & Ninguna \\
consejeria & Consejería general brindada & booleano & 0, 1 & presentes & -- & Requiere especificación del tipo \\
dieta & Consejería dietética brindada & booleano & 0, 1 & presentes & -- & Ninguna \\
% ---- Complicaciones y observaciones ----
complicaciones & Complicaciones durante hospitalización & categórico & valores diversos & muchos & -- & Requiere tabla de códigos completa \\
observaciones & Observaciones clínicas libres & texto libre & texto único por caso & muchos & -- & Ninguna \\
% ---- Variables de laboratorio ----
colesterol & Colesterol total & numérico & mg/dL o mmol/L & muchos & >0 & Requiere confirmación de unidad \\
creatinina & Creatinina sérica & numérico & mg/dL o $\mu$mol/L & muchos & >0 & Requiere confirmación de unidad \\
filtrado\_glomerular & Tasa de filtrado glomerular estimado & numérico & mL/min/1.73m² & muchos & >0 & Requiere confirmación de unidad \\
trigliceridos & Triglicéridos & numérico & mg/dL o mmol/L & muchos & >0 & Requiere confirmación de unidad \\
glicemia & Glucemia & numérico & mg/dL o mmol/L & muchos & >0 & Requiere confirmación de unidad \\
leuco & Leucocitos & numérico & células/mm³ o x10³/$\mu$L & muchos & >0 & Requiere confirmación de unidad \\
hb & Hemoglobina & numérico & g/dL & muchos & >0 & Requiere confirmación de unidad \\
ck & Creatina quinasa & numérico & U/L & muchos & >0 & Requiere confirmación de unidad \\
ckmb & Creatina quinasa MB & numérico & U/L o ng/mL & muchos & >0 & Requiere confirmación de unidad \\
% ---- Variables de arterias coronarias (acrónimos) ----
acd & Arteria coronaria derecha & booleano & 0, 1 & presentes & -- & Requiere confirmación del acrónimo \\
ada & Arteria descendente anterior & booleano & 0, 1 & presentes & -- & Requiere confirmación del acrónimo \\
acx & Arteria circunfleja & booleano & 0, 1 & presentes & -- & Requiere confirmación del acrónimo \\
% ---- Variables ecocardiográficas ----
fraccion\_eyeccion & Fracción de eyección del VI & numérico & porcentaje, 0--100 & muchos & 0--100 & Ninguna \\
ddvi & Diámetro diastólico del VI & numérico & mm & muchos & >0 & Requiere confirmación de unidad \\
dsvi & Diámetro sistólico del VI & numérico & mm & muchos & >0 & Requiere confirmación de unidad \\
tiv & Grosor del tabique interventricular & numérico & mm & muchos & >0 & Requiere confirmación de unidad \\
pp & Grosor de pared posterior & numérico & mm & muchos & >0 & Requiere confirmación de unidad \\
ud & Parámetro ecocardiográfico (desconocido) & numérico & (unidad desconocida) & muchos & >0 & Requiere aclaración completa \\
fs & Fracción de acortamiento & numérico & porcentaje, 0--100 & muchos & 0--100 & Requiere confirmación \\
tapse & TAPSE (función ventricular derecha) & numérico & mm & muchos & >0 & Requiere confirmación de unidad \\
ea & Relación E/A mitral (función diastólica) & numérico & ratio & muchos & >0 & Requiere confirmación \\
ee & Relación E/e' (función diastólica) & numérico & ratio & muchos & >0 & Requiere confirmación \\
pat & Parámetro ecocardiográfico (desconocido) & numérico & (unidad desconocida) & muchos & >0 & Requiere aclaración completa \\
% ---- Variables valvulares ecocardiográficas ----
insao & Insuficiencia aórtica & booleano & 0, 1 & muchos & -- & Requiere aclaración del acrónimo \\
estao & Estenosis aórtica & booleano & 0, 1 & muchos & -- & Requiere aclaración del acrónimo \\
insmit & Insuficiencia mitral & booleano & 0, 1 & muchos & -- & Requiere aclaración del acrónimo \\
estmit & Estenosis mitral & booleano & 0, 1 & muchos & -- & Requiere aclaración del acrónimo \\
% ---- Variable desenlace principal ----
mortality\_inhospital & Mortalidad intrahospitalaria & booleano & 0, 1 & ninguno & -- & \textbf{Variable objetivo del modelo} \\
\bottomrule
\endfoot

% ------------------------------------------
\end{longtable}
\renewcommand{\arraystretch}{1.2}
\normalsize

\section{Registro de dudas y resoluciones}
% Use una fila por duda específica. Mantenga el "Estado" actualizado.
\scriptsize
\setlength{\LTleft}{-1cm}
\setlength{\LTright}{-1cm}
\setlength{\tabcolsep}{1.5pt}
\renewcommand{\arraystretch}{1.15}
\begin{longtable}{@{}L{0.035\textwidth} L{0.10\textwidth} L{0.265\textwidth} L{0.17\textwidth} L{0.05\textwidth} L{0.05\textwidth} L{0.095\textwidth} L{0.065\textwidth} L{0.075\textwidth}@{}}
\toprule
\textbf{ID} & \textbf{Variable} & \textbf{Duda / Observación} & \textbf{Evidencia / Contexto} & \textbf{Impacto} & \textbf{Prioridad} & \textbf{Responsable (consulta a)} & \textbf{Estado} & \textbf{Fechas (sol./cierre)} \\
\midrule
\endfirsthead
\toprule
\textbf{ID} & \textbf{Variable} & \textbf{Duda / Observación} & \textbf{Evidencia / Contexto} & \textbf{Impacto} & \textbf{Prioridad} & \textbf{Responsable (consulta a)} & \textbf{Estado} & \textbf{Fechas (sol./cierre)} \\
\midrule
\endhead
\bottomrule
\endfoot
% ---- Dudas prioritarias sobre el dataset ----
001 & Variables duplicadas (grupo) & Se identifican 12 variables que aparecen duplicadas en el dataset. Se requiere aclaración sobre el significado de cada columna duplicada y cuál utilizar para el análisis. & Variables afectadas: presion\_arterial\_sistolica, presion\_arterial\_diastolica, asa, betabloqueadores, ieca, estatinas, clopidogrel, furosemida, nitratos, anticoagulantes, otros\_diureticos, fecha\_ingreso. Posibles hipótesis: (a) medición en momentos diferentes (ingreso vs egreso), (b) fusión incorrecta de datasets, (c) prescripción vs administración real & Alto & Alta & Cardiólogo & Pendiente & 2025-10-25 / \\
002 & asa & ¿La variable ASA representa dosis en mg, días de tratamiento, o simplemente presencia/ausencia? & Valores observados son numéricos diversos (rango 0--24). No se identifica patrón claro de dosificación estándar & Alto & Alta & Cardiólogo & Pendiente & 2025-10-25 / \\
003 & betabloqueadores & ¿Qué escala o codificación se utiliza para betabloqueadores? & Valores numéricos diversos (rango 0--36). No se identifica si es dosis, días, o código categórico & Alto & Alta & Cardiólogo & Pendiente & 2025-10-25 / \\
004 & ecg & Solicitar tabla de correspondencia completa para códigos ECG & Valores observados: 5, 10, 12, 15, 20, 25, 30, 35. Sin documentación disponible sobre significado clínico & Alto & Alta & Cardiólogo & Pendiente & 2025-10-25 / \\
005 & scacest, scacest\_secundario & Requiere explicación clínica detallada de la diferencia entre SCACEST primario y secundario & Ambas variables booleanas (0/1), pero no se comprende la distinción clínica ni criterios diagnósticos & Alto & Alta & Cardiólogo & Pendiente & 2025-10-25 / \\
006 & indice\_killip, indice\_mkillip & ¿Cuál es la diferencia entre clasificación Killip y Killip modificada? ¿Cuándo se utiliza cada una? & Ambas utilizan números romanos I--IV. En algunos registros difieren, en otros coinciden & Medio & Alta & Cardiólogo & Pendiente & 2025-10-25 / \\
007 & tiempo\_puerta\_aguja & Confirmar definición clínica exacta y unidad de medida & Asumimos minutos desde llegada hospitalaria hasta inicio de trombolisis, pero requiere confirmación oficial & Alto & Alta & Cardiólogo & Pendiente & 2025-10-25 / \\
008 & tiempo\_isquemia & Confirmar definición clínica exacta y unidad de medida & Asumimos minutos desde inicio de síntomas hasta reperfusión, pero requiere confirmación oficial & Alto & Alta & Cardiólogo & Pendiente & 2025-10-25 / \\
009 & tiempo\_respuesta, tiempo\_llegada & Confirmar unidades de medida y definiciones exactas de cada variable & Valores presentes solo cuando llamada\_emergencias=si. Asumimos minutos pero sin documentación & Medio & Media & Cardiólogo & Pendiente & 2025-10-25 / \\
010 & escala\_grace & Validar rango de valores observados contra rango teórico esperado (0--372) & Valores parecen consistentes pero requiere validación clínica del cálculo & Medio & Media & Cardiólogo & Pendiente & 2025-10-25 / \\
011 & Derivaciones ECG (V1--V9, D1--D3, AVL, AVF, AVR, V3R, V4R) & ¿Estas variables indican presencia de alteración en cada derivación? ¿Qué tipo de alteración (supradesnivel, infradesnivel, onda Q)? & Variables booleanas (0/1) con muchos faltantes. Requiere documentación de criterios diagnósticos utilizados & Alto & Alta & Cardiólogo & Pendiente & 2025-10-25 / \\
012 & avc, mpt, vam, mpp & Confirmar significado exacto de acrónimos & Asumimos: AVC=asistencia ventricular, MPT=marcapaso temporal, VAM=ventilación mecánica, MPP=marcapaso permanente & Medio & Media & Cardiólogo & Pendiente & 2025-10-25 / \\
013 & cabg & Confirmar acrónimo CABG (¿Coronary Artery Bypass Graft?) & Asumimos cirugía de revascularización coronaria pero requiere confirmación & Bajo & Baja & Cardiólogo & Pendiente & 2025-10-25 / \\
014 & reperfusion & Explicar diferencia entre categorías y criterios de clasificación & Valores: no, parcial, total, otro. Requiere definición de criterios clínicos utilizados & Alto & Alta & Cardiólogo & Pendiente & 2025-10-25 / \\
015 & coronariografia & Explicar diferencia entre valores observados & Valores: no, si, otro, centro. Especialmente aclarar significado de "centro" vs "otro" & Medio & Media & Cardiólogo & Pendiente & 2025-10-25 / \\
016 & Grupo variables de calidad & Grupo de variables con muchos faltantes que requieren contexto clínico & Variables: razones\_documentadas, riesgo\_beneficio, anti\_agregacion\_plaquetaria, proteccion\_embolica, funcion\_renal, volumen\_contraste, prescripcion\_optima. ¿Son indicadores de calidad o checklist? & Medio & Media & Cardiólogo & Pendiente & 2025-10-25 / \\
017 & Laboratorio (grupo) & Confirmar unidades de medida para todas las variables de laboratorio & Variables: colesterol, creatinina, filtrado\_glomerular, trigliceridos, glicemia, leuco, hb, ck, ckmb. Especificar unidades (mg/dL, mmol/L, etc.) & Alto & Alta & Laboratorio & Pendiente & 2025-10-25 / \\
018 & Ecocardiografía (grupo) & Confirmar significado de acrónimos: ud, pat, insao, estao, insmit, estmit & Asumimos insuficiencias y estenosis valvulares. ud y pat sin identificar & Medio & Media & Cardiología & Pendiente & 2025-10-25 / \\
019 & arteria & Solicitar tabla completa de códigos de arterias coronarias & Códigos: cd, cx, ada. Asumimos: CD=coronaria derecha, CX=circunfleja, ADA=descendente anterior & Medio & Media & Cardiólogo & Pendiente & 2025-10-25 / \\
020 & abordaje & Solicitar tabla completa de tipos de abordaje/intervención coronaria & Valores: stent\_farmaco, stent\_metalico, ninguno. ¿Existen otros valores posibles? & Medio & Media & Cardiólogo & Pendiente & 2025-10-25 / \\
021 & resultado & Explicar categorías de seguimiento y su significado clínico & Valores: vivo\_sin, vivo\_con, noevaluado, alta, fallecido. Requiere explicación de "vivo\_sin" vs "vivo\_con" (¿con/sin complicaciones?) & Medio & Media & Cardiólogo & Pendiente & 2025-10-25 / \\
022 & angina24h & ¿Se refiere a angina en 24h previas al ingreso o durante primeras 24h de hospitalización? & Variable booleana con contexto temporal ambiguo que afecta interpretación clínica & Bajo & Baja & Cardiólogo & Pendiente & 2025-10-25 / \\
023 & provincia, municipio, area\_salud & Normalizar caracteres especiales en nombres geográficos & Caracteres especiales mal codificados: Sancti Sp\'iritus, Cabaigu\'an, Camag\"uey, Manat\'i, G\"uines & Bajo & Baja & Admin. datos & Pendiente & 2025-10-25 / \\
024 & insulina & ¿Variable representa dosis, tipo de insulina, duración de tratamiento, o uso binario? & Valores numéricos diversos sin patrón identificable. Dificulta su uso en modelado & Medio & Media & Cardiólogo & Pendiente & 2025-10-25 / \\
025 & lugar\_trombolisis & ¿Es necesaria esta variable para el objetivo analítico del proyecto? & Valores: ucie, sala, servicio. Considerada candidata a eliminación según relevancia clínica & Bajo & Baja & Equipo ML & Pendiente & 2025-10-25 / \\
\end{longtable}
\renewcommand{\arraystretch}{1.2}
\normalsize


\section*{Acciones siguientes y próximos pasos}

\subsection*{Consultas prioritarias }
\begin{enumerate}
    \item \textbf{Duplicación de variables (Alta prioridad):} Aclarar el significado y uso correcto de las 12 variables que aparecen duplicadas en el dataset. Esta es la duda más crítica que afecta la calidad del análisis.
    \item \textbf{Escalas y codificaciones de medicamentos:} Definir las escalas utilizadas para asa, betabloqueadores, insulina y confirmar codificaciones de otros medicamentos.
    \item \textbf{Variables electrocardiográficas:} Proporcionar tabla de correspondencia para códigos ECG y documentar criterios para derivaciones (V1--V9, D1--D3, AVL, AVF, AVR, V3R, V4R).
    \item \textbf{Variables de tiempo:} Confirmar definiciones clínicas exactas y unidades de medida para tiempo\_puerta\_aguja, tiempo\_isquemia, tiempo\_respuesta, tiempo\_llegada.
    \item \textbf{Clasificaciones clínicas:} Explicar diferencias entre SCACEST primario/secundario, Killip/Killip modificado, y categorías de reperfusión.
    \item \textbf{Acrónimos y términos técnicos:} Confirmar significado de acrónimos no documentados (avc, mpt, vam, mpp, cabg, acd, ada, acx, insao, estao, insmit, estmit, ud, pat).
    \item \textbf{Tablas de códigos:} Solicitar tablas completas para arteria, abordaje, resultado, complicaciones.
    \item Especificar unidades de medida para todas las variables de laboratorio (colesterol, creatinina, filtrado glomerular, triglicéridos, glicemia, leucocitos, hemoglobina, CK, CK-MB).
    \item Confirmar rangos de referencia y límites de detección de los equipos utilizados.
\end{enumerate}



\subsection*{Tareas del equipo de machine learning}
\begin{enumerate}
    \item Mantener actualizado el "Registro de dudas y resoluciones" con estados y fechas.
    \item Documentar todas las decisiones de preprocesamiento tomadas en ausencia de aclaraciones.
    \item Priorizar variables según impacto en el objetivo analítico (predicción de mortalidad intrahospitalaria).
    \item Preparar pipeline de preprocesamiento flexible que permita incorporar aclaraciones posteriores.
    \item Normalizar caracteres especiales en nombres geográficos.
\end{enumerate}


\subsection*{Nota sobre el proceso de limpieza inicial}
Este dataset es resultado de una primera etapa de limpieza en la que se eliminaron:
\begin{itemize}
    \item Variables de identificación personal (nombre, primer\_apellido, segundo\_apellido, numero\_identidad, numero\_contacto) para cumplir con protección de datos.
    \item Variables redundantes identificadas en análisis preliminar (anno, numero como identificador, unidad).
\end{itemize}

El dataset original parece ser resultado de la fusión de múltiples fuentes o registros hospitalarios, lo cual explicaría algunas de las duplicaciones observadas. Se requiere confirmación de esta hipótesis por parte del equipo responsable de la recolección de datos.

\end{document}