\documentclass[11pt,a4paper]{article}

% Configuración básica
\usepackage[margin=1.8cm,bottom=2.5cm]{geometry}
\usepackage[spanish,es-nodecimaldot]{babel}
\usepackage[utf8]{inputenc}
\usepackage[T1]{fontenc}
\usepackage{lmodern}
\usepackage{csquotes}
\usepackage{microtype} % Mejora el espaciado y reduce overfull
\usepackage{hyperref}
\hypersetup{
    colorlinks=true,
    linkcolor=black,
    urlcolor=blue
}

% Tablas
\usepackage{array}
\usepackage{booktabs}
\usepackage{longtable}
\usepackage{multirow}
\usepackage{xcolor}
\usepackage{tabularx}
\usepackage{ragged2e}
\renewcommand{\arraystretch}{1.2}
\usepackage{ragged2e}   % \RaggedRight
\usepackage{makecell}   % celdas con saltos de línea: \makecell{...\\...}
\usepackage{caption}
\usepackage{underscore} % permite '_ ' en texto sin escapar
\usepackage{seqsplit}

% Ajustes para mejorar el espaciado y evitar overfull
\setlength{\emergencystretch}{4em}
\setlength{\hfuzz}{2pt}
\tolerance=5000
\hbadness=5000
\sloppy


% Utilidad para celdas de párrafo alineadas a la izquierda
\newcolumntype{L}[1]{>{\RaggedRight\arraybackslash}p{#1}}
% Columna que ajusta automáticamente el ancho
\newcolumntype{Y}{>{\RaggedRight\arraybackslash}X}

% Metadatos personalizables
\newcommand{\DatasetNombre}{RECUIMA (Registro Cubano de Infarto Agudo de Miocardio)}
\newcommand{\DatasetVersion}{v0.2 (post-limpieza inicial)}
\newcommand{\Proyecto}{Predicción de Mortalidad en Infarto Agudo de Miocardio}
\newcommand{\Responsable}{Equipo ML}

\newcommand{\us}{\_\allowbreak}
% Ajustes para longtable
\setlength{\LTpre}{6pt}
\setlength{\LTpost}{6pt}
\setlength{\tabcolsep}{4pt} % espacio entre columnas
\renewcommand{\arraystretch}{1.15}

\title{Planilla de dudas sobre variables del dataset}
\author{\Responsable}
\date{\today}

\begin{document}
\maketitle

\section*{Resumen del dataset}
\small
\setlength{\tabcolsep}{3pt}
\setlength{\arrayrulewidth}{0.6pt}
\begin{tabular}{|L{0.28\textwidth}|L{0.68\textwidth}|}
\hline
Nombre del dataset & \DatasetNombre \\ \hline
Versión & \DatasetVersion \\ \hline
Fuente / Sistema origen & Registro hospitalario de pacientes con infarto agudo de miocardio \\ \hline
Periodo de cobertura & 2016--2025 \\ \hline
Población y unidad de análisis & Pacientes ingresados con diagnóstico de IAM;unidad: episodio de internación \\ \hline
Número de registros / variables & 3,112 registros / 185 variables (después de limpieza inicial) \\ \hline
Fecha de extracción & 02/04/2025 \\ \hline
Objetivo analítico & Predicción de mortalidad intrahospitalaria en pacientes con IAM mediante modelos de machine learning \\ \hline
Restricciones legales / privacidad & Se eliminaron variables de identificación personal (nombres, números de identidad, números de contacto) para cumplir con protección de datos \\ \hline
Notas generales de calidad (duplicados, faltantes, codificaciones especiales) & Se observa presencia significativa de valores faltantes en múltiples variables. Existen duplicaciones de variables clave que requieren aclaración. El dataset parece ser resultado de la fusión de múltiples registros o fuentes. Se eliminaron variables redundantes identificadas en la limpieza inicial: \texttt{anno}, \texttt{numero} (identificador), \texttt{unidad}. \\ \hline
\end{tabular}

\vspace{1em}
\noindent\textbf{Nota importante sobre duplicaciones:} Se han identificado las siguientes variables que aparecen duplicadas en el dataset: \texttt{presion\_arterial\_sistolica}, \texttt{presion\_arterial\_diastolica}, \texttt{asa}, \texttt{betabloqueadores}, \texttt{ieca}, \texttt{estatinas}, \texttt{clopidogrel}, \texttt{furosemida}, \texttt{nitratos}, \texttt{anticoagulantes}, \texttt{otros\_diureticos}, \texttt{fecha\_egreso} y \texttt{fecha\_ingreso}. Se desconoce el motivo exacto de estas duplicaciones, aunque se presume que puede deberse a: (1) registro en diferentes momentos temporales (ingreso vs. egreso), (2) fusión de múltiples fuentes de datos, o (3) diferencias entre prescripción y administración real. Se requiere aclaración urgente sobre la interpretación correcta de estas columnas duplicadas.

\section*{Guía rápida}
- Tipo: numérico, categórico, booleano, fecha/hora, texto libre, identificador. 
- Códigos especiales: por ejemplo, -1, 9, 99, 999 = “desconocido/no aplica”. 
- Estados: Pendiente, Enviado, Resuelto, Rechazado, En progreso.

\section{Tabla maestra de variables}
\small % puedes cambiar a \footnotesize o \scriptsize si necesitas ahorrar espacio

% Ajustar los márgenes internos de longtable (no negativos)
\setlength{\LTleft}{\fill}
\setlength{\LTright}{\fill}

% Línea más visible entre celdas
\setlength{\arrayrulewidth}{0.6pt}

\begin{longtable}{|L{0.115\textwidth}|L{0.165\textwidth}|L{0.085\textwidth}|L{0.15\textwidth}|L{0.135\textwidth}|L{0.12\textwidth}|L{0.14\textwidth}|}
\caption{Tabla maestra de variables.}\\
\hline
\textbf{Variable} & \textbf{Descripción} & \textbf{Tipo} & \textbf{Unidad / Rango o Dominio} & \textbf{Faltantes y códigos} & \textbf{Reglas / Validación} & \textbf{Dudas principales} \\ \hline
\endfirsthead

\multicolumn{7}{c}{\small\itshape Continuación de la Tabla maestra de variables (encabezado repetido)}\\[3pt]
\hline
\textbf{Variable} & \textbf{Descripción} & \textbf{Tipo} & \textbf{Unidad / Rango o Dominio} & \textbf{Faltantes y códigos} & \textbf{Reglas / Validación} & \textbf{Dudas principales} \\ \hline
\endhead

\hline
\multicolumn{7}{r}{\small\itshape Continua en la siguiente página} \\ \hline
\endfoot

\hline
\endlastfoot
numero & Identificador único del paciente & numérico & enteros positivos & ninguno & único por paciente & Eliminada varias veces: variable redundante \\ \hline
anno & Año de registro & numérico & 2016--2025 & ninguno & -- & Eliminada varias veces: variable redundante \\ \hline
unidad & Código de unidad hospitalaria & numérico & códigos específicos & presentes & -- & Eliminada varias: variable redundante \\ \hline
\makecell[l]{fecha\\ingreso} & Fecha de ingreso hospitalario & fecha & formato dd/mm/yyyy & presentes & fecha válida & \textbf{DUPLICADA}: requiere aclaración \\ \hline
\makecell[l]{fecha\\egreso} & Fecha de egreso hospitalario & fecha & formato dd/mm/yyyy & presentes & >= fecha\_ingreso & \textbf{DUPLICADA}: asociada a reingresos \\ \hline
\makecell[l]{numero\\identidad} & Número de documento de identidad & texto & -- & -- & -- & Eliminada: protección de datos \\ \hline
\makecell[l]{numero\\contacto} & Número telefónico de contacto & texto & -- & -- & -- & Eliminada: protección de datos \\ \hline
nombre & Nombre del paciente & texto & -- & -- & -- & Eliminada: protección de datos \\ \hline
\makecell[l]{primer\\apellido} & Primer apellido del paciente & texto & -- & -- & -- & Eliminada: protección de datos \\ \hline
\makecell[l]{segundo\\apellido} & Segundo apellido del paciente & texto & -- & -- & -- & Eliminada: protección de datos \\ \hline
edad & Edad del paciente en años & numérico & 0--120 años & escasos & >0, <120 & Ninguna \\ \hline
sexo & Sexo del paciente & categórico & masculino, femenino & presentes & dominio cerrado & Requiere codificación binaria \\ \hline
\makecell[l]{color\\piel} & Etnia o color de piel registrado & categórico & blanca, mestiza, negra & presentes & dominio cerrado & Requiere codificación numérica \\ \hline
peso & Peso corporal del paciente & numérico & kg, 20--200 & presentes & >0 & Ninguna \\ \hline
talla & Estatura del paciente & numérico & cm, 100--220 & presentes & >0 & Ninguna \\ \hline
imc & Índice de masa corporal calculado & numérico & kg/m², 10--60 & presentes & peso/(talla/100)² & Ninguna \\ \hline
provincia & Nombre de la provincia & categórico & nombres normalizados & presentes & -- & Valores con caracteres especiales \\ \hline
municipio & Nombre del municipio & categórico & nombres normalizados & presentes & -- & Valores con caracteres especiales \\ \hline
area\_salud & Código de área de salud & categórico & códigos específicos & presentes & -- & Valores con caracteres especiales \\ \hline
idprovincia & Identificador numérico de provincia & numérico & enteros positivos & ninguno & -- & \textbf{DUPLICADA}: columna repetida \\ \hline
idmunicipio & Identificador numérico de municipio & numérico & enteros positivos & ninguno & -- & Ninguna \\ \hline
idareasalud & Identificador numérico de área de salud & numérico & enteros positivos & presentes & -- & Ninguna \\ \hline
\makecell[l]{atencion\\inicial} & Tipo de atención inicial recibida & categórico & servicio, cuerpo, sala & presentes & -- & Requiere documentación del significado \\ \hline
\makecell[l]{horario\\llegada} & Horario de llegada al hospital & categórico & 7am7pm, 7pm7am & presentes & -- & Requiere codificación binaria \\ \hline
ecg\_previo & Electrocardiograma previo realizado & booleano & si, no & presentes & -- & Requiere codificación binaria \\ \hline
ecg & Código de hallazgo electrocardiográfico & numérico & enteros (5--35) & presentes & -- & Requiere tabla de correspondencia \\ \hline
\makecell[l]{llamada\\emergencias} & Llamada al servicio de emergencias & categórico & si, no & presentes & -- & Requiere codificación binaria \\ \hline
\makecell[l]{tiempo\\respuesta} & Tiempo de respuesta de emergencias & numérico & probablemente minutos & presentes (solo si llamada=si) & >=0 & Requiere confirmación de unidad \\ \hline
\makecell[l]{tiempo\\llegada} & Tiempo de llegada al hospital & numérico & probablemente minutos & presentes (solo si llamada=si) & >=0 & Requiere confirmación de unidad \\ \hline
\makecell[l]{primera\\asistencia\\medica} & Tiempo hasta primera asistencia & numérico & probablemente minutos & presentes & >=0 & Requiere confirmación de unidad \\ \hline
scacest & Síndrome coronario agudo con elevación ST & booleano & 0, 1 & presentes & -- & Requiere explicación clínica detallada \\ \hline
\makecell[l]{scacest\\secundario} & SCACEST secundario & booleano & 0, 1 & presentes & -- & Requiere explicación clínica detallada \\ \hline
angina & Tipo de angina presentada & categórico & inestable, (otros) & presentes & -- & Requiere dominio completo de valores \\ \hline
angina24h & Angina en las últimas 24 horas & booleano & 0, 1 & presentes & -- & Requiere aclaración del período exacto \\ \hline
\makecell[l]{angina\\inestable} & Presencia de angina inestable & booleano & 0, 1 & muchos & -- & Posible variable de seguimiento \\ \hline
\makecell[l]{presenta-\\cion} & Forma de presentación clínica & categórico & dolor\_tipico, dolor\_atipico, sincope, asintomatico, otros & presentes & -- & Ninguna \\ \hline
\makecell[l]{depresion\\st} & Depresión del segmento ST & booleano & si, no & presentes & -- & Requiere explicación clínica y codificación \\ \hline
\makecell[l]{depresion\\ondat} & Depresión de onda T & categórico & (valores diversos) & presentes & -- & Requiere explicación clínica completa \\ \hline
\makecell[l]{supradesni-\\vel} & Supradesnivel del segmento ST & numérico & enteros & muchos & -- & Requiere explicación clínica y unidad \\ \hline
\makecell[l]{infradesni-\\vel} & Infradesnivel del segmento ST & numérico & enteros & muchos & -- & Requiere explicación clínica y unidad \\ \hline
v1 & Derivación precordial V1 & booleano & 0, 1 & muchos & -- & Requiere explicación: ¿alteración presente? \\ \hline
v2 & Derivación precordial V2 & booleano & 0, 1 & muchos & -- & Requiere explicación: ¿alteración presente? \\ \hline
v3 & Derivación precordial V3 & booleano & 0, 1 & muchos & -- & Requiere explicación: ¿alteración presente? \\ \hline
v4 & Derivación precordial V4 & booleano & 0, 1 & muchos & -- & Requiere explicación: ¿alteración presente? \\ \hline
v5 & Derivación precordial V5 & booleano & 0, 1 & muchos & -- & Requiere explicación: ¿alteración presente? \\ \hline
v6 & Derivación precordial V6 & booleano & 0, 1 & muchos & -- & Requiere explicación: ¿alteración presente? \\ \hline
v7 & Derivación precordial V7 & booleano & 0, 1 & muchos & -- & Requiere explicación: ¿alteración presente? \\ \hline
v8 & Derivación precordial V8 & booleano & 0, 1 & muchos & -- & Requiere explicación: ¿alteración presente? \\ \hline
v9 & Derivación precordial V9 & booleano & 0, 1 & muchos & -- & Requiere explicación: ¿alteración presente? \\ \hline
d1 & Derivación de miembro D1 & booleano & 0, 1 & muchos & -- & Requiere explicación: ¿alteración presente? \\ \hline
d2 & Derivación de miembro D2 & booleano & 0, 1 & muchos & -- & Requiere explicación: ¿alteración presente? \\ \hline
d3 & Derivación de miembro D3 & booleano & 0, 1 & muchos & -- & Requiere explicación: ¿alteración presente? \\ \hline
avl & Derivación aumentada AVL & booleano & 0, 1 & muchos & -- & Requiere explicación: ¿alteración presente? \\ \hline
avf & Derivación aumentada AVF & booleano & 0, 1 & muchos & -- & Requiere explicación: ¿alteración presente? \\ \hline
avr & Derivación aumentada AVR & booleano & 0, 1 & muchos & -- & Requiere explicación: ¿alteración presente? \\ \hline
v3r & Derivación precordial derecha V3R & booleano & 0, 1 & muchos & -- & Requiere explicación: ¿alteración presente? \\ \hline
v4r & Derivación precordial derecha V4R & booleano & 0, 1 & muchos & -- & Requiere explicación: ¿alteración presente? \\ \hline
\makecell[l]{presion\\arterial\\sistolica} & Presión arterial sistólica & numérico & mmHg, 60--250 & escasos & >0 & \textbf{DUPLICADA}: requiere aclaración \\ \hline
\makecell[l]{presion\\arterial\\diastolica} & Presión arterial diastólica & numérico & mmHg, 40--150 & escasos & >0 & \textbf{DUPLICADA}: requiere aclaración \\ \hline
\makecell[l]{frecuencia\\cardiaca} & Frecuencia cardíaca & numérico & latidos/min, 30--200 & escasos & >0 & Ninguna \\ \hline
shock & Presencia de shock & booleano & 0, 1 & presentes & -- & Requiere tipo de shock (cardiogénico, etc.) \\ \hline
indice\_mkillip & Clasificación Killip modificada & categórico & I, II, III, IV (romanos) & presentes & -- & Requiere explicación de diferencia con Killip \\ \hline
indice\_killip & Clasificación Killip & categórico & I, II, III, IV (romanos) & presentes & -- & Requiere explicación de diferencia con Killip-M \\ \hline
\makecell[l]{ingresos\\anteriores} & Ingresos previos por IAM & booleano & 0, 1 & presentes & -- & Requiere codificación binaria \\ \hline
\makecell[l]{diabetes\\mellitus} & Diabetes mellitus previa & booleano & 0, 1 & ninguno & -- & Ninguna \\ \hline
insulina & Tratamiento con insulina & numérico & (valores diversos) & muchos & -- & Requiere aclaración: ¿dosis, tipo, booleano? \\ \hline
\makecell[l]{insuficiencia\\cardiaca\\congestiva} & Insuficiencia cardíaca congestiva previa & booleano & 0, 1 & ninguno & -- & Ninguna \\ \hline
\makecell[l]{insuficiencia\\cardiaca} & Insuficiencia cardíaca (evento) & booleano & 0, 1 & muchos & -- & Diferencia con ICC como antecedente \\ \hline
\makecell[l]{hipertension\\arterial} & Hipertensión arterial previa & booleano & 0, 1 & ninguno & -- & Ninguna \\ \hline
\makecell[l]{hiperlipo-\\proteinemia} & Dislipidemia previa & booleano & 0, 1 & ninguno & -- & Ninguna \\ \hline
\makecell[l]{enfermedad\\arterias\\coronarias} & Enfermedad arterial coronaria previa & booleano & 0, 1 & ninguno & -- & Ninguna \\ \hline
infarto\_miocardio\_agudo & Infarto de miocardio previo & booleano & 0, 1 & ninguno & -- & Ninguna \\ \hline
fibrilacion\_auricular & Fibrilación auricular previa & booleano & 0, 1 & ninguno & -- & Ninguna \\ \hline
\makecell[l]{intervencion\\coronaria\\percutanea} & ICP previa & booleano & 0, 1 & ninguno & -- & Ninguna \\ \hline
cabg & Cirugía de revascularización coronaria previa (CABG) & booleano & 0, 1 & ninguno & -- & Requiere aclaración del acrónimo CABG \\ \hline
\makecell[l]{enfermedad\\venosa\\periferica} & Enfermedad venosa periférica & booleano & 0, 1 & ninguno & -- & Ninguna \\ \hline
\makecell[l]{insuficiencia\\renal\\cronica} & Insuficiencia renal crónica & booleano & 0, 1 & ninguno & -- & Ninguna \\ \hline
dialisis & Tratamiento con diálisis & booleano & 0, 1 & ninguno & -- & Ninguna \\ \hline
\makecell[l]{enfermedad\\cerebro\\vascular} & Enfermedad cerebrovascular previa & booleano & 0, 1 & ninguno & -- & Ninguna \\ \hline
anemia & Anemia previa o actual & booleano & 0, 1 & ninguno & -- & Ninguna \\ \hline
epoc & Enfermedad pulmonar obstructiva crónica & booleano & 0, 1 & ninguno & -- & Ninguna \\ \hline
otros & Otros antecedentes & booleano & 0, 1 & presentes & -- & Requiere especificación de contenido \\ \hline
otras & Otras complicaciones o condiciones & booleano & 0, 1 & muchos & -- & Requiere especificación de contenido \\ \hline
tabaquismo & Antecedente de tabaquismo & booleano & 0, 1 & ninguno & -- & Ninguna \\ \hline
\makecell[l]{tipo\\tabaquismo} & Tipo de tabaquismo & categórico & activo, noactivo & presentes & -- & Requiere codificación binaria \\ \hline
\makecell[l]{annos\\fumando} & Años de consumo de tabaco & numérico & años, 0--80 & presentes & >=0 & Ninguna \\ \hline
\makecell[l]{annos\\sin\\fumar} & Años sin fumar (ex-fumadores) & numérico & años, 0--80 & muchos & >=0 & Ninguna \\ \hline
asa & Ácido acetilsalicílico (aspirina) & numérico & (valores diversos) & presentes & -- & \textbf{DUPLICADA}: ¿dosis o booleano? \\ \hline
betabloqueadores & Betabloqueadores & numérico & (valores diversos) & presentes & -- & \textbf{DUPLICADA}: requiere escala o codificación \\ \hline
clopidogrel & Clopidogrel & categórico & si, no & presentes & -- & \textbf{DUPLICADA}: requiere codificación binaria \\ \hline
heparina & Heparina & categórico & si, no & presentes & -- & Requiere codificación binaria \\ \hline
estatinas & Estatinas (hipolipemiantes) & categórico & si, no & presentes & -- & \textbf{DUPLICADA}: requiere codificación binaria \\ \hline
furosemida & Furosemida (diurético de asa) & categórico & si, no & presentes & -- & \textbf{DUPLICADA}: requiere codificación binaria \\ \hline
nitratos & Nitratos & categórico & si, no & presentes & -- & \textbf{DUPLICADA}: requiere codificación binaria \\ \hline
anticoagulantes & Anticoagulantes & categórico & si, no & presentes & -- & \textbf{DUPLICADA}: requiere codificación binaria \\ \hline
anticalcicos & Antagonistas del calcio & categórico & si, no & presentes & -- & Requiere codificación binaria \\ \hline
anticalcico & Antagonista del calcio (¿duplicado?) & booleano & 0, 1 & presentes & -- & Posible duplicación de anticalcicos \\ \hline
ieca & Inhibidores de ECA & categórico & si, no & presentes & -- & \textbf{DUPLICADA}: requiere codificación binaria \\ \hline
\makecell[l]{otros\\diureticos} & Otros diuréticos & categórico & si, no & presentes & -- & \textbf{DUPLICADA}: requiere codificación binaria \\ \hline
\makecell[l]{estreptoquinasa\\recombinante} & Administración de estreptoquinasa & categórico & si, no & presentes & -- & Requiere codificación binaria \\ \hline
\makecell[l]{tiempo\\puerta\\aguja} & Tiempo puerta-aguja & numérico & probablemente minutos & presentes & >=0 & Requiere confirmación de unidad y definición \\ \hline
\makecell[l]{tiempo\\isquemia} & Tiempo de isquemia & numérico & probablemente minutos & presentes & >=0 & Requiere confirmación de unidad \\ \hline
\makecell[l]{escala\\grace} & Puntaje de riesgo GRACE & numérico & 0--372 & presentes & -- & Requiere validación de rango y significado \\ \hline
reperfusion & Tipo de reperfusión realizada & categórico & no, parcial, total, otro & presentes & -- & Requiere codificación y explicación detallada \\ \hline
coronariografia & Coronariografía realizada & categórico & no, si, otro, centro & presentes & -- & Requiere codificación y explicación \\ \hline
lugar\_trombolisis & Lugar de trombolisis & categórico & ucie, sala, servicio, otro & presentes & -- & Candidata a eliminación según análisis \\ \hline
ergometria & Prueba de esfuerzo (ergometría) & categórico & positiva, negativa, no & muchos & -- & Requiere dominio completo y codificación \\ \hline
avc & Asistencia ventricular o complicación & booleano & 0, 1 & ninguno & -- & Requiere aclaración del acrónimo \\ \hline
mpt & Marcapaso temporal & booleano & 0, 1 & ninguno & -- & Requiere confirmación del acrónimo \\ \hline
vam & Ventilación asistida mecánica & booleano & 0, 1 & ninguno & -- & Requiere confirmación del acrónimo \\ \hline
mpp & Marcapaso permanente & booleano & 0, 1 & ninguno & -- & Requiere confirmación del acrónimo \\ \hline
aminas & Uso de aminas vasoactivas & booleano & 0, 1 & ninguno & -- & Ninguna \\ \hline
balon & Balón de contrapulsación intraaórtico & booleano & 0, 1 & ninguno & -- & Ninguna \\ \hline
proxima & Próxima consulta o evento & numérico & enteros & muchos & -- & Requiere explicación del significado \\ \hline
\makecell[l]{fecha\\consulta} & Fecha de consulta de seguimiento & fecha & formato dd/mm/yyyy & muchos & fecha >= fecha\_egreso & Ninguna \\ \hline
resultado & Resultado del seguimiento & categórico & vivo\_sin, vivo\_con, noevaluado, alta, fallecido & muchos & -- & Requiere explicación de categorías \\ \hline
fecha\_defuncion & Fecha de defunción & fecha & formato dd/mm/yyyy & muchos & >= fecha\_ingreso & Ninguna \\ \hline
motivo & Motivo de reingreso & categórico & scacest (principalmente) & muchos & -- & Asociada a fecha\_ingreso duplicada \\ \hline
\makecell[l]{fecha\\realizacion} & Fecha de realización de procedimiento & fecha & formato dd/mm/yyyy & muchos & -- & Requiere especificación de procedimiento \\ \hline
\makecell[l]{razones\\documentadas} & Razones documentadas & booleano & 0, 1 & muchos & -- & Requiere contexto: ¿razones de qué? \\ \hline
\makecell[l]{riesgo\\beneficio} & Evaluación riesgo-beneficio & booleano & 0, 1 & muchos & -- & Requiere contexto específico \\ \hline
\makecell[l]{anti\\agregacion\\plaquetaria} & Antiagregación plaquetaria & booleano & 0, 1 & muchos & -- & Requiere aclaración del contexto \\ \hline
\makecell[l]{proteccion\\embolica} & Protección embólica & booleano & 0, 1 & muchos & -- & Requiere aclaración del contexto \\ \hline
\makecell[l]{funcion\\renal} & Evaluación de función renal & booleano & 0, 1 & muchos & -- & Requiere aclaración del contexto \\ \hline
\makecell[l]{volumen\\contraste} & Volumen de contraste utilizado & booleano & 0, 1 & muchos & -- & Requiere aclaración: ¿adecuado/inadecuado? \\ \hline
\makecell[l]{prescripcion\\optima} & Prescripción óptima al egreso & booleano & 0, 1 & muchos & -- & Requiere definición de "óptima" \\ \hline
rehabilitacion & Rehabilitación cardíaca & booleano & 0, 1 & muchos & -- & \textbf{DUPLICADA}: aparece dos veces \\ \hline
\makecell[l]{participacion\\registro} & Participación en registro & booleano & 0, 1 & muchos & -- & Requiere aclaración del propósito \\ \hline
\makecell[l]{coronariografias\\medico} & Coronariografías por médico & booleano & 0, 1 & muchos & -- & Requiere aclaración del significado \\ \hline
\makecell[l]{coronariografias\\centro} & Coronariografías por centro & booleano & 0, 1 & muchos & -- & Requiere aclaración del significado \\ \hline
idresultado & Identificador de resultado & numérico & enteros & muchos & -- & Requiere explicación del sistema de ID \\ \hline
arteria & Arteria afectada & categórico & cd, cx, ada, otros & muchos & -- & Requiere tabla completa de códigos \\ \hline
localizacion & Localización de lesión & categórico & proximal, media, distal, sin\_lesiones & muchos & -- & Ninguna \\ \hline
estenosis & Grado de estenosis & numérico & porcentaje, 0--100 & muchos & 0--100 & Ninguna \\ \hline
abordaje & Tipo de abordaje & categórico & stent\_farmaco, stent\_metalico, ninguno & muchos & -- & Requiere tabla completa de valores \\ \hline
\makecell[l]{fecha\\egreso} & Fecha de egreso hospitalario & fecha & formato dd/mm/yyyy & presentes & >= fecha\_ingreso & \textbf{DUPLICADA}: requiere aclaración \\ \hline
\makecell[l]{estado\\vital} & Estado vital al egreso & categórico & vivo, fallecido & presentes & -- & Requiere codificación binaria \\ \hline
\makecell[l]{otra\\institucion} & Traslado a otra institución & categórico & si, no & muchos & -- & Requiere codificación binaria \\ \hline
\makecell[l]{estadia\\ucie} & Estadía en UCIE & numérico & días & presentes & >=0 & Ninguna (unidad confirmada: días) \\ \hline
\makecell[l]{estadia\\uci} & Estadía en UCI & numérico & días & presentes & >=0 & Ninguna (unidad confirmada: días) \\ \hline
\makecell[l]{estadia\\intrahospitalaria} & Estadía intrahospitalaria total & numérico & días & presentes & >=0 & Ninguna (unidad confirmada: días) \\ \hline
\makecell[l]{consejeria\\antitabaquica} & Consejería antitabáquica brindada & booleano & 0, 1 & presentes & -- & Ninguna \\ \hline
consejeria & Consejería general brindada & booleano & 0, 1 & presentes & -- & Requiere especificación del tipo \\ \hline
dieta & Consejería dietética brindada & booleano & 0, 1 & presentes & -- & Ninguna \\ \hline
complicaciones & Complicaciones durante hospitalización & categórico & valores diversos & muchos & -- & Requiere tabla de códigos completa \\ \hline
observaciones & Observaciones clínicas libres & texto libre & texto único por caso & muchos & -- & Ninguna \\ \hline
colesterol & Colesterol total & numérico & mg/dL o mmol/L & muchos & >0 & Requiere confirmación de unidad \\ \hline
creatinina & Creatinina sérica & numérico & mg/dL o $\mu$mol/L & muchos & >0 & Requiere confirmación de unidad \\ \hline
\makecell[l]{filtrado\\glomerular} & Tasa de filtrado glomerular estimado & numérico & mL/min/1.73m² & muchos & >0 & Requiere confirmación de unidad \\ \hline
trigliceridos & Triglicéridos & numérico & mg/dL o mmol/L & muchos & >0 & Requiere confirmación de unidad \\ \hline
glicemia & Glucemia & numérico & mg/dL o mmol/L & muchos & >0 & Requiere confirmación de unidad \\ \hline
leuco & Leucocitos & numérico & células/mm³ o x10³/$\mu$L & muchos & >0 & Requiere confirmación de unidad \\ \hline
hb & Hemoglobina & numérico & g/dL & muchos & >0 & Requiere confirmación de unidad \\ \hline
ck & Creatina quinasa & numérico & U/L & muchos & >0 & Requiere confirmación de unidad \\ \hline
ckmb & Creatina quinasa MB & numérico & U/L o ng/mL & muchos & >0 & Requiere confirmación de unidad \\ \hline
acd & Arteria coronaria derecha & booleano & 0, 1 & presentes & -- & Requiere confirmación del acrónimo \\ \hline
ada & Arteria descendente anterior & booleano & 0, 1 & presentes & -- & Requiere confirmación del acrónimo \\ \hline
acx & Arteria circunfleja & booleano & 0, 1 & presentes & -- & Requiere confirmación del acrónimo \\ \hline
\makecell[l]{fraccion\\eyeccion} & Fracción de eyección del VI & numérico & porcentaje, 0--100 & muchos & 0--100 & Ninguna \\ \hline
ddvi & Diámetro diastólico del VI & numérico & mm & muchos & >0 & Requiere confirmación de unidad \\ \hline
dsvi & Diámetro sistólico del VI & numérico & mm & muchos & >0 & Requiere confirmación de unidad \\ \hline
tiv & Grosor del tabique interventricular & numérico & mm & muchos & >0 & Requiere confirmación de unidad \\ \hline
pp & Grosor de pared posterior & numérico & mm & muchos & >0 & Requiere confirmación de unidad \\ \hline
ud & Parámetro ecocardiográfico (desconocido) & numérico & (unidad desconocida) & muchos & >0 & Requiere aclaración completa \\ \hline
fs & Fracción de acortamiento & numérico & porcentaje, 0--100 & muchos & 0--100 & Requiere confirmación \\ \hline
tapse & TAPSE (función ventricular derecha) & numérico & mm & muchos & >0 & Requiere confirmación de unidad \\ \hline
ea & Relación E/A mitral (función diastólica) & numérico & ratio & muchos & >0 & Requiere confirmación \\ \hline
ee & Relación E/e' (función diastólica) & numérico & ratio & muchos & >0 & Requiere confirmación \\ \hline
pat & Parámetro ecocardiográfico (desconocido) & numérico & (unidad desconocida) & muchos & >0 & Requiere aclaración completa \\ \hline
insao & Insuficiencia aórtica & booleano & 0, 1 & muchos & -- & Requiere aclaración del acrónimo \\ \hline
estao & Estenosis aórtica & booleano & 0, 1 & muchos & -- & Requiere aclaración del acrónimo \\ \hline
insmit & Insuficiencia mitral & booleano & 0, 1 & muchos & -- & Requiere aclaración del acrónimo \\ \hline
estmit & Estenosis mitral & booleano & 0, 1 & muchos & -- & Requiere aclaración del acrónimo \\ \hline
\makecell[l]{mortality\\inhospital} & Mortalidad intrahospitalaria & booleano & 0, 1 & ninguno & -- & \textbf{Variable objetivo del modelo} \\ \hline
\end{longtable}

\renewcommand{\arraystretch}{1.2}
\normalsize

\section{Registro de dudas y resoluciones}
% Use una fila por duda específica. Mantenga el "Estado" actualizado.
\scriptsize
\setlength{\LTleft}{\fill}
\setlength{\LTright}{\fill}
\setlength{\tabcolsep}{4pt}
\renewcommand{\arraystretch}{1.15}

% Línea más visible entre celdas
\setlength{\arrayrulewidth}{0.6pt}

\begin{longtable}{|L{0.035\textwidth}|L{0.10\textwidth}|L{0.265\textwidth}|L{0.17\textwidth}|L{0.05\textwidth}|L{0.05\textwidth}|L{0.095\textwidth}|L{0.065\textwidth}|L{0.075\textwidth}|}
\caption{Registro de dudas y resoluciones.}\\
\hline
\textbf{ID} & \textbf{Variable} & \textbf{Duda / Observación} & \textbf{Evidencia / Contexto} & \textbf{Impacto} & \textbf{Prioridad} & \textbf{Responsable (consulta a)} & \textbf{Estado} & \textbf{Fechas (sol./cierre)} \\ \hline
\endfirsthead

\multicolumn{9}{c}{\small\itshape Continuación del Registro de dudas y resoluciones (encabezado repetido)}\\[3pt]
\hline
\textbf{ID} & \textbf{Variable} & \textbf{Duda / Observación} & \textbf{Evidencia / Contexto} & \textbf{Impacto} & \textbf{Prioridad} & \textbf{Responsable (consulta a)} & \textbf{Estado} & \textbf{Fechas (sol./cierre)} \\ \hline
\endhead

\hline
\multicolumn{9}{r}{\small\itshape Continua en la siguiente página} \\ \hline
\endfoot

\hline
\endlastfoot
% ---- Dudas prioritarias sobre el dataset ----
001 & Variables duplicadas (grupo) & Se identifican 12 variables que aparecen duplicadas en el dataset. Se requiere aclaración sobre el significado de cada columna duplicada y cuál utilizar para el análisis. & Variables afectadas: presion\_arterial\_sistolica, presion\_arterial\_diastolica, asa, betabloqueadores, ieca, estatinas, clopidogrel, furosemida, nitratos, anticoagulantes, otros\_diureticos, fecha\_ingreso. Posibles hipótesis: (a) medición en momentos diferentes (ingreso vs egreso), (b) fusión incorrecta de datasets, (c) prescripción vs administración real & Alto & Alta & Cardiólogo & Pendiente & 2025-10-25 / \\ \hline
002 & asa & ¿La variable ASA representa dosis en mg, días de tratamiento, o simplemente presencia/ausencia? & Valores observados son numéricos diversos (rango 0--24). No se identifica patrón claro de dosificación estándar & Alto & Alta & Cardiólogo & Pendiente & 2025-10-25 / \\ \hline
003 & betablo-
queadores & ¿Qué escala o codificación se utiliza para betabloqueadores? & Valores numéricos diversos (rango 0--36). No se identifica si es dosis, días, o código categórico & Alto & Alta & Cardiólogo & Pendiente & 2025-10-25 / \\ \hline
004 & ecg & Solicitar tabla de correspondencia completa para códigos ECG & Valores observados: 5, 10, 12, 15, 20, 25, 30, 35. Sin documentación disponible sobre significado clínico & Alto & Alta & Cardiólogo & Pendiente & 2025-10-25 / \\ \hline
005 & scacest, scacest\_ 
secundario & Requiere explicación clínica detallada de la diferencia entre SCACEST primario y secundario & Ambas variables booleanas (0/1), pero no se comprende la distinción clínica ni criterios diagnósticos & Alto & Alta & Cardiólogo & Pendiente & 2025-10-25 / \\ \hline
006 & indice\_killip, indice\_mkillip & ¿Cuál es la diferencia entre clasificación Killip y Killip modificada? ¿Cuándo se utiliza cada una? & Ambas utilizan números romanos I--IV. En algunos registros difieren, en otros coinciden & Medio & Alta & Cardiólogo & Pendiente & 2025-10-25 / \\ \hline
007 & tiempo
puerta\_aguja & Confirmar definición clínica exacta y unidad de medida & Asumimos minutos desde llegada hospitalaria hasta inicio de trombolisis, pero requiere confirmación oficial & Alto & Alta & Cardiólogo & Pendiente & 2025-10-25 / \\ \hline
008 & tiempo
isquemia & Confirmar definición clínica exacta y unidad de medida & Asumimos minutos desde inicio de síntomas hasta reperfusión, pero requiere confirmación oficial & Alto & Alta & Cardiólogo & Pendiente & 2025-10-25 / \\ \hline
009 & tiempo
respuesta, tiempo\_llegada & Confirmar unidades de medida y definiciones exactas de cada variable & Valores presentes solo cuando llamada\_emergencias=si. Asumimos minutos pero sin documentación & Medio & Media & Cardiólogo & Pendiente & 2025-10-25 / \\ \hline
010 & escala\_grace & Validar rango de valores observados contra rango teórico esperado (0--372) & Valores parecen consistentes pero requiere validación clínica del cálculo & Medio & Media & Cardiólogo & Pendiente & 2025-10-25 / \\ \hline
011 & Derivaciones ECG (V1--V9, D1--D3, AVL, AVF, AVR, V3R, V4R) & ¿Estas variables indican presencia de alteración en cada derivación? ¿Qué tipo de alteración (supradesnivel, infradesnivel, onda Q)? & Variables booleanas (0/1) con muchos faltantes. Requiere documentación de criterios diagnósticos utilizados & Alto & Alta & Cardiólogo & Pendiente & 2025-10-25 / \\ \hline
012 & avc, mpt, vam, mpp & Confirmar significado exacto de acrónimos & Asumimos: AVC=asistencia ventricular, MPT=marcapaso temporal, VAM=ventilación mecánica, MPP=marcapaso permanente & Medio & Media & Cardiólogo & Pendiente & 2025-10-25 / \\ \hline
013 & cabg & Confirmar acrónimo CABG (¿Coronary Artery Bypass Graft?) & Asumimos cirugía de revascularización coronaria pero requiere confirmación & Bajo & Baja & Cardiólogo & Pendiente & 2025-10-25 / \\ \hline
014 & reperfusion & Explicar diferencia entre categorías y criterios de clasificación & Valores: no, parcial, total, otro. Requiere definición de criterios clínicos utilizados & Alto & Alta & Cardiólogo & Pendiente & 2025-10-25 / \\ \hline
015 & coronario-
grafia & Explicar diferencia entre valores observados & Valores: no, si, otro, centro. Especialmente aclarar significado de "centro" vs "otro" & Medio & Media & Cardiólogo & Pendiente & 2025-10-25 / \\ \hline
016 & Grupo variables de calidad & Grupo de variables con muchos faltantes que requieren contexto clínico & Variables: razones\_documentadas, riesgo\_beneficio, anti\_agregacion\_plaquetaria, proteccion\_embolica, funcion\_renal, volumen\_contraste, prescripcion\_optima. ¿Son indicadores de calidad o checklist? & Medio & Media & Cardiólogo & Pendiente & 2025-10-25 / \\ \hline
017 & Laboratorio (grupo) & Confirmar unidades de medida para todas las variables de laboratorio & Variables: colesterol, creatinina, filtrado\_glomerular, trigliceridos, glicemia, leuco, hb, ck, ckmb. Especificar unidades (mg/dL, mmol/L, etc.) & Alto & Alta & Laboratorio & Pendiente & 2025-10-25 / \\ \hline
018 & Ecocardio-
grafía (grupo) & Confirmar significado de acrónimos: ud, pat, insao, estao, insmit, estmit & Asumimos insuficiencias y estenosis valvulares. ud y pat sin identificar & Medio & Media & Cardiología & Pendiente & 2025-10-25 / \\ \hline
019 & arteria & Solicitar tabla completa de códigos de arterias coronarias & Códigos: cd, cx, ada. Asumimos: CD=coronaria derecha, CX=circunfleja, ADA=descendente anterior & Medio & Media & Cardiólogo & Pendiente & 2025-10-25 / \\ \hline
020 & abordaje & Solicitar tabla completa de tipos de abordaje/intervención coronaria & Valores: stent\_farmaco, stent\_metalico, ninguno. ¿Existen otros valores posibles? & Medio & Media & Cardiólogo & Pendiente & 2025-10-25 / \\ \hline
021 & resultado & Explicar categorías de seguimiento y su significado clínico & Valores: vivo\_sin, vivo\_con, noevaluado, alta, fallecido. Requiere explicación de "vivo\_sin" vs "vivo\_con" (¿con/sin complicaciones?) & Medio & Media & Cardiólogo & Pendiente & 2025-10-25 / \\ \hline
022 & angina24h & ¿Se refiere a angina en 24h previas al ingreso o durante primeras 24h de hospitalización? & Variable booleana con contexto temporal ambiguo que afecta interpretación clínica & Bajo & Baja & Cardiólogo & Pendiente & 2025-10-25 / \\ \hline
023 & provincia, municipio, area\_salud & Normalizar caracteres especiales en nombres geográficos & Caracteres especiales mal codificados: Sancti Sp\'iritus, Cabaigu\'an, Camag\"uey, Manat\'i, G\"uines & Bajo & Baja & Admin. datos & Pendiente & 2025-10-25 / \\ \hline
024 & insulina & ¿Variable representa dosis, tipo de insulina, duración de tratamiento, o uso binario? & Valores numéricos diversos sin patrón identificable. Dificulta su uso en modelado & Medio & Media & Cardiólogo & Pendiente & 2025-10-25 / \\ \hline
025 & lugar
trombolisis & ¿Es necesaria esta variable para el objetivo analítico del proyecto? & Valores: ucie, sala, servicio. Considerada candidata a eliminación según relevancia clínica & Bajo & Baja & Equipo ML & Pendiente & 2025-10-25 / \\ \hline
\end{longtable}
\renewcommand{\arraystretch}{1.2}
\normalsize


\section*{Acciones siguientes y próximos pasos}

\subsection*{Consultas prioritarias }
\begin{enumerate}
    \item \textbf{Duplicación de variables (Alta prioridad):} Aclarar el significado y uso correcto de las 12 variables que aparecen duplicadas en el dataset. Esta es la duda más crítica que afecta la calidad del análisis.
    \item \textbf{Escalas y codificaciones de medicamentos:} Definir las escalas utilizadas para asa, betabloqueadores, insulina y confirmar codificaciones de otros medicamentos.
    \item \textbf{Variables electrocardiográficas:} Proporcionar tabla de correspondencia para códigos ECG y documentar criterios para derivaciones (V1--V9, D1--D3, AVL, AVF, AVR, V3R, V4R).
    \item \textbf{Variables de tiempo:} Confirmar definiciones clínicas exactas y unidades de medida para tiempo\_puerta\_aguja, tiempo\_isquemia, tiempo\_respuesta, tiempo\_llegada.
    \item \textbf{Clasificaciones clínicas:} Explicar diferencias entre SCACEST primario/secundario, Killip/Killip modificado, y categorías de reperfusión.
    \item \textbf{Acrónimos y términos técnicos:} Confirmar significado de acrónimos no documentados (avc, mpt, vam, mpp, cabg, acd, ada, acx, insao, estao, insmit, estmit, ud, pat).
    \item \textbf{Tablas de códigos:} Solicitar tablas completas para arteria, abordaje, resultado, complicaciones.
    \item Especificar unidades de medida para todas las variables de laboratorio (colesterol, creatinina, filtrado glomerular, triglicéridos, glicemia, leucocitos, hemoglobina, CK, CK-MB).
    \item Confirmar rangos de referencia y límites de detección de los equipos utilizados.
\end{enumerate}



\subsection*{Tareas del equipo de machine learning}
\begin{enumerate}
    \item Mantener actualizado el "Registro de dudas y resoluciones" con estados y fechas.
    \item Documentar todas las decisiones de preprocesamiento tomadas en ausencia de aclaraciones.
    \item Priorizar variables según impacto en el objetivo analítico (predicción de mortalidad intrahospitalaria).
    \item Preparar pipeline de preprocesamiento flexible que permita incorporar aclaraciones posteriores.
    \item Normalizar caracteres especiales en nombres geográficos.
\end{enumerate}


\subsection*{Nota sobre el proceso de limpieza inicial}
Este dataset es resultado de una primera etapa de limpieza en la que se eliminaron:
\begin{itemize}
    \item Variables de identificación personal (nombre, primer\_apellido, segundo\_apellido, numero\_identidad, numero\_contacto) para cumplir con protección de datos.
    \item Variables redundantes identificadas en análisis preliminar (anno, numero como identificador, unidad).
\end{itemize}

El dataset original parece ser resultado de la fusión de múltiples fuentes o registros hospitalarios, lo cual explicaría algunas de las duplicaciones observadas. Se requiere confirmación de esta hipótesis por parte del equipo responsable de la recolección de datos.

\end{document}